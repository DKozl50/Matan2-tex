\documentclass[a4paper, fleqn]{article}
\usepackage{header}

\title{Коллоквиум по Дискретной Математике}
\author{
    % Откомментируй себя или добавь в список!

    % Александр Богданов   \\ \href{https://t.me/SphericalPotatoInVacuum}{Telegram} \and
    % Алиса Вернигор       \\ \href{https://t.me/allisyonok}{Telegram} \and
    % Анастасия Григорьева \\ \href{https://t.me/weifoll}{Telegram} \and
    % Василий Шныпко       \\ \href{https://t.me/yourvash}{Telegram} \and
    % Данил Казанцев       \\ \href{https://t.me/vserosbuybuy}{Telegram} \and
    Денис Козлов         \\ \href{https://t.me/DKozl50}{Telegram} \and
    % Елизавета Орешонок   \\ \href{https://t.me/eaoresh}{Telegram} \and
    % Ира Голобородько     \\ \href{https://t.me/Ira4kgl}{Telegram} \and
    % Иван Пешехонов       \\ \href{https://t.me/JohanDDC}{Telegram} \and
    % Иван Добросовестнов  \\ \href{https://t.me/ivankot13}{Telegram} \and
    % Настя Городилова     \\ \href{https://t.me/nastygorodi}{Telegram} \and
    % Никита Насонков      \\ \href{https://t.me/nnv_nick}{Telegram} \and
    % Сергей Лоптев        \\ \href{https://t.me/beast_sl}{Telegram}
}

\date{Версия от {\ddmmyyyydate\today} \currenttime}

% После комментария с номером пишем формулировку вопроса в subsection*{...}
% К примеру:

% Вопрос 1
% \subsection*{Перечислимые множества суть, в точности, проекции разрешимых.}
% и тут уже решение формулировочки все дела

\begin{document}
    \maketitle

    \section*{Вычислимость}


    % Вопрос вычислимость 1

    % Вопрос вычислимость 2

    % Вопрос вычислимость 3

    % Вопрос вычислимость 4

    % Вопрос вычислимость 5

    % Вопрос вычислимость 6

    % Вопрос вычислимость 7

    % Вопрос вычислимость 8

    % Вопрос вычислимость 9

    % Вопрос вычислимость 10

    % Вопрос вычислимость 11

    % Вопрос вычислимость 12

    % Вопрос вычислимость 13

    % Вопрос вычислимость 14

    % Вопрос вычислимость 15
    
    \subsection*{Исчисление резолюций для произвольных множеств формул. Теорема о корректности. Теорема о полноте (без доказательства).}

    В исчислении резолюций есть 2 правила.


    \underline{Правило вывода (резолюции)}. $\boxed{\frac{\neg A \lor B \; \; \; A \lor C}{B \lor C} } $

    Частный случай ПР -- вывод пустой дизъюнкции (лжи): $\frac{A \; \; \; \neg A}{\perp}.$



    \underline{Правило подстановки.} $\boxed{\frac{\forall y A}{A(t/y)}} \; $, где $t$ -- произвольный терм, $A$ -- универсальна.


    Говорят, что формула $\varphi$ \underline{выводится} из множества  формул $\Gamma$ в исчислении резолюций, если $\varphi$ возможно получить из формул в $\Gamma$, применяя к ним правило резолюции и правило подстановки какое-то конечное число раз.

    Обозначение: $\Gamma \vdash \varphi.$

    \underline{Корректность ПР}. $\forall M \; \forall \pi$ если $[\neg A \lor B]_M (\pi) = 1$  и $[ A \lor C]_M (\pi) = 1,$ то $[B \lor C]_M (\pi) = 1.$

    Другими словами, импликация $( (\neg A \lor B) \land (A \lor C) \to (B \lor C) )$ общезначима.

    \begin{proof} По контрапозиции.

    $[B \lor C]_M (\pi) = 0 \implies \left([B]_M (\pi) = 0\right) \land \left([C]_M (\pi) = 0\right) \implies [\neg A \lor B]_M (\pi) = 0 \lor   [ A \lor C]_M (\pi) = 0 \implies$

    $ \neg([\neg A \lor B]_M (\pi) = 1 \land   [ A \lor C]_M (\pi) = 1).$
    \end{proof}


    \underline{Корректность ПП.} Пусть $A$ -- универсальна.

    $\forall M \; \forall \pi$ если $[ \, \forall y \,  A \, ]_M (\pi) = 1,$  то $[ \, A(t/y) \, ]_M (\pi) = 1.$

    \begin{proof}
    Допустим, $A \eqcirc \forall \, \overrightarrow{x} A_0$, где $A_0$ -- бескванторная формула.

    $[\forall \, y \, \forall \, \overrightarrow{x} \, A_0 ]_M (\pi) = 1 \iff \forall \, a \in M \, \forall  \overrightarrow{b} \in M^n \;  \; [A_0] \left(\pi^{a \; \; \overrightarrow{b}}_{y \; \; \overrightarrow{x}} \right) = 1 \iff \forall  \overrightarrow{b} \in M^n \;   \forall \, a \in M  \;  \; [A_0] \left(\pi^{\overrightarrow{b} \; \; a}_{\overrightarrow{x} \; \; y} \right) = 1.$

    Теперь положим $a := [ \, t \,]  \left(\pi_{\overrightarrow{x}}^{\overrightarrow{b}} \right) \implies \forall \overrightarrow{b} \in M^n \; [A_0] \left( \pi_{\overrightarrow{x} \; \; \, y}^{\overrightarrow{b} \; \; \; [ \, t \,]  \left(\pi_{\overrightarrow{x}}^{\overrightarrow{b}} \right)} \right) = 1. \; \; \circledast$

    С другой стороны $[A(t/y)]_M(\pi) = 1 \iff [ \, \forall \overrightarrow{x} \; A_0 (t/y) \,  ]_M (\pi) = 1 \iff \forall \overrightarrow{b} \in M^n \; [ \, A_0(t/y) \, ]_M \left(\pi_{\overrightarrow{x}}^{\overrightarrow{b}} \right) = 1.$

    Так как в $A_0$  нет кванторов, $t-y-A_0.$ Тогда по лемме о корректной подстановке

    $\forall \overrightarrow{b} \in M^n \; [ \, A_0(t/y) \, ]_M \left(\pi_{\overrightarrow{x}}^{\overrightarrow{b}} \right) = 1 \iff \forall \overrightarrow{b} \in M^n \; [A_0] \left( \pi_{\overrightarrow{x} \; \; \, y}^{\overrightarrow{b} \; \; \; [ \, t \,]  \left(\pi_{\overrightarrow{x}}^{\overrightarrow{b}} \right)} \right) = 1,$ что совпадает с  $\circledast.$ Равносильность доказана.

    \end{proof}

    \underline{Теорема о корректности.} Если $\Gamma \vdash \varphi,$ то $\Gamma \vDash \varphi.$

    \begin{proof} Индукция по построению вывода в ИР. 

    Применяем утверждения о корректности ПП и ПР, доказанные выше.
    \end{proof}

    \underline{Теорема о полноте}. $\Delta$ -- набор формул в сигнатуре $\sigma$. Если $\Delta \nvdash \perp$, то $\exists M \, \exists \pi \; [\Delta]_M (\pi) = 1,$ причём $|M| \leq max(|\mathbb{N}|, \, | \sigma|). $


    % Вопрос вычислимость 16

    % Вопрос вычислимость 17

    % Вопрос вычислимость 18

    % Вопрос вычислимость 19

    % Вопрос вычислимость 20

    % Вопрос вычислимость 21

    % Вопрос вычислимость 22

    % Вопрос вычислимость 23


    \section*{Логика}


    % Вопрос логика 1

    % Вопрос логика 2

    % Вопрос логика 3

    % Вопрос логика 4

    % Вопрос логика 5

    % Вопрос логика 6

    % Вопрос логика 7

    % Вопрос логика 8

    % Вопрос логика 9

    % Вопрос логика 10

    % Вопрос логика 11

    % Вопрос логика 12

    % Вопрос логика 13

    % Вопрос логика 14

    % Вопрос логика 15

    % Вопрос логика 16

    % Вопрос логика 17

    % Вопрос логика 18

    % Вопрос логика 19

    % Вопрос логика 20

    % Вопрос логика 21

    % Вопрос логика 22

    % Вопрос логика 23

\end{document}
