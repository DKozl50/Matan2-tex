\documentclass[a4paper, fleqn]{article}
\usepackage{header}

\title{Коллоквиум по Дискретной Математике}
\author{
    % Откомментируй себя или добавь в список!

    % Александр Богданов   \\ \href{https://t.me/SphericalPotatoInVacuum}{Telegram} \and
    % Алиса Вернигор       \\ \href{https://t.me/allisyonok}{Telegram} \and
    % Анастасия Григорьева \\ \href{https://t.me/weifoll}{Telegram} \and
    % Василий Шныпко       \\ \href{https://t.me/yourvash}{Telegram} \and
    % Данил Казанцев       \\ \href{https://t.me/vserosbuybuy}{Telegram} \and
    Денис Козлов         \\ \href{https://t.me/DKozl50}{Telegram} \and
    % Елизавета Орешонок   \\ \href{https://t.me/eaoresh}{Telegram} \and
    % Ира Голобородько     \\ \href{https://t.me/Ira4kgl}{Telegram} \and
    % Иван Пешехонов       \\ \href{https://t.me/JohanDDC}{Telegram} \and
    % Иван Добросовестнов  \\ \href{https://t.me/ivankot13}{Telegram} \and
    % Настя Городилова     \\ \href{https://t.me/nastygorodi}{Telegram} \and
    % Никита Насонков      \\ \href{https://t.me/nnv_nick}{Telegram} \and
    % Сергей Лоптев        \\ \href{https://t.me/beast_sl}{Telegram} \and
    Оля Козлова \\ \href{https://t.me/grenlayk}{Telegram}
}

\date{Версия от {\ddmmyyyydate\today} \currenttime}

% После комментария с номером пишем формулировку вопроса в subsection*{...}
% К примеру:

% Вопрос 1
% \subsection*{Перечислимые множества суть, в точности, проекции разрешимых.}
% и тут уже решение формулировочки все дела

\begin{document}
    \maketitle

    \section*{Вычислимость}


    % Вопрос вычислимость 1

    % Вопрос вычислимость 2

    % Вопрос вычислимость 3

    % Вопрос вычислимость 4

    % Вопрос вычислимость 5

    % Вопрос вычислимость 6

    % Вопрос вычислимость 7

    % Вопрос вычислимость 8

    % Вопрос вычислимость 9

    % Вопрос вычислимость 10

    % Вопрос вычислимость 11

    % Вопрос вычислимость 12

    % Вопрос вычислимость 13

    % Вопрос вычислимость 14

    % Вопрос вычислимость 15

    % Вопрос вычислимость 16

    % Вопрос вычислимость 17

    % Вопрос вычислимость 18

    % Вопрос вычислимость 19

    % Вопрос вычислимость 20

    % Вопрос вычислимость 21

    % Вопрос вычислимость 22

    % Вопрос вычислимость 23


    \section*{Логика}


    % Вопрос логика 1

    % Вопрос логика 2

    % Вопрос логика 3

    % Вопрос логика 4

    % Вопрос логика 5

    % Вопрос логика 6

    % Вопрос логика 7
    \subsection*{7.
    Общезначимые и выполнимые формулы. [F, 12]
    Эквивалентность формул первого порядка. [F, с. 4]
    Лемма о фиктивном кванторе. [F, 10]
    Квантор всеобщности и общезначимость. [F, 12]}

    \subsubsection*{Общезначимые и выполнимые формулы.}

    Пусть фиксирована сигнатура $\sigma$. 
    \begin{definition}
        Формула $\varphi$ {\it общезначима}, если для любой интерпретации $\mathcal{M}$, для 
        любой оценки переменных $\pi$ в этой самой интерпретации: $\text{Var} \to M$ 
        $$
            [\varphi]_{\mathcal{M}}(\pi) = 1.
        $$
        (Как ее ни интерпретируй, что с ней ни делай, будет принимать значение 1, 
        которое всегда истинно.)
    \end{definition}

    \begin{example}
        $$
            \psi = Px \implies (\neg Px \implies Qyz)
        $$
        Если $Px$ ложь, то $\psi$ истинна, иначе получаем $\psi \colon 1 \implies 
        (0 \implies *)$ --- тоже истина (запись варварская, ну и что?) $\implies 
        \psi $ общезначима. 
    \end{example}
    % \begin{example}
    %     $$
    %         \phi = \forall x Px \implies \exists x Px.
    %     $$
    %     Ничего не зависит от $P$, то есть $\phi$ всегда верно, если $M \neq 
    %     \varnothing$ (по определению $M$, это правда) $\implies \phi $ общезначима.
    % \end{example}
    \begin{example}
        Рассмотрим формулу
        $$
            \varphi \eqcirc Px \lor \neg Qy.
        $$
        Положим $M = \NN$, $P^{\mathcal{M}} = Q^{\mathcal{M}} =$ 
        ``равно 0'', $\pi(x) = 2020$, $\pi(y) = 0$.
        Тогда значение $\varphi$ ложно $ \implies \varphi $ не общезначима.
    \end{example}

    \begin{definition}
        Формула $\varphi$ {\it выполнима}, если существует интерпретация $\mathcal{M}$ 
        и оценка $\pi$ такая, что
        $$
            [\varphi]_{\mathcal{M}}(\pi) = 1.
        $$
        Общезначимая формула истинна всегда, выполнимая истинна в каком-то случае.
    \end{definition}

    \begin{lemma}
        Формула $\varphi$ общезначима $\iff$ $\neg \varphi$ не выполнима.
    \end{lemma}

    \begin{lemma}
        Формула $\varphi$ выполнима $\iff$ $\neg \varphi$ не общезначима.
    \end{lemma}

    \subsubsection*{Эквивалентность формул первого порядка.}

    Формулы первого порядка --- формулы, в которых кванторы берутся только по элементам 
    носителя.
    \begin{example}
        первый порядок: $\forall x \in M$, 
        не первый порядок: $\forall y \subseteq M$
    \end{example}

    \begin{definition}
        Формулы $\phi$ и $\psi$ (логически) эквивалентны, если для любой интерпретации
        $\mathcal{M}$, для любой оценки $\pi$
        $$
            [\varphi]_{\mathcal{M}}(\pi) = [\psi]_{\mathcal{M}}(\pi).
        $$
    \end{definition}

    \begin{lemma}
        Определенная выше эквивалентность является отношением эквивалентности:
        \begin{enumerate}[topsep=0pt]
            \item $\varphi \equiv \varphi$;
            \item $\varphi \equiv \psi \implies \psi \equiv \varphi$;
            \item $\varphi \equiv \psi \land \psi \equiv \theta \implies \psi \equiv \theta$.
        \end{enumerate}
    \end{lemma}

    \begin{lemma}[О конгруэнции отношения эквивалентности]
        Если $\varphi \equiv \varphi^{\prime}$, то
        \begin{itemize}[topsep=0pt]
            \item $\neg \varphi \equiv \varphi^{\prime}$;
            \item $\varphi \land \psi \equiv \varphi^{\prime} \land \psi$.
            \item $\varphi \lor \psi \equiv \varphi^{\prime} \lor \psi$.
            \item $\varphi \implies \psi \equiv \varphi^{\prime} \implies \psi$.
            \item $\psi \implies \varphi \equiv \psi \implies \varphi^{\prime}$.
            \item $\forall x \varphi \equiv \forall x \varphi^{\prime}$.
            \item $\exists x \varphi \equiv \exists x \varphi^{\prime}$.
        \end{itemize}
        Так как значение сложной формулы определяется через значения ее подформул и 
        для любой интерпретации $\mathcal{M}$, для любой оценки $\pi$ значение 
        $\varphi$ можно заменить на значение $\varphi^{\prime}$.
    \end{lemma}

    \begin{proof}
        Докажем одно из утверждений. 

        $[\forall x \varphi]_{\mathcal{M}}(\pi) = 1 
        \iff \forall m \in M~[\varphi]\left( \pi_{x}^{m} \right) = 1$, 
        но $[\varphi]\left( \pi_{x}^{m} \right) = 
        [\varphi^{\prime}]\left( \pi_{x}^{m} \right)$, тогда 
        $[\varphi^{\prime}]\left( \pi_{x}^{m} \right) = 1$.
    \end{proof}

    \begin{lemma}
        $\varphi \equiv \psi \iff $ общезначима формула
        $$
            (\varphi \implies \psi) \land (\psi \implies \varphi).
        $$
        Импликация истинна $\iff$ значение посылки не больше значения заключения, 
        а в лемме сказано, что они одинаковы.
    \end{lemma}
    \begin{lemma}
        Формула $\varphi$ общезначима тогда и только тогда, когда
        $$
            \varphi \equiv \top \ (\text{``заведомая истина''}).
        $$
    \end{lemma}

    \begin{corollary}
        Все общезначимые формулы эквивалентны друг другу.
    \end{corollary}

    \subsubsection*{Лемма о фиктивном кванторе.}

    \begin{lemma}[О фиктивном кванторе.]
        Если $x \notin \text{FV}(\varphi)$, то $\forall x \varphi \equiv \varphi$.
        Аналогично, $\exists \varphi \equiv \varphi$.
    \end{lemma}

    \begin{proof}
        Рассмотрим произвольную интерпретацию $\mathcal{M}$ и оценку $\pi$. По определению: 
        $$
            [\forall x \varphi]_{\mathcal{M}}(\pi) = 1 \iff \forall m \in M~ [\varphi](\pi_{x}^{m}) = 1.
        $$
        где 
        $$
            \pi_{x}^{m}(z) = 
            \begin{cases}
                m, & y \eqcirc x,  \\
                \pi(y), & y \not \eqcirc x.
            \end{cases}
        $$
        Вспомним лемму о том, что $\forall z \in \text{FV}(\psi) \colon \pi_{1}(z) = 
        \pi_{2}(z)$, то $[\psi]_{\mathcal{M}}(\pi_{1}) = [\psi]_{\mathcal{M}}(\pi_{2})$.
        
        Оценки $\pi_{x}^{m}$ и $\pi$ подходят под эту лемму, так как отличаются только в $x$, 
        а $x \notin \text{FV}(\varphi)$. 

        Тогда
        $$
            [\forall x \varphi]_{\mathcal{M}}(\pi) = 1 \iff \forall m \in M~ [\varphi](\pi_{x}^{m}) = 1 \iff
            \forall m \in M~ [\varphi](\pi) = 1 \iff [\varphi](\pi) = 1
        $$
        Последний переход верен, так как значение $[\varphi](\pi)$ не зависит от $m$.
    \end{proof}

    \begin{corollary}
        $\forall x \exists x \varphi \equiv \exists x \varphi$. 
        (Так как $x \notin \text{FV}(\exists x \varphi)$.)
    \end{corollary}


    \subsubsection*{Квантор всеобщности и общезначимость.}

    \begin{lemma}
        Формула $\varphi$ общезначима $\iff$ $\forall x \varphi$ общезначима.
    \end{lemma}
    
    \begin{proof}~
        \begin{description}
            \item[$\implies$] Дано: $\forall \mathcal{M},~\forall \pi \colon [\varphi]_{\mathcal{M}}(\pi) = 1$.
            Хотим доказать: $\forall \mathcal{M},~\forall \rho \colon [\forall x \varphi]_{\mathcal{M}}(\rho) = 1$.
            Зафиксируем какие-то $\mathcal{M}$ и $\rho$, тогда
            $$
                [\forall x \varphi](\rho) = 1 \iff \forall m \in M~ [\varphi]\left(\rho_{x}^{m}\right) = 1.
            $$
            Последнее --- то, что мы хотим получить. Однако 
            $$
                ~\forall \pi \colon [\varphi]_{\mathcal{M}}(\pi) = 1 \implies
                ~\forall \pi \colon \forall m \in M [\varphi]_{\mathcal{M}}(\pi) = 1 \implies
                \forall m \in M [\varphi]_{\mathcal{M}}(\rho_{x}^{m}) = 1
            $$
            (Добавили квантор, который ничего не меняет и подставили $\pi = \rho_{x}^{m}$, получили
            то, что хотели для выбранной $\rho$.)
            
            \item[$\impliedby$] Дано: $\forall \mathcal{M},~\forall \pi \colon [\forall\varphi]_{\mathcal{M}}(\pi) = 1$.
            Хотим доказать: $\forall \mathcal{M},~\forall \rho \colon [\varphi]_{\mathcal{M}}(\rho) = 1$.

            Зафиксируем какие-то $\mathcal{M}$ и $\rho$, тогда
            $$
                \forall \pi \colon [\forall\varphi]_{\mathcal{M}}(\pi) = 1 \iff
                \forall \pi \colon \forall m \in M \colon [\varphi]_{\mathcal{M}}\left(\pi_{x}^{m}\right) = 1
            $$
            Возьмем $\pi = \rho$, $m = \rho(x)$, тогда $\pi_{x}^{m} = \rho_{x}^{\rho(x)} = \rho$. 
            Получаем $[\varphi]_{\mathcal{M}}(\rho) = 1$.
        \end{description}
    \end{proof}

    % Вопрос логика 8
    \subsection*{8.
    Основные эквивалентности логики первого порядка. 
    Замена подформулы на эквивалентную.}

    % Вопрос логика 9
    \subsection*{9.
    Булевы комбинации формул. 
    Булева функция, соответствующая булевой комби- нации. 
    Теорема о приведении булевой комбинации к дизъюнктивной 
    нормальной форме и к конъюнктивной нормальной форме.}

    % Вопрос логика 10

    % Вопрос логика 11

    % Вопрос логика 12

    % Вопрос логика 13

    % Вопрос логика 14

    % Вопрос логика 15

    % Вопрос логика 16

    % Вопрос логика 17

    % Вопрос логика 18

    % Вопрос логика 19

    % Вопрос логика 20

    % Вопрос логика 21

    % Вопрос логика 22

    % Вопрос логика 23

\end{document}
