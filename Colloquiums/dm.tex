\documentclass[a4paper, fleqn]{article}
\usepackage{header}

\title{Коллоквиум по Дискретной Математике}
\author{
    % Откомментируй себя или добавь в список!

    % Александр Богданов   \\ \href{https://t.me/SphericalPotatoInVacuum}{Telegram} \and
    % Алиса Вернигор       \\ \href{https://t.me/allisyonok}{Telegram} \and
    % Анастасия Григорьева \\ \href{https://t.me/weifoll}{Telegram} \and
    % Василий Шныпко       \\ \href{https://t.me/yourvash}{Telegram} \and
    % Данил Казанцев       \\ \href{https://t.me/vserosbuybuy}{Telegram} \and
    Денис Козлов         \\ \href{https://t.me/DKozl50}{Telegram} \and
    % Елизавета Орешонок   \\ \href{https://t.me/eaoresh}{Telegram} \and
    % Ира Голобородько     \\ \href{https://t.me/Ira4kgl}{Telegram} \and
    % Иван Пешехонов       \\ \href{https://t.me/JohanDDC}{Telegram} \and
    % Иван Добросовестнов  \\ \href{https://t.me/ivankot13}{Telegram} \and
    % Настя Городилова     \\ \href{https://t.me/nastygorodi}{Telegram} \and
    % Никита Насонков      \\ \href{https://t.me/nnv_nick}{Telegram} \and
    % Сергей Лоптев        \\ \href{https://t.me/beast_sl}{Telegram}
}

\date{Версия от {\ddmmyyyydate\today} \currenttime}

% После комментария с номером пишем формулировку вопроса в subsection*{...}
% К примеру:

% Вопрос 1
% \subsection*{Перечислимые множества суть, в точности, проекции разрешимых.}
% и тут уже решение формулировочки все дела

\begin{document}
    \maketitle

    \section*{Вычислимость}


    % Вопрос вычислимость 1

    % Вопрос вычислимость 2

    % Вопрос вычислимость 3

    % Вопрос вычислимость 4

    % Вопрос вычислимость 5

    % Вопрос вычислимость 6

    % Вопрос вычислимость 7

    % Вопрос вычислимость 8

    % Вопрос вычислимость 9

    % Вопрос вычислимость 10

    % Вопрос вычислимость 11

    % Вопрос вычислимость 12

    % Вопрос вычислимость 13

    % Вопрос вычислимость 14

    % Вопрос вычислимость 15

    % Вопрос вычислимость 16

    % Вопрос вычислимость 17

    % Вопрос вычислимость 18

    % Вопрос вычислимость 19

    % Вопрос вычислимость 20

    % Вопрос вычислимость 21

    % Вопрос вычислимость 22

    % Вопрос вычислимость 23


    \section*{Логика}


    % Вопрос логика 1

    % Вопрос логика 2

    % Вопрос логика 3

    % Вопрос логика 4

    % Вопрос логика 5

    % Вопрос логика 6

    % Вопрос логика 7

    % Вопрос логика 8

    % Вопрос логика 9

    % Вопрос логика 10
    \setcounter{section}{2}
    \subsection*{10. Лемма о корректной постановке}
    \begin{lemma}[73]
        В любой интерпретации при любой оценке $\pi$ для всех $\varphi \in \text{Fm}_{\sigma}, \, t, s \in \text{Tm}_{\sigma}, \, x \in \text{Var}$, если $t-x-\varphi$, то \\[-10 pt]
        \[ [s(t/x)](\pi) = [s](\pi + (x \to [t](\pi))) \text{ и } [\varphi(t/x)](\pi) = [\varphi](\pi + (x \to [t](\pi))). \]
    \end{lemma}
    
    \begin{proof}
        Так же, как при доказательстве леммы 18 (являющейся частным случаем данной) проведем индукцию по построению: пусть $s = z \ne x$, тогда $[s(t/x)](\pi) = \pi(z) = [s](\pi + (x \to [t](\pi)))$, если $е = x$, то $[s(t/x)](\pi) = [t](\pi) = [s](\pi + (x \to [t](\pi)))$. Случай $s = \F_i$ тривиален (значение $s$ --- константа, не зависящая от $x$).
        
        Если $s = \F_j(s_1, s_2, \ldots, s_{a_j})$, то по предположению индукции:
        \begin{multline*}
            [s(t/x)](\pi) = \F_j([s_1(t/x)](\pi), \ldots, [s_{a_j}(t/x)](\pi)) = \\
            = \F_j([s_1](\pi + (x \to [t](\pi))), \ldots, [s_{a_j}](\pi + (x \to [t](\pi)))) = [s](\pi + (x \to [t](\pi)))
        \end{multline*}
        Случай $\varphi = \mathbbold{P}_i$ тривиален.
        
        Если $\varphi = \mathbbold{P}_j(s_1, t_2, \ldots, s_{a_j})$, то по предположению индукции:
        \begin{multline*}
            [\varphi(t/x)](\pi) = \mathbbold{P}_j([s_1(t/x)](\pi), \ldots, [s_{a_j}(t/x)](\pi)) = \\
            = \mathbbold{P}_j([s_1](\pi + (x \to [t](\pi))), \ldots, [s_{a_j}](\pi + (x \to [t](\pi)))) = [\varphi](\pi + (x \to [t](\pi)))
        \end{multline*}
    \end{proof}

    % Вопрос логика 11

    % Вопрос логика 12

    % Вопрос логика 13

    % Вопрос логика 14

    % Вопрос логика 15

    % Вопрос логика 16

    % Вопрос логика 17

    % Вопрос логика 18

    % Вопрос логика 19

    % Вопрос логика 20

    % Вопрос логика 21

    % Вопрос логика 22

    % Вопрос логика 23

\end{document}
