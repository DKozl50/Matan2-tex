\documentclass[a4paper, fleqn]{article}
\usepackage{header}

\title{Коллоквиум по Дискретной Математике}
\author{
    % Откомментируй себя или добавь в список!

    % Александр Богданов   \\ \href{https://t.me/SphericalPotatoInVacuum}{Telegram} \and
    % Алиса Вернигор       \\ \href{https://t.me/allisyonok}{Telegram} \and
    % Анастасия Григорьева \\ \href{https://t.me/weifoll}{Telegram} \and
    % Василий Шныпко       \\ \href{https://t.me/yourvash}{Telegram} \and
    % Данил Казанцев       \\ \href{https://t.me/vserosbuybuy}{Telegram} \and
    Денис Козлов         \\ \href{https://t.me/DKozl50}{Telegram} \and
    % Елизавета Орешонок   \\ \href{https://t.me/eaoresh}{Telegram} \and
    % Ира Голобородько     \\ \href{https://t.me/Ira4kgl}{Telegram} \and
    % Иван Пешехонов       \\ \href{https://t.me/JohanDDC}{Telegram} \and
    % Иван Добросовестнов  \\ \href{https://t.me/ivankot13}{Telegram} \and
    % Настя Городилова     \\ \href{https://t.me/nastygorodi}{Telegram} \and
    % Никита Насонков      \\ \href{https://t.me/nnv_nick}{Telegram} \and
    % Сергей Лоптев        \\ \href{https://t.me/beast_sl}{Telegram}
}

\date{Версия от {\ddmmyyyydate\today} \currenttime}

% После комментария с номером пишем формулировку вопроса в subsection*{...}
% К примеру:

% Вопрос 1
% \subsection*{Перечислимые множества суть, в точности, проекции разрешимых.}
% и тут уже решение формулировочки все дела

\begin{document}
    \maketitle

    \section*{Вычислимость}


    % Вопрос вычислимость 1

    % Вопрос вычислимость 2

    % Вопрос вычислимость 3

    % Вопрос вычислимость 4

    % Вопрос вычислимость 5

    % Вопрос вычислимость 6

    % Вопрос вычислимость 7

    % Вопрос вычислимость 8

    % Вопрос вычислимость 9

    % Вопрос вычислимость 10

    % Вопрос вычислимость 11

    % Вопрос вычислимость 12

    % Вопрос вычислимость 13

    % Вопрос вычислимость 14

    % Вопрос вычислимость 15

    % Вопрос вычислимость 16

    % Вопрос вычислимость 17

    % Вопрос вычислимость 18

    % Вопрос вычислимость 19

    % Вопрос вычислимость 20

    % Вопрос вычислимость 21

    % Вопрос вычислимость 22

    % Вопрос вычислимость 23


    \section*{Логика}


    % Вопрос логика 1

    % Вопрос логика 2

    % Вопрос логика 3

    % Вопрос логика 4

    % Вопрос логика 5

    % Вопрос логика 6

    % Вопрос логика 7

    % Вопрос логика 8

    % Вопрос логика 9

    % Вопрос логика 10

    % Вопрос логика 11

    % Вопрос логика 12

    % Вопрос логика 13

    % Вопрос логика 14

    % Вопрос логика 15

    % Вопрос логика 16

    % Вопрос логика 17

    % Вопрос логика 18

    % Вопрос логика 19

    % Вопрос логика 20

    % Вопрос логика 21

    % Вопрос логика 22

    % Вопрос логика 23

\end{document}
