\documentclass[a4paper, fleqn]{article}
\usepackage{header}

\title{Коллоквиум 1}
\author{
        % ОТКОММЕНТИРУЙ СЕБЯ
    % Александр Богданов   \\ \href{https://t.me/SphericalPotatoInVacuum}{Telegram} \and
    % Алиса Вернигор       \\ \href{https://t.me/allisyonok}{Telegram} \and
    % Анастасия Григорьева \\ \href{https://t.me/weifoll}{Telegram} \and
    % Василий Шныпко       \\ \href{https://t.me/yourvash}{Telegram} \and
    % Данил Казанцев       \\ \href{https://t.me/vserosbuybuy}{Telegram} \and
    Денис Козлов         \\ \href{https://t.me/DKozl50}{Telegram} \and
    Елизавета Орешонок   \\ \href{https://t.me/eaoresh}{Telegram} \and
    % Иван Пешехонов       \\ \href{https://t.me/JohanDDC}{Telegram} \and
    % Иван Добросовестнов  \\ \href{https://t.me/ivankot13}{Telegram} \and
    % Настя Городилова     \\ \href{https://t.me/nastygorodi}{Telegram} \and
    % Никита Насонков      \\ \href{https://t.me/nnv_nick}{Telegram} \and
    % Сергей Лоптев        \\ \href{https://t.me/beast_sl}{Telegram}
}

\date{Версия от {\ddmmyyyydate\today} \currenttime}

\begin{document}
    \maketitle
    % Решение вопроса пишем после комментария 
    % комментарии трогать пожалуйста не надо, будет круто
    
    % давайте для постоянства формулировку в сабсекшн кидать
    % доказательства советуют оборачивать в \begin{proof} \end{proof}
    
    % вопрос 1

    % вопрос 2

    % вопрос 3
        
    % вопрос 4
        
    % вопрос 5
    \subsection*{5}
	\textbf{ Сформулировать и доказать признак сравнения числовых рядов, основанный на пределе $\lim\dfrac{a_n}{b_n}$.} \\[5 pt]
	Пусть $\sum a_n$ и $\sum b_n$ --- положительные ряды, и $\exists \lim\limits_{n \to \infty} \dfrac{a_n}{b_n} \in (0;\, +\infty)$. \\[3 pt]
	Тогда ряд $\sum a_n$ сходится $\Leftrightarrow$ $\sum b_n$ сходится. \\
	\begin{proof}
	$c = \lim\limits_{n \to \infty} \dfrac{a_n}{b_n} > 0$ \\[3 pt]
	По определению предела: \\[3 pt]
	$\forall \varepsilon > 0 \; \exists n_0 : c - \varepsilon \le \dfrac{a_n}{b_n} \le c + \varepsilon \;\; \forall n \ge n_0 \; 
	\Rightarrow \; (c - \varepsilon) b_n \le a_n \le (c + \varepsilon) b_n \; \Rightarrow \\[3 pt]
	\Rightarrow \; \sum (c - \varepsilon) b_n  \le \sum a_n \le \sum (c + \varepsilon) b_n \;
	\Leftrightarrow \; C_1 \sum b_n  \le \sum a_n \le C_2 \sum b_n $
	\end{proof}    
    % вопрос 6
        
    % вопрос 7
        
    % вопрос 8
        
    % вопрос 9
        
    % вопрос 10
        
    % вопрос 11
        
    % вопрос 12
        
    % вопрос 13
        
    % вопрос 14
        
    % вопрос 15
    \subsection*{15}
        \textbf{ Доказать, что всякий раз, когда признак Даламбера даёт ответ на вопрос о сходимости или расходимости ряда, 
	радикальный признак Коши также даёт (тот же) ответ на этот вопрос.} \\[5 pt]
	Пусть $a_n > 0$. Тогда 
	$\;\;\; \left\{\begin{array}{lllllll}
	\varlimsup \, \dfrac{a_{n+1}}{a_n} & < & 1 & \Rightarrow & \varlimsup \sqrt[n]{a_n} & < & 1,\\[5 pt]
	\varliminf \, \dfrac{a_{n+1}}{a_n} & > & 1 & \Rightarrow & \varlimsup \sqrt[n]{a_n} & > & 1,\\[5 pt]
	\end{array}\right.$. \\
	\begin{proof}
	Для доказательства основного утверждения докажем неравенство: \\[3 pt]
	$\varliminf \, \dfrac{a_{n+1}}{a_n} \le \varliminf \sqrt[n]{a_n} \le \varlimsup \sqrt[n]{a_n} \le \varlimsup \, \dfrac{a_{n+1}}{a_n} \\[3 pt]
	\varliminf \sqrt[n]{a_n} \le \varlimsup \sqrt[n]{a_n} \,$ очевидно, докажем 
	$\, \varlimsup \sqrt[n]{a_n} \le \varlimsup \, \dfrac{a_{n+1}}{a_n}$ \\[3 pt]
	(левое неравенство доказывается аналогично): \\[3 pt]
	Пусть $q = \varlimsup \sqrt[n]{a_n}, \; p = \varlimsup \, \dfrac{a_{n+1}}{a_n}$. \\[3 pt]
	От противного: пусть $p < q$:\\[3 pt]
	$\forall \varepsilon > 0 \; \exists \{ n_k \} : \sqrt[n_k]{a_{nk}} \ge q - \varepsilon \; \Rightarrow \; a_{nk} \ge (q - \varepsilon)^{n_k}$ \\[3 pt]
	$\exists n_0 : \dfrac{a_{n+1}}{a_n} \le p + \varepsilon, \; n \ge n_0 \; \Rightarrow \; a_{n0 + m} \le a_{n0} (p + \varepsilon)^m$ \\[3 pt]
	$(q - \varepsilon)^{n_k} \le a_{nk} \le a_{n0} (p + \varepsilon)^{n_k - n_0} \; \Rightarrow \; $
	$\dfrac{a_{n0}}{(p + \varepsilon)^{n_0}} \ge \left( \dfrac{q - \varepsilon}{p + \varepsilon} \right)^{n_k} \;\, \forall k = 1, 2, \dots;$ \\[3 pt]
	но $\; \dfrac{q - \varepsilon}{p + \varepsilon} > 1 \; $ при малом $\varepsilon$ по предположению $\Rightarrow$ \\[3 pt]
	$\; \Rightarrow \; \left( \dfrac{q - \varepsilon}{p + \varepsilon} \right)^{n_k}$ --- бесконечно большое, тогда как
	$\dfrac{a_{n0}}{(p + \varepsilon)^{n_0}} = C$ --- некоторая константа. \\[3 pt]
	Получили неравенство $C \ge +\infty$ --- противоречие, следовательно, предположение неверно, и неравенство выполняется. \\[3 pt]
	Из $\; \varliminf \, \dfrac{a_{n+1}}{a_n} \le \varliminf \sqrt[n]{a_n} \le \varlimsup \sqrt[n]{a_n} \le \varlimsup \, \dfrac{a_{n+1}}{a_n}\; $ 
        исходное утверждение следует очевидно.
	\end{proof}    
    % вопрос 16
        
    % вопрос 17
        
    % вопрос 18
        
    % вопрос 19
        
    % вопрос 20
    \subsection*{20}
        \textbf{ Привести пример положительного ряда, вопрос о поведении которого не может быть решен с помощью признака Гаусса.} \\[5 pt]
        $\sum\limits_{n=1}^{\infty} \dfrac1{n \ln^p n}$ --- положительный ряд, $a_n = \dfrac1{n \ln^p n}$ \\[3 pt]
        Рассмотрим отношение: \\[3 pt]
        $\dfrac{a_{n + 1}}{a_n} = \dfrac{\dfrac1{(n + 1) \ln^p (n + 1)}}{\dfrac1{n \ln^p n}} = 
        \dfrac{n}{n + 1} \cdot \dfrac{\ln^p n}{\ln^p (n + 1)} = 
        \dfrac{1}{1 + \frac1n} \cdot \dfrac{\ln^p n}{\left( \ln n + \ln \left(1 + \frac1n \right) \right)^p} = \\[3 pt]
        = \left[\text{ По формуле Тейлора для $(1+x)^{-1}$ и $\ln (1+x) \sim x$ }\right] \; 
        \left(1 - \frac1n + \frac1{n^2} + o\left(\frac1{n^2}\right)\right) \cdot \left(1 + \frac{\frac1n + o\left(\frac1n\right)}{\ln n}\right)^{-p} = \\[3 pt]
        \left[\text{ Перешли к менее строгому приближению и  снова разложили $(1+x)^{-p}$ }\right] \\[3 pt] 
        = \left(1 - \frac1n + o\left(\frac1{n \ln n}\right)\right) \cdot \left(1 - \frac{p}{n \ln n} + o\left(\frac1{n \ln n}\right)\right)
        = 1 - \dfrac1n - \dfrac{p}{n \ln n} + o\left(\frac1{n \ln n}\right)$ \\[3 pt]
        Для использования признака Гаусса должны получить приближение 
        $1 - \frac{q}n + O\left( \frac1{n^{1+\delta}} \right), \; \delta > 0$, \\[3 pt]
        но $ - \dfrac{p}{n \ln n} + o\left(\frac1{n \ln n}\right) \ne O\left( \frac1{n^{1+\delta}} \right)$, т.к.
        $\dfrac1{\ln n} > \dfrac1{n^{\delta}}$ при $n \to \infty \;\; \forall \delta > 0$    
    % вопрос 21
        
    % вопрос 22
        
    % вопрос 23
        
    % вопрос 24
        
    % вопрос 25
    \subsection*{25}
        \textbf{ Доказать, что если ряд сходится условно, то его положительная и отрицательная части расходятся.} \\[5 pt]
	$\sum a_n$ --- ряд, $S_{+} = \sum a_n^{+}$ и $S_{-} = \sum a_n^{-}$
	 --- положительная и отрицательная части суммы соответственно. \\[3 pt]
	 $\left\{\begin{array}{lll} 
	 \sum a_n &=& C,\\[5 pt]
	 \sum |a_n| &=& \pm \infty
	 \end{array}\right. \Rightarrow \; S_{+}$ и $S_{-}$ расходятся.
	\begin{proof}
	По определению: \\[3 pt]
	$\sum a_n = S_{+} - S_{-}, \;\; \sum |a_n| = S_{+} + S_{-}$. \\[3 pt]
	От противного: пусть\\[-20 pt]
	\begin{enumerate}
	\item $S_{+}, \; S_{-}$ конечны. Тогда $\;\sum |a_n| = S_{+} + S_{-} = C_1 + C_2 = const$ --- сходится, противоречие.
	\item $S_{+} \, $ конечна, $S_{-}$ расходится (симметричный случай аналогично). \\[0 pt]
	Тогда $\; \sum a_n = S_{+} - S_{-} = C_1 - \underbrace{S_{-}}_{\text{беск. большое}} = -\infty$ --- расходится, противоречие.
	\end{enumerate}
	\end{proof}    
    % вопрос 26
        
    % вопрос 27
        
    % вопрос 28
        
    % вопрос 29
        
    % вопрос 30
        
    % вопрос 31
        
    % вопрос 32
        
    % вопрос 33
        
    % вопрос 34
        
    % вопрос 35
    \subsection*{35}
        \textbf{Сформулировать признак Абеля. Вывести утверждение признака Абеля из признака Дирихле.} \\[5 pt]
	\textbf{Признак Абеля. } Если $\{ a_n \}$ монотонна и ограничена $|a_n| \le C$, а $\sum b_n$ сходится,
	ряд $\; \sum a_n \cdot b_n \;$ также сходится.\\[5 pt]
	Пусть некоторая последовательность $a_n \cdot b_n$ удовлетворяет признаку Абеля. \\[3 pt]
	У монотонной ограниченной последовательности существует конечный предел: $\lim a_n = A$. \\[3 pt]
	Представим исходную последовательность в виде суммы: \\[3 pt]
	$a_n \cdot b_n = A \cdot b_n + (a_n - A) b_n \; \Rightarrow \; 
	\sum a_n \cdot b_n = \underbrace{\sum A \cdot b_n}_{\text{сходится}} + \sum (a_n - A) b_n$ \\[3 pt]
	$a_n \to A \; \Rightarrow \; (a_n - A) \to 0$, причем, т.к. $\{ a_n \}$ монотонная, $\{ (a_n - A) \}$ монотонно стремится к 0. \\[3 pt]
	Т.к. ряд $\;\sum b_n$ сходится, последовательность его частичных сумм также сходится. \\[3 pt]
	$\left\{\begin{array}{lll} 
	\{ (a_n - A) \} &\downarrow& 0,\\[10 pt]
	\left\{ \sum\limits_{n=1}^N b_n \right\} &\le& B
	\end{array}\right. \Rightarrow \; \sum (a_n - A) b_n$ сходится по признаку Дирихле.\\[3 pt]
	$\sum a_n \cdot b_n = \underbrace{\sum A \cdot b_n}_{\text{сходится}} +\underbrace{\sum (a_n - A) b_n}_{\text{сходится}}$ --- сходится.    
    % вопрос 36
        
    % вопрос 37
        
    % вопрос 38
        
    % вопрос 39
        
    % вопрос 40
        
    % вопрос 41
        
    % вопрос 42
        
    % вопрос 43
        
    % вопрос 44
        
    % вопрос 45
    \subsection*{45}
        \textbf{Как определяется соответствующий бесконечному произведению ряд? Сформулировать и доказать утверждение об их взаимосвязи.} \\[5 pt]
	Пусть $\prod\limits_{n = 1}^{\infty} a_n$ --- бесконечное произведение. \\[3 pt]
	Тогда ряд $\sum\limits_{n = 1}^{\infty} \ln a_n$ называется соответствующим этому бесконечному произведению. \\[3 pt]
	Так как $a_n = e ^{\ln a_n}$, верно равенство $\; \prod\limits_{n = 1}^{\infty} a_n = \prod\limits_{n = 1}^{\infty} e^{\ln a_n} = $ 
	{\large $e^{\;\sum\limits_{n = 1}^{\infty} \ln a_n}$ } (по свойству степени)    
    % вопрос 46
        
    % вопрос 47
        
    % вопрос 48
        
    % вопрос 49
        
    % вопрос 50
        
    % вопрос 51
        
    % вопрос 52
        
    % вопрос 53
        
    % вопрос 54
        
    % вопрос 55
    \subsection*{55}
        \textbf{ Доказать, что если две функциональные последовательности сходятся равномерно к предельным функциям, то их сумма также сходится равномерно к сумме двух этих предельных функций.} \\[5 pt]
	$\left\{\begin{array}{lll} 
	f_n &\overset{D}{\rightrightarrows}& f, \\[5 pt]
	g_n &\overset{D}{\rightrightarrows}& g
	\end{array}\right. \Rightarrow \; (f_n + g_n) \overset{D}{\rightrightarrows} (f + g)$
	\begin{proof}
	По определению равномерной сходимости: \\[3 pt]
	$\forall \varepsilon > 0 \; \exists N_1(\varepsilon) : |f_n(x) - f(x)| < \dfrac{\varepsilon}2 \;\;\; \forall n \ge N_1(\varepsilon)\;\; \forall x \in D$, \\[3 pt]
	$\forall \varepsilon > 0 \; \exists N_2(\varepsilon) : |g_n(x) - g(x)| < \dfrac{\varepsilon}2 \;\;\; \forall n \ge N_2(\varepsilon)\;\; \forall x \in D$ \\[3 pt]
	Тогда \\[3 pt]
	$\forall x \in D \;\; \forall n \ge max(N_1, N_2) : |(f_n(x) + g_n(x)) - (f(x) + g(x))|  = |(f_n(x) - f(x)) + (g_n(x) - g(x))|  \le \\[3 pt]
	\le |f_n(x) - f(x)| + |g_n(x) - g(x)| < \dfrac{\varepsilon}2 + \dfrac{\varepsilon}2 = \varepsilon$, т.е. \\[3 pt]
	$\forall \varepsilon > 0 \; \exists N = max(N_1, N_2) : |(f_n(x) + g_n(x)) - (f(x) + g(x))| < \varepsilon \;\;\; n \ge N \;\; \forall x \in D$, \\[5 pt]
	т.е. сумма $(f_n + g_n)$ равномерно сходится к $(f + g)$ на $D$.
	\end{proof}    
    % вопрос 56
        
    % вопрос 57
        
    % вопрос 58
        
    % вопрос 59
        
    % вопрос 60
        
    % вопрос 61
        
    % вопрос 62
        
    % вопрос 63
        
    % вопрос 64
        
    % вопрос 65
    \subsection*{65}
	\textbf{ Сформулировать и доказать теорему о почленном переходе к пределу в функциональной последовательности.} \\[5 pt]
	$-\infty \le a < b \le +\infty$, рассмотрим $D = (a;\,b), \; D = [a;\,b]$ \\[3 pt]
	Пусть $\; f_n \overset{D}{\rightrightarrows} f, \;\; x \in D, \;\; y_n = \lim\limits_{x \to x_0} f_n(x), \;\; \{ y_n \}$ сходится к $y$ \\[3 pt]
	Тогда $\; \lim\limits_{x \to x_0} f(x) = y$, \\[3 pt]
	т.е. $\; \lim\limits_{x \to x_0} \underbrace{\left( \lim\limits_{n \to \infty} f_n(x) \right)}_{f(x)} = 
	\underbrace{\lim\limits_{n \to \infty} \overbrace{\left( \lim\limits_{x \to x_0} f_n(x) \right)}^{y_n}}_{y}$
	\begin{proof}
	По определению предела сходящейся последовательности: \\[3 pt]
	$\forall \varepsilon > 0 \; \exists N : |y - y_n| < \dfrac{\varepsilon}3, \;\; \norm{f_n - f} < \dfrac{\varepsilon}3 \;\; \forall n \ge N$, \\[3 pt]
	$\forall \varepsilon > 0 \; \exists \delta > 0 : |x - x_0| < \delta \;\Rightarrow\; |f_n(x) - y_n| < \dfrac{\varepsilon}3 \;\;\; \forall n$ \\[5 pt]
	Тогда \\[5 pt]
	$|y - f_n(x)| \le |y - y_n| + |y_n - f_n(x)| + |f_n(x) - f(x)| < \varepsilon$, \\[5 pt]
	т.е. $f_n(x) \xrightarrow{x \to x_0} y$, что и требовалось доказать.
	\end{proof}    
    % вопрос 66
        
    % вопрос 67
        
    % вопрос 68
        
    % вопрос 69
        
    % вопрос 70
        
    % вопрос 71
        
    % вопрос 72
        
    % вопрос 73
        
    % вопрос 74
        
    % вопрос 75
    \subsection*{75}
	\textbf{ Как применяются признаки Даламбера и Коши для исследования сходимости функционального ряда?} \\[5 pt]
	\textbf{ Признак Даламбера} \\[5 pt] 
	Если $\exists q < 1 : \; |a_{n+1}(x)| \le q \cdot |a_n(x)|$ при $\forall n \ge n_0, \; x \in D$, 
	причем $a_{n0}(x)$ ограничена на $D$ (т.е. $\norm{a_{n0}} < \infty$), \\[1 pt]
	то $\; \sum a_n(x)$ сходится на $D$ абсолютно и равномерно. \\[5 pt]
	\textbf{ Радикальный признак Коши} \\[5 pt] 
	Если $\sum u_n$ --- знакоположительный числовой ряд, и существует конечный предел \\[3 pt]
	$l = \lim\limits_{n \to \infty} \sqrt[n]{u_n}$, то \\[-20 pt]
	\begin{enumerate}
	\item $l < 1 \;\; \Rightarrow \; $ ряд сходится
	\item $l > 1 \;\; \Rightarrow \; $ ряд расходится
	\item $l = 1 \;\; \Rightarrow \; $ необходимо дополнительное исследование
	\end{enumerate}
	Заметно, что признаки практически идентичны соответствующим признакам для числовых рядов.    
    % вопрос 76
        
    % вопрос 77
        
    % вопрос 78
        
    % вопрос 79
        
    % вопрос 80
        
    % вопрос 81
        
    % вопрос 82
        
    % вопрос 83
        
    % вопрос 84
        
    % вопрос 85
    \subsection*{85}
	\textbf{ Сформулировать и доказать теорему Абеля о сходимости степенного ряда.} \\[5 pt]
	\textbf{ Теорема Абеля} \\[-15 pt] 
	\begin{enumerate}
	\item[$1)$] Если степенной ряд $\sum c_n (x - x_0)^n$ сходится в точке $x_1 \ne x_0$, 
	то он сходится при всех $x : |x - x_0| < |x_1 - x_0|$
	\item[$2)$] Если степенной ряд $\sum c_n (x - x_0)^n$ расходится в точке $x_2 \ne x_0$, 
	то он расходится при всех $x : |x - x_0| > |x_2 - x_0|$\\[-30 pt]
	\end{enumerate}
	\begin{proof}
	$\left| \sum\limits_{n = m}^N c_n (x - x_0)^n \right| = 
	\left| \sum\limits_{n = m}^N c_n \cdot (x - x_0)^n \cdot \left( \dfrac{x - x_0}{x_1 - x_0} \right)^n \right| \le  \\[3 pt]
	\sum\limits_{n = m}^N \underbrace{\left| c_n \cdot (x - x_0)^n \right|}_{< \varepsilon \; \forall m \ge n_0} \cdot 
	\underbrace{\left| \dfrac{x - x_0}{x_1 - x_0} \right|^n}_{q^n}  \le
	\varepsilon \cdot (q^m + \ldots + q^N) \le \varepsilon \cdot q^m \cdot \dfrac1{1 - q} \to 0$
	\end{proof}    
    % вопрос 86
        
    % вопрос 87
        
    % вопрос 88
        
    % вопрос 89
        
    % вопрос 90
        
    % вопрос 91
        
    % вопрос 92
        
    % вопрос 93
        
    % вопрос 94
        
    % вопрос 95
    \subsection*{95}
	\textbf{ Сформулировать и доказать теорему о почленном интегрировании степенного ряда.} \\[5 pt]
	$\int\limits_{x_0}^x \left( \sum\limits_{n = 0}^{\infty} c_n(t - x_0)^n \right) dt = 
	\sum\limits_{n = 0}^{\infty} \dfrac{c_n}{n + 1}(x - x_0)^{n + 1} = 
	\sum\limits_{n = 1}^{\infty} \dfrac{c_{n - 1}}n (x - x_0)^n$
	\begin{proof}
	$\int\limits_{x_0}^x \left( \sum\limits_{n = 0}^{\infty} c_n(t - x_0)^n \right) dt = 
	\sum\limits_{n = 0}^{\infty} \int\limits_{x_0}^x c_n(t - x_0)^n dt = 
	\sum\limits_{n = 0}^{\infty} \dfrac{c_n}{n + 1}(t - x_0)^{n + 1}\Big|_{x_0}^x = \\[3 pt]
	= \sum\limits_{n = 0}^{\infty} \dfrac{c_n}{n + 1}(x - x_0)^{n + 1} - 
	\sum\limits_{n = 0}^{\infty} \dfrac{c_n}{n + 1}(x_0 - x_0)^{n + 1} = 
	\sum\limits_{n = 0}^{\infty} \dfrac{c_n}{n + 1}(x - x_0)^{n + 1} = \sum\limits_{n = 1}^{\infty} \dfrac{c_{n - 1}}n (x - x_0)^n$
	\end{proof}    
    % вопрос 96
        
    % вопрос 97
        
    % вопрос 98
        
    % вопрос 99
        
    % вопрос 100
        
    % вопрос 101
        
    % вопрос 102

\end{document}
