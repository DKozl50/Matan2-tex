\documentclass[a4paper, fleqn]{article}

\usepackage{header}

\title{Коллоквиум 1}
\author{
        % ОТКОММЕНТИРУЙ СЕБЯ
    % Александр Богданов   \\ \href{https://t.me/SphericalPotatoInVacuum}{Telegram} \and
    % Алиса Вернигор       \\ \href{https://t.me/allisyonok}{Telegram} \and
    % Анастасия Григорьева \\ \href{https://t.me/weifoll}{Telegram} \and
    % Василий Шныпко       \\ \href{https://t.me/yourvash}{Telegram} \and
    % Данил Казанцев       \\ \href{https://t.me/vserosbuybuy}{Telegram} \and
    Денис Козлов         \\ \href{https://t.me/DKozl50}{Telegram} \and
    % Елизавета Орешонок   \\ \href{https://t.me/eaoresh}{Telegram} \and
    % Иван Пешехонов       \\ \href{https://t.me/JohanDDC}{Telegram} \and
    % Иван Добросовестнов  \\ \href{https://t.me/ivankot13}{Telegram} \and
    % Настя Городилова     \\ \href{https://t.me/nastygorodi}{Telegram} \and
    % Никита Насонков      \\ \href{https://t.me/nnv_nick}{Telegram} \and
    % Сергей Лоптев        \\ \href{https://t.me/beast_sl}{Telegram}
}

\date{Версия от {\ddmmyyyydate\today} \currenttime}

\begin{document}
    \maketitle
    % Решение вопроса пишем после комментария 
    % комментарии трогать пожалуйста не надо, будет круто
    
    % давайте для постоянства формулировку в сабсекшн кидать
    % доказательства советуют оборачивать в \begin{proof} \end{proof}
    
    % вопрос 1

    % вопрос 2

    % вопрос 3
        
    % вопрос 4
        
    % вопрос 5
        
    % вопрос 6
        
    % вопрос 7
        
    % вопрос 8
        
    % вопрос 9
        
    % вопрос 10
        
    % вопрос 11
        
    % вопрос 12
        
    % вопрос 13
        
    % вопрос 14
        
    % вопрос 15
        
    % вопрос 16
        
    % вопрос 17
        
    % вопрос 18
        
    % вопрос 19
        
    % вопрос 20
        
    % вопрос 21
        
    % вопрос 22
        
    % вопрос 23
        
    % вопрос 24
        
    % вопрос 25
        
    % вопрос 26
        
    % вопрос 27
        
    % вопрос 28
        
    % вопрос 29
        
    % вопрос 30
        
    % вопрос 31
        
    % вопрос 32
        
    % вопрос 33
        
    % вопрос 34
        
    % вопрос 35
        
    % вопрос 36
    
    \subsection*{36. Что такое перестановка членов ряда? Приведите пример.}
    
    Пусть $f: \NN \to \NN$ биекция.
    
    Говорят, что ряд $\sum b_n$ получен из ряда $\sum a_n$ перестановкой членов, если $\exists$ биекция $f: \; b_n = a_{f(n)}.$ 
    
    \underline{Пример.}
    
    $\displaystyle \sum_{n = 1}^{\infty} a_n = \sum_{n = 1}^{\infty} \frac{(-1)^n}{n} = -1 + \frac{1}{2} - \frac{1}{3} + \frac{1}{4} - \dots = -\ln 2.$
    
    Пусть $\sum b_n$ получен так: сложим сначала $p$ положительных слагаемых из $\sum a_n$, потом $q$ отрицательных, затем снова $p$ положительных и так далее ($p, q \in \NN$, берем слагаемые по возрастанию их индексов).
    
    % вопрос 37
        
    % вопрос 38
        
    % вопрос 39
        
    % вопрос 40
        
    % вопрос 41
        
    % вопрос 42
        
    % вопрос 43
        
    % вопрос 44
        
    % вопрос 45
        
    % вопрос 46
        
    \subsection*{46. В каком случае бесконечное произведение называется сходящимся абсолютно? Сформулируйте и докажите критерий абсолютной сходимости бесконечного произведения.}
    
    $\displaystyle \prod_{n = 1}^{\infty} a_n$ наз-ся абсолютно сходящимся, если абсолютно сх-ся соответствующий ряд из логарифмов $\displaystyle \sum_{n = 1}^{\infty} \ln a_n$.
    
    Критерий абс. сх-ти:
    
    \fbox {$\displaystyle \prod_{n = 1}^{\infty} a_n$ сход. абс. $\iff \sum_{n = 1}^{\infty} (a_n - 1)$ сход. абс.}
    
    \begin{proof}
    
    Пусть $a_n = 1 + \alpha_n; \; \alpha_n \to 0. \; \circledast$
    
    Тогда $\ln a_n = \ln (1 + \alpha_n) = \alpha_n + \overline{o} (\alpha_n ) =  \alpha_n(1 + \overline{o} (1)) \implies
    |\ln a_n| = |\alpha_n| \cdot (1 + \overline{o} (1)),$ то есть $|\ln a_n| \sim |\alpha_n|.$
    
    \textit{Возможно, тут стоит упомянуть, что необходимое условие сходимости $\displaystyle \sum_{n = 1}^{\infty} |\ln a_n|$ это $|\ln a_n| \to 0 \iff a_n \to 1.$ Поэтому, если $\displaystyle \prod_{n = 1}^{\infty} a_n$ сход. абс., то $\circledast$ у нас верно всегда.}
    
    \end{proof}
    
    % вопрос 47
        
    % вопрос 48
        
    % вопрос 49
        
    % вопрос 50
        
    % вопрос 51
        
    % вопрос 52
        
    % вопрос 53
        
    % вопрос 54
        
    % вопрос 55
        
    % вопрос 56
    
    \subsection*{56. Докажите, что если 2 функциональные последовательности сходятся равномерно к ограниченным предельным функциям, то их произведение также сходится равномерно к произведению этих предельных функций.}
        
    \begin{proof}
    
    Пусть наши последовательности - $\{f_n\}, \, \{g_n\};$ их предельные функции - $f, g$ соотв.
    
    Знаем: $\forall \; \varepsilon_1, \varepsilon_2 \; \exists \; N_1(\varepsilon_1), \, N_2(\varepsilon_2): |f_n(x) - f(x)| < \varepsilon_1; \; |g_m(x) - g(x)| < \varepsilon_2$ при $n \geq N_1(\varepsilon_1), \;m \geq N_2(\varepsilon_2).$
    
     Пусть $|f(x)|$ ограничен ограничен какой-нибудь константой $C_1$.
    
    Так как $|g(x)|$ ограничен, то $|g_n(x)|$ ограничен какой-нибудь константой $C_2$. Следовательно,
    \begin{flalign}
    & |f_n(x)\cdot g_n(x) - f(x)\cdot g(x)| = \\
    & = |f_n(x)\cdot g_n(x) - f(x)\cdot g_n(x) + f(x)\cdot g_n(x) - f(x)\cdot g(x)| \leq \\
    & \leq |f_n(x)\cdot g_n(x) - f(x)\cdot g_n(x)| + |f(x)\cdot g(x) - f(x)\cdot g_n(x)| =\\
    & = |g_n(x)| \cdot |f_n(x) - f(x)| + |f(x)| \cdot |g(x) - g_n(x)| \leq C_2 \cdot \epsilon_1 + C_1 \cdot \epsilon_2 \text{ (начиная с $n = \max(N_1(\varepsilon_1), N_2(\varepsilon_2))$.}
    \end{flalign}
    Теперь возьмем произвольный $\varepsilon > 0$, и положим $\varepsilon_1 = \frac{\varepsilon}{3 \cdot C_2}; \; \varepsilon_2 = \frac{\varepsilon}{3 \cdot C_1}.$  
    
    Начиная с $n = \max(N_1(\varepsilon_1), N_2(\varepsilon_2))$ верно, что $ |f_n(x)\cdot g_n(x) - f(x)\cdot g(x)| \leq \varepsilon/3 + \varepsilon/3 < \varepsilon.$ Мы победили.
    \end{proof}
        
    % вопрос 57
        
    % вопрос 58
        
    % вопрос 59
        
    % вопрос 60
        
    % вопрос 61
        
    % вопрос 62
        
    % вопрос 63
        
    % вопрос 64
        
    % вопрос 65
        
    % вопрос 66
    
    \subsection*{66. Сформулируйте теорему о почленном дифференцировании функциональной последовательности.}
    
    $-\infty \leq a < b \leq +\infty, \; \; D = (a, b)$ или $D = [a, b].$
    
    Пусть $f_n$ дифф. на мн-ве $D$, и $f'_n  \stackrel{D}{\rightrightarrows} g, \; \exists \; c \in D : \{f_n(c)\}$ сходится.
    
    Тогда $\exists$ такая предельная функция $f : f_n \stackrel{D}{\to} f$ (причем, если $D$ ограничена, то $f_n \stackrel{D}{\rightrightarrows} f$), что $f$ дифф., и $f' = g.$
    
    Говоря иначе, $\displaystyle \left( \lim_{n \to \infty} f_n(x) \right)' = \lim_{n \to \infty} f'_n(x).$
    
    % вопрос 67
        
    % вопрос 68
        
    % вопрос 69
    
    \subsection*{69. Дайте определение равномерной сходимости функционального ряда.}
        
        $D \subseteq \RR, \; a_n : D \to \RR.$ 
        Рассмотрим функциональный ряд $\displaystyle \sum_{n = 1}^{\infty} a_n(x),$ и его ч.с. $S_N (x) := \displaystyle \sum_{n = 1}^{N} a_n(x).$
        
        Говорят, что ряд сх-ся равномерно на $D$, если последовательность $\{ S_N \}$ сх-ся равномерно на $D$.
        
        
    % вопрос 70
        
    % вопрос 71
        
    % вопрос 72
        
    % вопрос 73
        
    % вопрос 74
        
    % вопрос 75
        
    % вопрос 76
    
    % вопрос 77
    
    \subsection*{77. Сформулируйте признак Лейбница равномерной сходимости знакочередующегося функционального ряда.}
    
    Рассмотрим знакочередующийся функциональный ряд: $\displaystyle \sum_{n = 1}^{\infty} (-1)^n u_n(x), \; u_n(x) \geq 0$ на $D$.
    
    Если $u_n(x) \downarrow_{(n)}$ и $u_n \stackrel{D}{\rightrightarrows} 0$, то ряд сходится равномерно.
        
    % вопрос 78
        
    % вопрос 79
    
    \subsection*{79. Сформулируйте признак Абеля равномерной сходимости функционального ряда.}
        
        Рассмотрим ряд $\displaystyle \sum_{n = 1}^{\infty} a_n(x) b_n(x) = \circledast$.
        
        Если $a_n(x)$ мотонна по $n$ (при $\forall \, x \in D \subseteq \RR$) и $ \| a_n \| \leq C$ при всех $n$,
        
        а ряд $\sum b_n(x)$ сх-ся равномерно, то $\circledast$ сх-ся равномерно.
        
    % вопрос 80
        
    % вопрос 81
        
    % вопрос 82
        
    % вопрос 83
        
    % вопрос 84
        
    % вопрос 85
        
    % вопрос 86
    
    \subsection*{86. Докажите, что если степенной ряд $\displaystyle \sum c_n (x - x_0)^n$ расходится в точке $x_1$, то он расходится во всех точках $x$, для которых $|x - x_0| > |x_1 - x_0|.$}
        
        \begin{proof}
        
        Докажем, что если $\displaystyle \sum c_n (x - x_0)^n$ сходится в точке $x_1$, то он сходится во всех точках $x$, для которых $|x - x_0| < |x_1 - x_0| \;  \circledast.$ Из этого будет следовать сформулированное выше утверждение (методом от противного).
        
        Итак, доказываем $\circledast$. (Будем рассматривать нетривиальный случай $x_1 \neq x_0$, иначе очевидно).
        
        $ \Bigg| \displaystyle \sum_{n = m}^{N} c_n (x - x_0)^n \Bigg| =
        \Bigg| \sum_{n = m}^{N} c_n \cdot (x_1 - x_0)^n \cdot \left( \frac{x - x_0}{x_1 - x_0} \right)^n \Bigg| \leq
        \sum_{n = m}^{N}  \big| c_n \cdot (x_1 - x_0)^n \big| \cdot \bigg| \frac{x - x_0}{x_1 - x_0} \bigg|^n = \bigstar$.
        
        Заметим, что $\big| c_n \cdot (x_1 - x_0)^n \big| < \varepsilon $ при $m \geq n_0 (\varepsilon)$ (следствие из необходимого условия сходимости).
        
        Далее, (при наших условиях) $\sum \bigg| \frac{x - x_0}{x_1 - x_0} \bigg|^n$ образуют геом. прогрессию, где $q = \bigg| \frac{x - x_0}{x_1 - x_0} \bigg| < 1.$
        
        Так что $\bigstar \leq \varepsilon \cdot (q^m + \dots + q^n)
        \leq \varepsilon \cdot  q^m \cdot \frac{1}{1 - q} \to 0.$ 
        
        Почему к нулю? При $m \to \infty $ выражение $q^m \cdot \frac{1}{1 - q}$ остается ограниченным одной и той же константой, а $\varepsilon$ - это произвольная сколь угодно малая величина.
        
        Итог: ряд сходится по критерию Коши.
        
        
        \end{proof}
        
    % вопрос 87
    \subsection*{87. Выведите формулу Коши-Адамара для радиуса сходимости степенного ряда.}
    
    Для степенного ряда $\displaystyle \sum_{n = 0}^{\infty} c_n \cdot (x - x_0)^n,$ где $\{ c_n \}$ - числовая посл-ть, $x_0 \in \RR$ фиксирован,  $x \in \RR$ - переменная, радиус сходимости $R$ вычислим по формуле Коши-Адамара:
    
    \fbox{$R =  \frac{1}{\overline{\lim} \sqrt[n]{|c_n|}}$}
    
        \begin{proof} 
        В нашем ряде $a_n(x) = c_n \cdot (x - x_0)^n.$ Применим радикальный признак Коши: 
        
        $\sqrt[n]{|a_n(x)|} = \sqrt[n]{|c_n|} \cdot |x - x_0| \implies
        \overline{\lim} \sqrt[n]{|a_n(x)|} = \overline{\lim} \sqrt[n]{|c_n|} \cdot |x - x_0| =
        |x - x_0| \cdot \overline{\lim} \sqrt[n]{|c_n|} \implies $ 
        
        если $|x - x_0| \cdot \overline{\lim} \sqrt[n]{|c_n|} < 1,$ то ряд сх-ся;
        
        если $|x - x_0| \cdot \overline{\lim} \sqrt[n]{|c_n|} > 1,$ то ряд расх-ся.
        
        Введем $R := \frac{1}{\overline{\lim} \sqrt[n]{|c_n|}}.$
        
        Из полученных результатов ясно, что $|x - x_0| < R \iff |x - x_0| \cdot \overline{\lim} \sqrt[n]{|c_n|} < 1$ и ряд сходится; 
        
        $|x - x_0| > R \iff |x - x_0| \cdot \overline{\lim} \sqrt[n]{|c_n|} > 1$ и ряд расходится. А это определение радиуса сходимости.
        
        \end{proof}
        
    % вопрос 88
        
    % вопрос 89
        
    % вопрос 90
        
    % вопрос 91
        
    % вопрос 92
        
    % вопрос 93
        
    % вопрос 94
        
    % вопрос 95
        
    % вопрос 96
    
    \subsection*{96. Запишите формулу Тейлора для бесконечно дифференцируемой функции с остаточным членом в формах Лагранжа и Коши.}
    
    Если функция $f(x)$ беск. дифф. в точке $x_0$, то $f(x)$ можно сопоставить в соотв. ее ряд Тейлора:
    
    $\displaystyle \sum_{n = 0}^{\infty} \frac{f^{(n)} (x_0)}{n!} (x - x_0)^n.$ При этом $f(x) = \displaystyle \sum_{n = 0}^{N} \frac{f^{(n)} (x_0)}{n!} (x - x_0)^n + r_N (x).$
    
    Фор-ла Лагранжа: $r_N(x) = \frac{ f^{(N + 1)} (x_0 + \Theta(x - x_0))}{(N + 1)!} (x - x_0)^{N + 1}, \; \Theta \in (0, 1). $
    
    Фор-ла Коши: $r_N(x) = \frac{f^{(N + 1)} (x_0 + \Theta(x - x_0))}{N!} (1 - \Theta)^N (x - x_0)^{N + 1}, \; \Theta \in (0, 1).$
        
        
    % вопрос 97
    
    \subsection*{97. Сформулируйте и докажите утверждение о единственности разложения функции в степенной ряд.}
    
    
    Если $f(x) = \displaystyle \sum_{n = 0}^{\infty} c_n \cdot (x - x_0)^n, \; |x - x_0| < \delta$ (говоря иначе, функция представлена степенным рядом в некой окр-ти $x_0$); то этот степенной ряд - ее ряд Тейлора. 
    
        \begin{proof} 
    \begin{flalign}
    & f^{(k)} (x) = 
    \sum_{n = 0}^{\infty} c_n \cdot n \cdot (n - 1) \dots (n - k + 1) \cdot (x - x_0)^{n - k} = \sum_{n = k}^{\infty} c_n \cdot n \cdot (n - 1) \dots (n - k + 1) \cdot (x - x_0)^{n - k} \implies \\
    & f^{(k)} (x_0) = c_k \cdot k! \implies c_k = \frac{f^{(k)}(x_0)}{k!}.
    \end{flalign}
    
    \textit{(Мы заменили в первом переходе нижнюю границу суммирования с нуля на k, так как все предыдущие слагаемые зануляются)}
    
    То есть функция может быть представлена в виде степенного ряда единственным образом - и это будет ее р.Т.\\
    
        \end{proof}
    % вопрос 98
        
    % вопрос 99
        
    % вопрос 100
        
    % вопрос 101
        
    % вопрос 102

\end{document}
