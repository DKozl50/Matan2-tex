\documentclass[a4paper, fleqn]{article}
\usepackage{header}

\title{Коллоквиум 1}
\author{
        % ОТКОММЕНТИРУЙ СЕБЯ
    % Александр Богданов   \\ \href{https://t.me/SphericalPotatoInVacuum}{Telegram} \and
    % Алиса Вернигор       \\ \href{https://t.me/allisyonok}{Telegram} \and
    % Анастасия Григорьева \\ \href{https://t.me/weifoll}{Telegram} \and
    % Василий Шныпко       \\ \href{https://t.me/yourvash}{Telegram} \and
    % Данил Казанцев       \\ \href{https://t.me/vserosbuybuy}{Telegram} \and
    Денис Козлов         \\ \href{https://t.me/DKozl50}{Telegram} \and
    % Елизавета Орешонок   \\ \href{https://t.me/eaoresh}{Telegram} \and
    % Иван Пешехонов       \\ \href{https://t.me/JohanDDC}{Telegram} \and
    % Иван Добросовестнов  \\ \href{https://t.me/ivankot13}{Telegram} \and
    % Настя Городилова     \\ \href{https://t.me/nastygorodi}{Telegram} \and
    % Никита Насонков      \\ \href{https://t.me/nnv_nick}{Telegram} \and
    Сергей Лоптев        \\ \href{https://t.me/beast_sl}{Telegram}
}

\date{Версия от {\ddmmyyyydate\today} \currenttime}


\begin{document}
    \maketitle
    % Решение вопроса пишем после комментария 
    % комментарии трогать пожалуйста не надо, будет круто
    
    % давайте для постоянства формулировку в сабсекшн кидать
    % доказательства советуют оборачивать в \begin{proof} \end{proof}
    
    % вопрос 1

    % вопрос 2

    % вопрос 3
        
    % вопрос 4
    \subsection*{4. Сформулируйте и докажите признак сравнения положительных числовых рядов, основанный на неравенстве $\frac{a_{n+1}}{a_n} \leq \frac{b_{n+1}}{b_n}$.}
    \begin{proposition}[Сравнение отношений]
        Пусть $\frac{a_{n+1}}{a_n} \leq \frac{b_{n+1}}{b_n}$ при $n \geq n_0$. Тогда:
        \begin{flalign*}
            & \sum b_n \text{ сходится } \implies \sum a_n \text{ сходится }\\
            & \sum a_n \text{ расходится } \implies \sum b_n \text{ расходится }
        \end{flalign*}
    \end{proposition}
    \begin{proof} 
        Предполагаем, что $a_n > 0, b_n > 0$.
        \begin{flalign*}
            & a_{n_0 + 1} \leq \frac{a_{n_0}}{b_{n_0}} \cdot b_{n_0 + 1} \\
            & a_{n_0 + 2} \leq \frac{a_{n_0 + 1}}{b_{n_0 + 1}} \cdot b_{n_0 + 2} \leq \frac{a_{n_0}}{b_{n_0}} \cdot b_{n_0 + 2} \\
            & \cdots \\
            & a_{n_0 + k} \leq \frac{a_{n_0}}{b_{n_0}} \cdot b_{n_0 + k} \\
            & \sum_{n=n_0}^N a_n \leq \frac{a_{n_0}}{b_{n_0}} \cdot \sum_{n=n_0}^N b_n
        \end{flalign*}
    \end{proof}
        
    % вопрос 5
        
    % вопрос 6
        
    % вопрос 7
        
    % вопрос 8
        
    % вопрос 9
        
    % вопрос 10
        
    % вопрос 11
        
    % вопрос 12
        
    % вопрос 13
        
    % вопрос 14
    \subsection*{14. Сформулируйте (предельный) радикальный признак Коши для положительного ряда.}
    \begin{proposition}[Радикальный признак Коши.]
        Пусть $a_n \geq 0$. Тогда:
        \begin{equation*}
            \varlimsup a_n = \begin{dcases}
                < 1 &\implies \text{ ряд } \sum a_n \text{ сход. } \\
                > 1 &\implies \text{ ряд } \sum a_n \text{ расх. }
            \end{dcases}
        \end{equation*}
    \end{proposition}
    \begin{proof}
        Пусть $q = \varlimsup \sqrt[n]{a_n}$.
        
        Пусть $q < 1$.
        \begin{flalign*}
            & \forall \varepsilon > 0\ \exists n_0 : \sqrt[n]{a_n} \leq q + \varepsilon \text{ при } n \geq n_0 \\
            & \forall \varepsilon > 0\ \exists n_0 : a_n \leq \left(q + \varepsilon\right)^n \text{ при } n \geq n_0
        \end{flalign*}
        Пусть $\varepsilon : q + \varepsilon < 1$

        Тогда $\sum a_n$ сходится, поскольку сходится $\sum \left(q + \varepsilon\right)^n$.

        Пусть $q > 1$.
        
        Тогда $\exists \left\{n_k\right\} : \sqrt[n_k]{a_{n_k}} \geq q - \varepsilon$ при $k = k_0, k_0 + 1, \ldots$

        Пусть $\varepsilon : q - \varepsilon \geq 1$

        Тогда $a_{n_k} \geq \left(q - \varepsilon\right)^{n_k} \geq 1 \implies \sum_{k=1}^\infty a_{n_k} = \infty, \sum_{n=1}^\infty a_n \geq \sum_{k=1}^\infty a_{n_k}$
        
        $\implies \text{ ряд } \sum a_n \text{ расходится }$
    \end{proof}

        
    % вопрос 15
        
    % вопрос 16
        
    % вопрос 17
        
    % вопрос 18
        
    % вопрос 19
        
    % вопрос 20
        
    % вопрос 21
        
    % вопрос 22
        
    % вопрос 23
        
    % вопрос 24
    \subsection*{24. Докажите, что ряд сходится абсолютно ровно в том случае, когда сходятся его положительная и отрицательная части.}
    \begin{proposition}
        Ряд $\sum a_n$ сходится абсолютно $\iff \sum a_n^+, \sum a_n^- < \infty$ сходятся. 
    \end{proposition}
    \begin{proof}
        Если $\sum \left|a_n\right| < \infty$, то $S_N^+,\ S_N^-$ ограничены $\implies$ сходятся.

        Если $S_N^+ \to S^+,\ S_N^- \to S^-$, то $\sum_{n=1}^N a_n \to S^+ - S^-,\ \sum_{n=1}^N \left|a_n\right| \to S^+ + S^-$.
    \end{proof}
        
    % вопрос 25
        
    % вопрос 26
        
    % вопрос 27
        
    % вопрос 28
        
    % вопрос 29
        
    % вопрос 30
        
    % вопрос 31
        
    % вопрос 32
        
    % вопрос 33
        
    % вопрос 34
    \subsection*{34. Сформулируйте признак Дирихле. Приведите пример его применения.}
    \begin{proposition}[Признак Дирихле.]
        Если $a_n \searrow 0$ и $\left| \sum_{n=1}^N b_n \right| = \left| B_N \right| \leq C$~--- ограничена, то ряд $\sum a_n b_n$ сходится. 
    \end{proposition}
    \begin{example}
        \begin{flalign*}    
            &\sum_{n=1}^\infty \frac{\sin nx}{n^p},\ x \neq \pi k,\ p > 0 \\
            &a_n = \frac{1}{n^p},\ b_n = \sin nx \\
            &B_n = \sin x + \sin 2x + \ldots + \sin nx = \frac{\cos \frac{x}{2} - \cos\left(\left(n + \frac{1}{2}\right)x\right)}{2 \sin \frac{x}{2}}; \hspace{1cm} \left| B_n \right| \leq \frac{1}{\left| \sin \frac{x}{2} \right|} \\
            &\implies \text{ ряд сходится по признаку Дирихле. }
        \end{flalign*}
    \end{example}
        
    % вопрос 35
        
    % вопрос 36
        
    % вопрос 37
        
    % вопрос 38
        
    % вопрос 39
        
    % вопрос 40
        
    % вопрос 41
        
    % вопрос 42
        
    % вопрос 43
        
    % вопрос 44
    \subsection*{44. Пусть последовательности $\left\{ a_n\right\},\ \left\{A_n\right\},\ A_n \neq 0$ таковы, что $a_n = \frac{A_n}{A_{n-1}} \cdot c_n$ и бесконечное произведение $\prod c_n$ сходится. Докажите, что существует число $C \neq 0$ такое, что $\prod_{n=1}^N a_n = A_N \left(C + o(1)\right)$.}
    \begin{proof}
        \begin{flalign*}
            &a_n = \frac{A_n}{A_{n-1}} \cdot c_n, \hspace{1cm} \prod c_n \text{ сходится, то есть } \prod_{n=1}^N c_n \to P \neq 0 \\
            &\prod_{n=1}^N = \frac{\cancel{A_1}}{A_0} \cdot c_1 \cdot \frac{\cancel{A_2}}{\cancel{A_1}} \cdot c_2 \cdot \ldots \cdot \frac{A_N}{\cancel{A_{N-1}}} \cdot c_N = A_N \cdot \underbrace{\frac{1}{A_0} \cdot \prod_{n=1}^N c_n}_{\to \frac{P}{A_0} \neq 0} \\
            & \implies \prod_{n=1}^N a_n = A_N \cdot \left(C + o(1)\right), \hspace{1cm} C = \frac{P}{A_0} \neq 0
        \end{flalign*}
    \end{proof}
        
    % вопрос 45
        
    % вопрос 46
        
    % вопрос 47
        
    % вопрос 48
        
    % вопрос 49
        
    % вопрос 50
        
    % вопрос 51
        
    % вопрос 52
        
    % вопрос 53
        
    % вопрос 54
    \subsection*{54. Приведите пример функциональной последовательности $\left\{f_n(x)\right\}$ (с нетривиальной зависимостью от $n$ и $x$), равномерно сходящейся на некотором множестве (с обоснованием).}
    \begin{example}
        \begin{flalign*}
            &f_n(x) = \frac{1}{n + x}, \hspace{1cm} D = \left[ 0; +\infty \right) \\
            &f_n(x) \overset{D}{\rightrightarrows} 0 \\
            & \norm{f_n - 0} = \norm{f_n} = \sup_{x \in D} \left| f_n(x) \right| = \frac{1}{n} \to 0 \\
            & \implies \text{ последовательность сходится равномерно.}
        \end{flalign*}
    \end{example}
        
    % вопрос 55
        
    % вопрос 56
        
    % вопрос 57
        
    % вопрос 58
        
    % вопрос 59
        
    % вопрос 60
        
    % вопрос 61
        
    % вопрос 62
        
    % вопрос 63
        
    % вопрос 64
    \subsection*{64. Покажите на примере как доказать неравномерность сходимости функциональной последовательности с помощью локализации особенности (с обоснованием).}

    % вопрос 65
        
    % вопрос 66
        
    % вопрос 67
        
    % вопрос 68
        
    % вопрос 69
        
    % вопрос 70
        
    % вопрос 71
        
    % вопрос 72
        
    % вопрос 73
        
    % вопрос 74
    \subsection*{74. Сформулируйте мажорантный признак Вейерштрасса абсолютной и равномерной сходимости функционального ряда.}
    \begin{proposition}[Признак Вейерштрасса для функционального ряда.]
        Если $\left|a_n(x)\right| \leq b_n$ при $\forall n \geq n_0,\ \forall x \in D$, а ряд $\sum b_n$ сходится, то $\sum a_n(x)$ сходится на $D$ абсолютно и равномерно.
    \end{proposition}
        
    % вопрос 75
        
    % вопрос 76
        
    % вопрос 77
        
    % вопрос 78
        
    % вопрос 79
        
    % вопрос 80
        
    % вопрос 81
        
    % вопрос 82
        
    % вопрос 83
        
    % вопрос 84
    \subsection*{84. Что можно утверждать про равномерную сходимость степенного ряда?}
        
    % вопрос 85
        
    % вопрос 86
        
    % вопрос 87
        
    % вопрос 88
        
    % вопрос 89
        
    % вопрос 90
        
    % вопрос 91
        
    % вопрос 92
        
    % вопрос 93
        
    % вопрос 94
    \subsection*{94. Сформулируйте и докажите теорему о почленном дифференцировании степенного ряда.}
    \begin{theorem}[Почленное дифференцирование степенного ряда.]
        $\sum c_n\left(x - x_0\right)^n$, $R > 0$~--- его радиус сходимости.

        При почленном дифференцировании получаем ряд $\sum_{n=0}^\infty c_n \cdot n \cdot \left(x - \right)^{n - 1}$.

        Его радиус сходимости равен радиусу сходимости исходного ряда, то есть он сходится равномерно при $\left|x - x_0\right| \leq r < R$.
    \end{theorem}
        
    % вопрос 95
        
    % вопрос 96
        
    % вопрос 97
        
    % вопрос 98
        
    % вопрос 99
        
    % вопрос 100
        
    % вопрос 101
        
    % вопрос 102

\end{document}
