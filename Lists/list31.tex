\documentclass[a4paper, fleqn]{article}
\usepackage{header}

\title{Семинарский лист 3.1}
\author{
    % Александр Богданов   \\ \href{https://t.me/SphericalPotatoInVacuum}{Telegram} \and
    % Алиса Вернигор       \\ \href{https://t.me/allisyonok}{Telegram} \and
    % Анастасия Григорьева \\ \href{https://t.me/weifoll}{Telegram} \and
    % Василий Шныпко       \\ \href{https://t.me/yourvash}{Telegram} \and
    % Данил Казанцев       \\ \href{https://t.me/vserosbuybuy}{Telegram} \and
    Денис Козлов         \\ \href{https://t.me/DKozl50}{Telegram} \and
    % Елизавета Орешонок   \\ \href{https://t.me/eaoresh}{Telegram} \and
    % Ира Голобородько     \\ \href{https://t.me/Ira4kgl}{Telegram} \and
    % Иван Пешехонов       \\ \href{https://t.me/JohanDDC}{Telegram} \and
    % Иван Добросовестнов  \\ \href{https://t.me/ivankot13}{Telegram} \and
    % Настя Городилова     \\ \href{https://t.me/nastygorodi}{Telegram} \and
    % Никита Насонков      \\ \href{https://t.me/nnv_nick}{Telegram} \and
    % Сергей Лоптев        \\ \href{https://t.me/beast_sl}{Telegram}
}

\date{Версия от {\ddmmyyyydate\today} \currenttime}

\begin{document}
\maketitle

\section*{Вычислите предел}
% \subsection*{Задача 1}

% \subsection*{Задача 2}

% \subsection*{Задача 3}

% \subsection*{Задача 4}

\section*{Найдите множество значений $y$, при которых интеграл существует в собственном смысле, и
исследуйте на этом множестве на непрерынвость функцию от $y$, заданную интегралом.}
% \subsection*{Задача 5}

% \subsection*{Задача 6}

% \subsection*{Задача 7}

% \subsection*{Задача 8}

\section*{Найдите производную функции, заданной интегралом}
% \subsection*{Задача 9}

% \subsection*{Задача 10}

\subsection*{Задача 11}
\begin{flalign*}
    & J(y) = \int_y^{y^2} e^{-x^2 y} dx\,, \qquad
    \frac{dJ}{dy} = \int_y^{y^2} \frac{\partial }{\partial y} e^{-x^2 y} dx + 
    e^{-x^2 y} \bigg|_{x=y^2} 2y - 
    e^{-x^2 y} \bigg|_{x=y} = \\
    & = \int_y^{y^2} -x^2 e^{-x^2 y} dx + 2y e^{-y^5} - e^{-y^3} \text{ удачи лол}
\end{flalign*}

% \subsection*{Задача 12}

\section*{Покажите, что функция $u(x) = \int_0^{\pi} e^{x \cos t} dt$ удовлетворяет дифееренциальному
уравнению $xu'' + u' - xu = 0$}
\subsection*{Задача 13}
\begin{flalign*}
    & u(x) = \int_0^{\pi} e^{x \cos t} dt\,, \qquad
    u'(x) = \int_0^{\pi} \frac{\partial}{\partial x} e^{x \cos t} dt = 
    \int_0^{\pi} \cos t e^{x \cos t} dt\,, \qquad
    u''(x) = \int_0^{\pi} \frac{\partial^2}{\partial x^2} e^{x \cos t} dt = 
    \int_0^{\pi} \cos^2 t e^{x \cos t} dt \\
    & xu'' + u' - xu = 
    \int_0^{\pi} x \cos^2 t e^{x \cos t} dt + \int_0^{\pi} \cos t e^{x \cos t} dt -
    \int_0^{\pi} x e^{x \cos t} dt = 
    \int_0^{\pi} \left( x \cos^2 t + \cos t - x \right) e^{x \cos t} dt = \\
    & = \int_0^{\pi} \left( \cos t - x \sin^2 t \right) e^{x \cos t} dt = 
    - \int_0^{\pi} x \sin^2 t e^{x \cos t} dt + 
    \int_0^{\pi} \underbrace{\cos t}_{dv} \underbrace{e^{x \cos t}}_{u} dt = \\
    & - \int_0^{\pi} x \sin^2 t e^{x \cos t} dt + 
    \sin t e^{x \cos t} \bigg|_0^{\pi} + \int_0^{\pi} x \sin^2 t e^{x\cos t} dt = 
    \sin t e^{x \cos t} \bigg|_0^{\pi} = 0
\end{flalign*}

\section*{Применяя метод дифференцирования по параметру, вычислите интеграл}
% \subsection*{Задача 14}

\subsection*{Задача 15}
\begin{flalign*}
    & J(p) = \int_0^{\pi} \frac{\ln(1 + p \sin x)}{\sin x} dx\,, \;\; p \in [0, 1] \\
    & \frac{d}{dp} J(p) = \int_0^{\pi} \frac{\partial }{\partial p} \frac{\ln(1 + p \sin x)}{\sin x} dx =
    \int_0^{\pi} \frac{\sin x}{(1 + p \sin x) \sin x}  dx = 
    \int_0^{\pi} \frac{1}{1 + p \sin x} dx \\
    & t = \tg \sfrac{x}{2}, \;\; x = 2 \arctan t, \;\; dx = \frac{2}{1 + t^2} dt, \;\;
    \sin x = \frac{2t}{1 + t^2} \\
    & \int_0^{\infty} \frac{1}{1 + p \sin x} dx = 
    \int_0^{\infty} \frac{2}{1 + t^2} \cdot \frac{1}{1 + \frac{2pt}{1 + t^2}} dt =
    \int_0^{\infty} \frac{2}{t^2 + 2pt + 1 } dt = \int_0^{\infty} \frac{2}{(t + p)^2 + 1 - p^2} dt = 
    \left\{ \begin{array} {rl}
        z & = t + p \\
        dt & = dz
    \end{array}  \right\} = \\
    & \int_p^{\infty} \frac{2}{z^2 + 1 - p^2} dz = 
    2 \frac{1}{\sqrt{1 - p^2}} \arctan \frac{z}{\sqrt{1 - p^2}} \bigg|_p^{\infty} = 
    \frac{2}{\sqrt{1 - p^2}} \left( \frac{\pi}{2} - \arctan \frac{p}{\sqrt{1 - p^2}} \right) =
    \left\{ \begin{array} {rl}
        \sin \alpha & = p \\
        \cos \alpha & = \sqrt{1 - p^2} \\
        \tan \alpha & = \frac{p}{\sqrt{1 - p^2}} 
    \end{array}  \right\} = \\
    & \frac{1}{\sqrt{1 - p^2}} \left( \pi - 2\arcsin p \right) =
    \frac{\pi}{\sqrt{1 - p^2}} - \frac{2\arcsin p}{\sqrt{1 - p^2}} = J'(p) \\
    & J(p) = \int J'(p) dp = \int \left( \frac{\pi}{\sqrt{1 - p^2}} - \frac{2\arcsin p}{\sqrt{1 - p^2}} \right) dp = 
    \pi \arcsin p - \int \frac{2 \arcsin p}{\sqrt{1 - p^2}} dp = \left\{ \begin{array} {rl}
        u &= \arcsin p \\
        du &= \frac{1}{\sqrt{1-p^2}} dp
    \end{array}  \right\} = \\
    & \pi \arcsin p - \int 2 u du = 
    \pi \arcsin p - u^2 + C = 
    \pi \arcsin p - \arcsin^2 p + C = J(p) \\
    & J(p) = \int_0^{\pi} \frac{\ln(1 + p \sin x)}{\sin x} dx, \qquad
    J(0) = \int_0^{\pi} \frac{\ln(1)}{\sin x} dx = 0 \;\; \Rightarrow \;\; 
    \pi \arcsin 0 - \arcsin^2 0 + C = 0 \;\; \Rightarrow \;\; C = 0 \\
    & J(p) = \pi \arcsin p - \arcsin^2 p
\end{flalign*}

% \subsection*{Задача 16}

\subsection*{Задача 17}
\begin{flalign*}
    & J(p) = \int_0^{\pi} \ln \left( 1 + 2p \cos x + p^2 \right) dx \\
    & J'(p) = \int_0^{\pi} \frac{\partial }{\partial p} \ln \left( 1 + 2p \cos x + p^2 \right) dx = 
    \int_0^{\pi} \frac{2p + 2\cos x}{1 + 2p\cos x + p^2} dx
\end{flalign*}

\section*{Применяя интегрирование по параметру под знаком интеграла, вычислите интеграл $a > b > 0$}
\subsection*{Задача 18}


% \subsection*{Задача 19}

\end{document}
