\documentclass[a4paper, fleqn]{article}
\usepackage{header}

\title{Семинарский лист 3.1}
\author{
    % Александр Богданов   \\ \href{https://t.me/SphericalPotatoInVacuum}{Telegram} \and
    % Алиса Вернигор       \\ \href{https://t.me/allisyonok}{Telegram} \and
    % Анастасия Григорьева \\ \href{https://t.me/weifoll}{Telegram} \and
    % Василий Шныпко       \\ \href{https://t.me/yourvash}{Telegram} \and
    % Данил Казанцев       \\ \href{https://t.me/vserosbuybuy}{Telegram} \and
    Денис Козлов         \\ \href{https://t.me/DKozl50}{Telegram} \and
    % Елизавета Орешонок   \\ \href{https://t.me/eaoresh}{Telegram} \and
    % Ира Голобородько     \\ \href{https://t.me/Ira4kgl}{Telegram} \and
    % Иван Пешехонов       \\ \href{https://t.me/JohanDDC}{Telegram} \and
    % Иван Добросовестнов  \\ \href{https://t.me/ivankot13}{Telegram} \and
    % Настя Городилова     \\ \href{https://t.me/nastygorodi}{Telegram} \and
    % Никита Насонков      \\ \href{https://t.me/nnv_nick}{Telegram} \and
    % Сергей Лоптев        \\ \href{https://t.me/beast_sl}{Telegram}
}

\date{Версия от {\ddmmyyyydate\today} \currenttime}

\begin{document}
\maketitle

\section*{Вычислите предел}
% \subsection*{Задача 1}

% \subsection*{Задача 2}

% \subsection*{Задача 3}

% \subsection*{Задача 4}

\section*{Найдите множество значений $y$, при которых интеграл существует в собственном смысле, и
исследуйте на этом множестве на непрерынвость функцию от $y$, заданную интегралом.}
% \subsection*{Задача 5}

% \subsection*{Задача 6}

% \subsection*{Задача 7}

% \subsection*{Задача 8}

\section*{Найдите производную функции, заданной интегралом}
% \subsection*{Задача 9}

% \subsection*{Задача 10}

% \subsection*{Задача 11}

% \subsection*{Задача 12}

\section*{Покажите, что функция $u(x) = \int_0^{\pi} e^{x \cos t} dt$ удовлетворяет дифееренциальному
уравнению $xu'' + u' - xu = 0$}
\subsection*{Задача 13}
\begin{flalign*}
    & u(x) = \int_0^{\pi} e^{x \cos t} dt\,, \qquad
    u'(x) = \int_0^{\pi} \frac{\partial}{\partial x} e^{x \cos t} dt = 
    \int_0^{\pi} \cos t e^{x \cos t} dt\,, \qquad
    u''(x) = \int_0^{\pi} \frac{\partial^2}{\partial x^2} e^{x \cos t} dt = 
    \int_0^{\pi} \cos^2 t e^{x \cos t} dt \\
    & xu'' + u' - xu = 
    \int_0^{\pi} x \cos^2 t e^{x \cos t} dt + \int_0^{\pi} \cos t e^{x \cos t} dt -
    \int_0^{\pi} x e^{x \cos t} dt = 
    \int_0^{\pi} \left( x \cos^2 t + \cos t - x \right) e^{x \cos t} dt = \\
    & = \int_0^{\pi} \left( \cos t - x \sin^2 t \right) e^{x \cos t} dt = 
    - \int_0^{\pi} x \sin^2 t e^{x \cos t} dt + 
    \int_0^{\pi} \underbrace{\cos t}_{dv} \underbrace{e^{x \cos t}}_{u} dt = \\
    & - \int_0^{\pi} x \sin^2 t e^{x \cos t} dt + 
    \sin t e^{x \cos t} \bigg|_0^{\pi} + \int_0^{\pi} x \sin^2 t e^{x\cos t} dt = 
    \sin t e^{x \cos t} \bigg|_0^{\pi} = 0
\end{flalign*}

\section*{Применяя метод дифференцирования по параметру, вычислите интеграл}
% \subsection*{Задача 14}

% \subsection*{Задача 15}

% \subsection*{Задача 16}

% \subsection*{Задача 17}

\section*{Применяя интегрирование по параметру под знаком интеграла, вычислите интеграл $a > b > 0$}
% \subsection*{Задача 18}

% \subsection*{Задача 19}

\end{document}
