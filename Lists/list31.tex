\documentclass[a4paper, fleqn]{article}
\usepackage{header}

\title{Семинарский лист 3.1}
\author{
    % Александр Богданов   \\ \href{https://t.me/SphericalPotatoInVacuum}{Telegram} \and
    % Алиса Вернигор       \\ \href{https://t.me/allisyonok}{Telegram} \and
    % Анастасия Григорьева \\ \href{https://t.me/weifoll}{Telegram} \and
    % Василий Шныпко       \\ \href{https://t.me/yourvash}{Telegram} \and
    % Данил Казанцев       \\ \href{https://t.me/vserosbuybuy}{Telegram} \and
    Денис Козлов         \\ \href{https://t.me/DKozl50}{Telegram} \and
    Елизавета Орешонок   \\ \href{https://t.me/eaoresh}{Telegram} \and
    % Ира Голобородько     \\ \href{https://t.me/Ira4kgl}{Telegram} \and
    % Иван Пешехонов       \\ \href{https://t.me/JohanDDC}{Telegram} \and
    % Иван Добросовестнов  \\ \href{https://t.me/ivankot13}{Telegram} \and
    % Настя Городилова     \\ \href{https://t.me/nastygorodi}{Telegram} \and
    % Никита Насонков      \\ \href{https://t.me/nnv_nick}{Telegram} \and
    % Сергей Лоптев        \\ \href{https://t.me/beast_sl}{Telegram}
}

\date{Версия от {\ddmmyyyydate\today} \currenttime}

\begin{document}
\maketitle

\section*{Вычислите предел}
\subsection*{Задача 1}
    \begin{flalign*}
        & \lim_{y \to 0} \int\limits_{y}^{\sqrt3 + y} \dfrac{dx}{1 + x^2 + y^2} \\[5 pt]
        & \text{Сделаем замену $x = t + y, \, dx = dt$, перейдя к собственному интегралу:} \\
        & \lim_{y \to 0} \int\limits_{0}^{\sqrt3} \dfrac{dt}{1 + (t + y)^2 + y^2}
        = \lim_{y \to 0} \int\limits_{0}^{\sqrt3} \dfrac{dt}{1 + 2ty +2y^2 + t^2} \\
        & \text{Функция Ф$(t;y) = \dfrac{dt}{1 + 2ty +2y^2 + t^2} \,$ непрерывна на прямоугольнике $[0; \sqrt3] \times [-0.5; 0.5] \Rightarrow $} \\
        & \Rightarrow \text{переходим к } y = 0: \: \int\limits_{0}^{\sqrt3} \dfrac{dt}{1 + t^2} = \arctg t \Bigm|_{0}^{\sqrt3}
        = \dfrac{\pi}3 - 0 = \dfrac{\pi}3
    \end{flalign*}
    
\subsection*{Задача 2}
    \begin{flalign*}
        & \lim_{y \to +0} \int\limits_{0}^{1} \dfrac{x}y\, e^{-x^2 / y} dx \\[5 pt]
        & \text{Сделаем замену $t = \dfrac{x^2}y, \, dt = \dfrac{2x\,dx}y \Leftrightarrow dx = \dfrac{y}{3x}\,dt$:} \\
        & \lim_{y \to +0} \int\limits_{0}^{1/y} \dfrac12\, e^{-t} dt = -\dfrac12\, \lim_{y \to +0} e^{-t} \Bigm|_{0}^{1/y}
        = -\dfrac12\, \lim_{y \to +0} \left( e^{-\frac1y} - 1 \right) = -\dfrac12 \cdot (0 - 1) = \dfrac12
    \end{flalign*}
    
\subsection*{Задача 3}
    \begin{flalign*}
        & \lim_{y \to +\infty} \int\limits_{1}^{2} \dfrac{\ln (x + y)}{\ln (x^2 + y^2)} dx, \;\; G = [1; 2] \times [1; +\infty) \\[5 pt]
        & \text{Область не ограничена (бесконечна по $y$), значит, функция $f(x, y)$ должна быть равномерно непрерывна.} \\
        & \text{Функция равномерно непрерывна, если её частные производные ограничены:} \\
        & \dfrac{\partial}{\partial x} \dfrac{\ln (x + y)}{\ln (x^2 + y^2)} 
        = \dfrac{\dfrac{\ln (x^2 + y^2)}{x+ y} - \dfrac{2x \ln (x + y)}{x^2 + y^2}}{\ln^2 (x^2 + y^2)}, \; 
        \lim_{y \to \infty} \dfrac{\dfrac{\ln (x^2 + y^2)}{x+ y} - \dfrac{2x \ln (x + y)}{x^2 + y^2}}{\ln^2 (x^2 + y^2)} = 0 \\
        & \dfrac{\partial}{\partial y} \dfrac{\ln (x + y)}{\ln (x^2 + y^2)} 
        = \dfrac{\dfrac{\ln (x^2 + y^2)}{x+ y} - \dfrac{2y \ln (x + y)}{x^2 + y^2}}{\ln^2 (x^2 + y^2)}, \; 
        \lim_{y \to \infty} \dfrac{\dfrac{\ln (x^2 + y^2)}{x+ y} - \dfrac{2y \ln (x + y)}{x^2 + y^2}}{\ln^2 (x^2 + y^2)} = 0 \\
        & \dfrac{\partial f}{\partial x}, \dfrac{\partial f}{\partial x} \text{ ограничены на } G \Rightarrow f \text{ равном. непр. на } G. \\
        & \lim_{y \to +\infty} \int\limits_{1}^{2} \dfrac{\ln (x + y)}{\ln (x^2 + y^2)}\, dx
        = \int\limits_{1}^{2}  \lim_{y \to +\infty} \dfrac{\ln (x + y)}{\ln (x^2 + y^2)}\, dx
        = \int\limits_{1}^{2}  \lim_{y \to +\infty} \dfrac{\dfrac1{x + y}}{\dfrac{2y}{x^2 + y^2}}\, dx = \\
        & = \int\limits_{1}^{2}  \lim_{y \to +\infty} \dfrac{(x + y)^2 - 2xy}{2y(x + y)}\, dx
        = \int\limits_{1}^{2}  \lim_{y \to +\infty} \left( \dfrac{x + y}{2y} - \dfrac{x}{x + y} \right) dx
        = \dfrac12 \int\limits_{1}^{2} dx = \dfrac{x}2 \Bigm|_{1}^{2} = \dfrac12
    \end{flalign*}

% \subsection*{Задача 4}

\section*{Найдите множество значений $y$, при которых интеграл существует в собственном смысле, и
исследуйте на этом множестве на непрерынвость функцию от $y$, заданную интегралом.}
\subsection*{Задача 5}
    \begin{flalign*}
        & f(x, y) = x^y \text{ интегрируема по Риману при } y > 0, \\
        & \int\limits_{0}^{1} x^y dx = \dfrac1{y + 1} x^{y + 1} \Bigm|_{0}^{1} = \dfrac1{y + 1} \text{ --- непрерывна при } y \ne -1 
        \Rightarrow \text{непрерывна при } y > 0 
    \end{flalign*}
    
\subsection*{Задача 6}
    \begin{flalign*}
        & f(x, y) = \ln(x^2 + y^2) \text{ интегрируема по Риману при } y \ne 0, \\
        & \int\limits_{0}^{1} \ln(x^2 + y^2) dx \text{ непрерывна при } y \ne 0
    \end{flalign*}

% \subsection*{Задача 7}

% \subsection*{Задача 8}

\section*{Найдите производную функции, заданной интегралом}
% \subsection*{Задача 9}

% \subsection*{Задача 10}

\subsection*{Задача 11}
\begin{flalign*}
    & J(y) = \int_y^{y^2} e^{-x^2 y} dx\,, \qquad
    \frac{dJ}{dy} = \int_y^{y^2} \frac{\partial }{\partial y} e^{-x^2 y} dx + 
    e^{-x^2 y} \bigg|_{x=y^2} 2y - 
    e^{-x^2 y} \bigg|_{x=y} = \\
    & = \int_y^{y^2} -x^2 e^{-x^2 y} dx + 2y e^{-y^5} - e^{-y^3} \text{ удачи лол}
\end{flalign*}

% \subsection*{Задача 12}

\section*{Покажите, что функция $u(x) = \int_0^{\pi} e^{x \cos t} dt$ удовлетворяет дифееренциальному
уравнению $xu'' + u' - xu = 0$}
\subsection*{Задача 13}
\begin{flalign*}
    & u(x) = \int_0^{\pi} e^{x \cos t} dt\,, \qquad
    u'(x) = \int_0^{\pi} \frac{\partial}{\partial x} e^{x \cos t} dt = 
    \int_0^{\pi} \cos t e^{x \cos t} dt\,, \qquad
    u''(x) = \int_0^{\pi} \frac{\partial^2}{\partial x^2} e^{x \cos t} dt = 
    \int_0^{\pi} \cos^2 t e^{x \cos t} dt \\
    & xu'' + u' - xu = 
    \int_0^{\pi} x \cos^2 t e^{x \cos t} dt + \int_0^{\pi} \cos t e^{x \cos t} dt -
    \int_0^{\pi} x e^{x \cos t} dt = 
    \int_0^{\pi} \left( x \cos^2 t + \cos t - x \right) e^{x \cos t} dt = \\
    & = \int_0^{\pi} \left( \cos t - x \sin^2 t \right) e^{x \cos t} dt = 
    - \int_0^{\pi} x \sin^2 t e^{x \cos t} dt + 
    \int_0^{\pi} \underbrace{\cos t}_{dv} \underbrace{e^{x \cos t}}_{u} dt = \\
    & - \int_0^{\pi} x \sin^2 t e^{x \cos t} dt + 
    \sin t e^{x \cos t} \bigg|_0^{\pi} + \int_0^{\pi} x \sin^2 t e^{x\cos t} dt = 
    \sin t e^{x \cos t} \bigg|_0^{\pi} = 0
\end{flalign*}

\section*{Применяя метод дифференцирования по параметру, вычислите интеграл}
% \subsection*{Задача 14}

\subsection*{Задача 15}
\begin{flalign*}
    & J(p) = \int_0^{\pi} \frac{\ln(1 + p \sin x)}{\sin x} dx\,, \;\; p \in [0, 1] \\
    & \frac{d}{dp} J(p) = \int_0^{\pi} \frac{\partial }{\partial p} \frac{\ln(1 + p \sin x)}{\sin x} dx =
    \int_0^{\pi} \frac{\sin x}{(1 + p \sin x) \sin x}  dx = 
    \int_0^{\pi} \frac{1}{1 + p \sin x} dx \\
    & t = \tg \sfrac{x}{2}, \;\; x = 2 \arctan t, \;\; dx = \frac{2}{1 + t^2} dt, \;\;
    \sin x = \frac{2t}{1 + t^2} \\
    & \int_0^{\infty} \frac{1}{1 + p \sin x} dx = 
    \int_0^{\infty} \frac{2}{1 + t^2} \cdot \frac{1}{1 + \frac{2pt}{1 + t^2}} dt =
    \int_0^{\infty} \frac{2}{t^2 + 2pt + 1 } dt = \int_0^{\infty} \frac{2}{(t + p)^2 + 1 - p^2} dt = 
    \left\{ \begin{array} {rl}
        z & = t + p \\
        dt & = dz
    \end{array}  \right\} = \\
    & \int_p^{\infty} \frac{2}{z^2 + 1 - p^2} dz = 
    2 \frac{1}{\sqrt{1 - p^2}} \arctan \frac{z}{\sqrt{1 - p^2}} \bigg|_p^{\infty} = 
    \frac{2}{\sqrt{1 - p^2}} \left( \frac{\pi}{2} - \arctan \frac{p}{\sqrt{1 - p^2}} \right) =
    \left\{ \begin{array} {rl}
        \sin \alpha & = p \\
        \cos \alpha & = \sqrt{1 - p^2} \\
        \tan \alpha & = \frac{p}{\sqrt{1 - p^2}} 
    \end{array}  \right\} = \\
    & \frac{1}{\sqrt{1 - p^2}} \left( \pi - 2\arcsin p \right) =
    \frac{\pi}{\sqrt{1 - p^2}} - \frac{2\arcsin p}{\sqrt{1 - p^2}} = J'(p) \\
    & J(p) = \int J'(p) dp = \int \left( \frac{\pi}{\sqrt{1 - p^2}} - \frac{2\arcsin p}{\sqrt{1 - p^2}} \right) dp = 
    \pi \arcsin p - \int \frac{2 \arcsin p}{\sqrt{1 - p^2}} dp = \left\{ \begin{array} {rl}
        u &= \arcsin p \\
        du &= \frac{1}{\sqrt{1-p^2}} dp
    \end{array}  \right\} = \\
    & \pi \arcsin p - \int 2 u du = 
    \pi \arcsin p - u^2 + C = 
    \pi \arcsin p - \arcsin^2 p + C = J(p) \\
    & J(p) = \int_0^{\pi} \frac{\ln(1 + p \sin x)}{\sin x} dx, \qquad
    J(0) = \int_0^{\pi} \frac{\ln(1)}{\sin x} dx = 0 \;\; \Rightarrow \;\; 
    \pi \arcsin 0 - \arcsin^2 0 + C = 0 \;\; \Rightarrow \;\; C = 0 \\
    & J(p) = \pi \arcsin p - \arcsin^2 p
\end{flalign*}

% \subsection*{Задача 16}

\subsection*{Задача 17}
\begin{flalign*}
    & J(p) = \int_0^{\pi} \ln \left( 1 + 2p \cos x + p^2 \right) dx \\
    & J'(p) = \int_0^{\pi} \frac{\partial }{\partial p} \ln \left( 1 + 2p \cos x + p^2 \right) dx = 
    \int_0^{\pi} \frac{2p + 2\cos x}{1 + 2p\cos x + p^2} dx
\end{flalign*}

\section*{Применяя интегрирование по параметру под знаком интеграла, вычислите интеграл $a > b > 0$}
\subsection*{Задача 18}
\begin{flalign*}
    & \int_0^1 \frac{x^b - x^a}{\ln x} dx \qquad
    \text{Интегрирование по параметру — значит нужно найти типа первообразной подынтегральной} \\
    & \text{Рассмотрим } f(x, y) = x^y\,, \quad
    \int_a^b f dy = \int_a^b x^y dy = \int_a^b e^{y \ln x} dy = \frac{e^{y \ln x}}{\ln x} \bigg|_a^b = 
    \frac{e^{b\ln x} - e^{a \ln x}}{\ln x} = \frac{x^b - x^a}{\ln x} \\
    & \int_0^1 \frac{x^b - x^a}{\ln x} dx = \int_0^1 \int_a^b x^y dy dx = \int_a^b \int_0^1 x^y dx dy = 
    \int_a^b \frac{x^{y+1}}{y+1} \bigg|_0^1 dy = 
    \int_a^b \frac{1}{y+1} dy = \ln(y+1) \big|_a^b = \ln \left( \frac{b + 1}{a + 1} \right) 
\end{flalign*}

% \subsection*{Задача 19}

\end{document}
