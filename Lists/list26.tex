\documentclass[a4paper, fleqn]{article}
\usepackage{header}

\title{Семинарский лист 2.6}
\author{
    % Александр Богданов   \\ \href{https://t.me/SphericalPotatoInVacuum}{Telegram} \and
    % Алиса Вернигор       \\ \href{https://t.me/allisyonok}{Telegram} \and
    % Анастасия Григорьева \\ \href{https://t.me/weifoll}{Telegram} \and
    % Василий Шныпко       \\ \href{https://t.me/yourvash}{Telegram} \and
    % Данил Казанцев       \\ \href{https://t.me/vserosbuybuy}{Telegram} \and
    Денис Козлов         \\ \href{https://t.me/DKozl50}{Telegram} \and
    % Елизавета Орешонок   \\ \href{https://t.me/eaoresh}{Telegram} \and
    % Ира Голобородько     \\ \href{https://t.me/Ira4kgl}{Telegram} \and
    % Иван Пешехонов       \\ \href{https://t.me/JohanDDC}{Telegram} \and
    % Иван Добросовестнов  \\ \href{https://t.me/ivankot13}{Telegram} \and
    % Настя Городилова     \\ \href{https://t.me/nastygorodi}{Telegram} \and
    % Никита Насонков      \\ \href{https://t.me/nnv_nick}{Telegram} \and
    % Сергей Лоптев        \\ \href{https://t.me/beast_sl}{Telegram}
}

\date{Версия от {\ddmmyyyydate\today} \currenttime}

\begin{document}
    \maketitle
    
    \section*{Переходя к полярным или обощенным полярным координатам, вычислите площадь фигуры, ограниченной кривой.}
    % \subsection*{Задача 1}
    
    % \subsection*{Задача 2}
    
    % \subsection*{Задача 3}
    
    % \subsection*{Задача 4}
    
    % \subsection*{Задача 5}
    
    % \subsection*{Задача 6}

    \section*{Найдите объем тела, заданного неравенствами.}
    % \subsection*{Задача 7}
    
    % \subsection*{Задача 8}
    
    % \subsection*{Задача 9}
    
    % \subsection*{Задача 10}
    
    \subsection*{Задача 11}
    \begin{flalign*}
        & x^2 + y^2 + z^2 \leq 3, \;\; x^2 + y^2 \leq 2z
        \;\; \Rightarrow \;\;
        r^2 \leq 3, \;\; r^2 \cos^2 \theta \leq 2 r \cos^2 \theta 
        \;\; \Rightarrow \;\;
        r \leq 2 \frac{\cos \theta}{\sin^2 \theta}, r \leq \sqrt{3}\\
        & \text{Хотим найти пересечение параболоида и сферы для интегрирования} \\
        & x^2 + y^2 + z^2 - 3 = x^2 + y^2 - 2z \;\; \Rightarrow \;\; z^2 + 2z - 3 = 0 \Rightarrow z = 1 
        \textit{ отрицательные не подойдут } \Rightarrow \cos \theta = \frac{1}{\sqrt{3}} \\
        & V = \int\limits_0^{2\pi} d \varphi \int\limits_0^{\arccos \frac{1}{\sqrt{3}}} \sin \theta d\theta 
        \int\limits_0^{2 \frac{\cos \theta}{\sin^2 \theta}} r^2 dr + 
        \int\limits_0^{2 \pi} d \varphi \int\limits_{\arccos \frac{1}{\sqrt{3}}}^{\frac{\pi}{2}} \sin \theta d \theta
        \int\limits_0^{\sqrt{3}} r^2 dr = \\
        & = 2 \pi \left(
        \int\limits_0^{\arccos{\frac{1}{\sqrt{3}}}} \frac{8}{3} \frac{\cos^3 \theta}{\sin^6 \theta} \sin \theta d \theta +
        \int\limits_{\arccos \frac{1}{\sqrt{3}}}^{\frac{\pi}{2}} \sqrt{3} \sin \theta d \theta
        \right) = \left\{ \begin{array} {rl}
            t & = \sin \theta \\
            \cos^2 \theta & = 1 - t^2 \\
            dt & = \cos \theta d\theta \\
        \end{array}  \right\} = \\
        & = 2 \pi \left( \frac{8}{3} \int\limits_0^{\arccos \frac{1}{\sqrt{3}}} \frac{1 - t^2}{t^5} dt + 
        \sqrt{3} \left( 0 + \frac{1}{\sqrt{3}}  \right) 
        \right) = \\
        & = 2 \pi \left( 1 + 
        \frac{8}{3} \left( - \frac{1}{6\arccos^6 \frac{1}{\sqrt{3}}} + \frac{1}{4 \arccos^4 \frac{1}{\sqrt{3}} }  \right)  \right) 
    \end{flalign*}
    
    % \subsection*{Задача 12}
    
    % \subsection*{Задача 13}
    
    % \subsection*{Задача 14}
    
\end{document}
