\documentclass[a4paper, fleqn]{article}
\usepackage{header}

\title{Семинарский лист 2.6}
\author{
    % Александр Богданов   \\ \href{https://t.me/SphericalPotatoInVacuum}{Telegram} \and
    % Алиса Вернигор       \\ \href{https://t.me/allisyonok}{Telegram} \and
    % Анастасия Григорьева \\ \href{https://t.me/weifoll}{Telegram} \and
    % Василий Шныпко       \\ \href{https://t.me/yourvash}{Telegram} \and
    % Данил Казанцев       \\ \href{https://t.me/vserosbuybuy}{Telegram} \and
    % Денис Козлов         \\ \href{https://t.me/DKozl50}{Telegram} \and
    % Елизавета Орешонок   \\ \href{https://t.me/eaoresh}{Telegram} \and
    % Ира Голобородько     \\ \href{https://t.me/Ira4kgl}{Telegram} \and
    % Иван Пешехонов       \\ \href{https://t.me/JohanDDC}{Telegram} \and
    % Иван Добросовестнов  \\ \href{https://t.me/ivankot13}{Telegram} \and
    % Настя Городилова     \\ \href{https://t.me/nastygorodi}{Telegram} \and
    % Никита Насонков      \\ \href{https://t.me/nnv_nick}{Telegram} \and
    % Сергей Лоптев        \\ \href{https://t.me/beast_sl}{Telegram}
}

\date{Версия от {\ddmmyyyydate\today} \currenttime}

\begin{document}
    \maketitle
    
    Перейдя к полярным или обобщенным полярным координатам, вычислить площадь фигуры, ограниченной кривой.
    \subsection*{Задача 1}
    \begin{align*}
        & (x^2 + y^2)^2 = 2x^3 \\
        & \text{В полярных координатах: } x = r \cos \varphi, \, y = r \sin \varphi, |J| = r \\
        & (x^2 + y^2)^2 = 2x^3 \; \Leftrightarrow \; r^4 = 2r^3 \cos^3 \varphi \; 
        \Leftrightarrow \; r = 2\cos^3\varphi \; \Leftrightarrow \; \cos \varphi = \sqrt[3]{\dfrac{r}2} \\
        & r \ge 0 \; \Rightarrow \; \cos \varphi \ge 0 \; \Rightarrow \; \varphi \in \left[ -\frac{\pi}2; \frac{\pi}2 \right] \\
        & \text{Площадь фигуры: } \\
        & S = \dfrac12 \int\limits_{-\frac{\pi}2}^{\frac{\pi}2} r^2(\varphi)\, d\varphi = 
        \dfrac12 \int\limits_{-\frac{\pi}2}^{\frac{\pi}2} 4\cos^6\varphi\, d\varphi = 
        2 \int\limits_{-\frac{\pi}2}^{\frac{\pi}2} \cos^6\varphi\, d\varphi =
        \dfrac{\cos^5\varphi \sin\varphi}3 \Bigm|_{-\frac{\pi}2}^{\frac{\pi}2} + \dfrac53 \int\limits_{-\frac{\pi}2}^{\frac{\pi}2} \cos^4\varphi\, d\varphi = \\
        & = \dfrac{5\cos^3\varphi \sin\varphi}{12} \Bigm|_{-\frac{\pi}2}^{\frac{\pi}2} + \dfrac54 \int\limits_{-\frac{\pi}2}^{\frac{\pi}2} \cos^2\varphi\, d\varphi =
        \dfrac{5\cos\varphi \sin\varphi}{8} \Bigm|_{-\frac{\pi}2}^{\frac{\pi}2} + \dfrac58 \int\limits_{-\frac{\pi}2}^{\frac{\pi}2} d\varphi = \dfrac{5\pi}8
    \end{align*}
    
    % \subsection*{Задача 2}
    
    \subsection*{Задача 3}
    \begin{align*}
        & (x^2 + y^2)^2 = xy \\
        & \text{В полярных координатах: } x = r \cos \varphi, \, y = r \sin \varphi \\
        & (x^2 + y^2)^2 = xy \; \Leftrightarrow \; r^4 = r^2 \sin\varphi \cos\varphi \; 
        \Leftrightarrow \; r^2 = \dfrac12 \sin 2\varphi \; \Leftrightarrow \; \sin 2\varphi = 2r^2 \\
        & r^2 \ge 0 \; \Rightarrow \; \sin 2\varphi \ge 0 \; \Rightarrow \; \varphi \in \left[ 0; \frac{\pi}2 \right] \cup \left[ \pi; \frac{3\pi}2 \right]\\
        & \text{Площадь фигуры: } \\
        & S = \dfrac12 \left( \int\limits_{0}^{\frac{\pi}2} r^2(\varphi)\, d\varphi + \int\limits_{\pi}^{\frac{3\pi}2} r^2(\varphi)\, d\varphi \right) = 
         \dfrac12 \left( \int\limits_{0}^{\frac{\pi}2} \sin 2\varphi\, d\varphi + \int\limits_{\pi}^{\frac{3\pi}2} \sin 2\varphi\, d\varphi \right) = \\
        & = -\dfrac14 \left( \cos 2\varphi \Bigm|_{0}^{\frac{\pi}2} + \cos 2\varphi \Bigm|_{\pi}^{\frac{3\pi}2} \right) = 
        -\dfrac14 \left( -1 - 1 - 1 - 1 \right) = 1
    \end{align*}
    
    % \subsection*{Задача 4}
    
    % \subsection*{Задача 5}
    
    % \subsection*{Задача 6}
    
    % \subsection*{Задача 7}
    
    % \subsection*{Задача 8}
    
    % \subsection*{Задача 9}
    
    % \subsection*{Задача 10}
    
    % \subsection*{Задача 11}
    
    % \subsection*{Задача 12}
    
    % \subsection*{Задача 13}
    
    % \subsection*{Задача 14}
    
\end{document}
