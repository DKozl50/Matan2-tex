\documentclass[a4paper, fleqn]{article}
\usepackage{header}

\title{Семинарский лист 2.6}
\author{
    % Александр Богданов   \\ \href{https://t.me/SphericalPotatoInVacuum}{Telegram} \and
    % Алиса Вернигор       \\ \href{https://t.me/allisyonok}{Telegram} \and
    % Анастасия Григорьева \\ \href{https://t.me/weifoll}{Telegram} \and
    % Василий Шныпко       \\ \href{https://t.me/yourvash}{Telegram} \and
    % Данил Казанцев       \\ \href{https://t.me/vserosbuybuy}{Telegram} \and
    % Денис Козлов         \\ \href{https://t.me/DKozl50}{Telegram} \and
    Елизавета Орешонок   \\ \href{https://t.me/eaoresh}{Telegram} \and
    % Ира Голобородько     \\ \href{https://t.me/Ira4kgl}{Telegram} \and
    % Иван Пешехонов       \\ \href{https://t.me/JohanDDC}{Telegram} \and
    % Иван Добросовестнов  \\ \href{https://t.me/ivankot13}{Telegram} \and
    % Настя Городилова     \\ \href{https://t.me/nastygorodi}{Telegram} \and
    % Никита Насонков      \\ \href{https://t.me/nnv_nick}{Telegram} \and
    % Сергей Лоптев        \\ \href{https://t.me/beast_sl}{Telegram}
}

\date{Версия от {\ddmmyyyydate\today} \currenttime}

\begin{document}
    \maketitle
    

    Перейдя к полярным или обобщенным полярным координатам, вычислить площадь фигуры, ограниченной кривой.
    \subsection*{Задача 1}
    \begin{align*}
        & (x^2 + y^2)^2 = 2x^3 \\
        & \text{В полярных координатах: } x = r \cos \varphi, \, y = r \sin \varphi, |J| = r \\
        & (x^2 + y^2)^2 = 2x^3 \; \Leftrightarrow \; r^4 = 2r^3 \cos^3 \varphi \; 
        \Leftrightarrow \; r = 2\cos^3\varphi \; \Leftrightarrow \; \cos \varphi = \sqrt[3]{\dfrac{r}2} \\
        & r \ge 0 \; \Rightarrow \; \cos \varphi \ge 0 \; \Rightarrow \; \varphi \in \left[ -\frac{\pi}2; \frac{\pi}2 \right] \\
        & \text{Площадь фигуры: } \\
        & S = \dfrac12 \int\limits_{-\frac{\pi}2}^{\frac{\pi}2} r^2(\varphi)\, d\varphi = 
        \dfrac12 \int\limits_{-\frac{\pi}2}^{\frac{\pi}2} 4\cos^6\varphi\, d\varphi = 
        2 \int\limits_{-\frac{\pi}2}^{\frac{\pi}2} \cos^6\varphi\, d\varphi =
        \dfrac{\cos^5\varphi \sin\varphi}3 \Bigm|_{-\frac{\pi}2}^{\frac{\pi}2} + \dfrac53 \int\limits_{-\frac{\pi}2}^{\frac{\pi}2} \cos^4\varphi\, d\varphi = \\
        & = \dfrac{5\cos^3\varphi \sin\varphi}{12} \Bigm|_{-\frac{\pi}2}^{\frac{\pi}2} + \dfrac54 \int\limits_{-\frac{\pi}2}^{\frac{\pi}2} \cos^2\varphi\, d\varphi =
        \dfrac{5\cos\varphi \sin\varphi}{8} \Bigm|_{-\frac{\pi}2}^{\frac{\pi}2} + \dfrac58 \int\limits_{-\frac{\pi}2}^{\frac{\pi}2} d\varphi = \dfrac{5\pi}8
    \end{align*}
    
    \subsection*{Задача 2}

    $\underline{(x^2 + y^2)^3 = x^4 + y^4}$

    $\bullet \; $ Используем пол. координаты: $\begin{cases}
    x = r \cos \varphi;\\
    y = r \sin \varphi \\
    \end{cases}$


    $\bullet \; $ Кривая становится $r^6 = r^4 \cos ^4 \varphi  + r^4 \sin ^4 \varphi \iff r^2 = \cos^4 \varphi + \sin^4 \varphi = (\cos^2 \varphi + \sin^2 \varphi)^2 - 2 \cos^2 \varphi \, \sin^2 \varphi = \\ 1 - \frac{\sin^2 2 \varphi}{2} \implies r = \sqrt{1 - \frac{\sin^2 2 \varphi}{2}}.$

    $\bullet \; $ $\displaystyle S = \int\limits_{0}^{2 \pi} d \varphi \int\limits_{0}^{\sqrt{1 - \sin^2 2 \varphi/2}} \underbrace{r}_{\text{Якобиан}} \; dr = \int\limits_{0}^{2 \pi} \frac{1}{2} \cdot \left( 1 - \frac{\sin^2 2 \varphi}{2} \right) \; d \varphi = \frac{1}{2} \int \limits_{0}^{2 \pi} 1 -  \frac{(1/2 - 1/2 \cos 4 \varphi)}{2} = \frac{1}{4} \int \limits_{0}^{2 \pi} \frac{3}{2} + \frac{\cos 4 \varphi}{2} \; d \varphi  = \frac{3 \pi}{4} + \frac{1}{16} \int\limits_{0}^{8 \pi} \cos \Theta \; d \Theta =\boxed{ \frac{3 \pi}{4}} \; .$

    
    \subsection*{Задача 3}
    \begin{align*}
        & (x^2 + y^2)^2 = xy \\
        & \text{В полярных координатах: } x = r \cos \varphi, \, y = r \sin \varphi \\
        & (x^2 + y^2)^2 = xy \; \Leftrightarrow \; r^4 = r^2 \sin\varphi \cos\varphi \; 
        \Leftrightarrow \; r^2 = \dfrac12 \sin 2\varphi \; \Leftrightarrow \; \sin 2\varphi = 2r^2 \\
        & r^2 \ge 0 \; \Rightarrow \; \sin 2\varphi \ge 0 \; \Rightarrow \; \varphi \in \left[ 0; \frac{\pi}2 \right] \cup \left[ \pi; \frac{3\pi}2 \right]\\
        & \text{Площадь фигуры: } \\
        & S = \dfrac12 \left( \int\limits_{0}^{\frac{\pi}2} r^2(\varphi)\, d\varphi + \int\limits_{\pi}^{\frac{3\pi}2} r^2(\varphi)\, d\varphi \right) = 
         \dfrac12 \left( \int\limits_{0}^{\frac{\pi}2} \sin 2\varphi\, d\varphi + \int\limits_{\pi}^{\frac{3\pi}2} \sin 2\varphi\, d\varphi \right) = \\
        & = -\dfrac14 \left( \cos 2\varphi \Bigm|_{0}^{\frac{\pi}2} + \cos 2\varphi \Bigm|_{\pi}^{\frac{3\pi}2} \right) = 
        -\dfrac14 \left( -1 - 1 - 1 - 1 \right) = 1
    \end{align*}
    
    % \subsection*{Задача 4}
    
    \subsection*{Задача 5}
    \begin{align*}
        & (x^2 + y^2)^3 = x^4y \\
        & \text{В полярных координатах: } x = r \cos \varphi, \, y = r \sin \varphi \\
        & (x^2 + y^2)^3 = x^4y \; \Leftrightarrow \; r^6 = r^5 \sin\varphi \cos^4\varphi \; 
        \Leftrightarrow \; r = \sin\varphi \cos^4\varphi \\
        & r \ge 0, \cos^4\varphi \ge 0 \; \Rightarrow \; \sin\varphi = \dfrac{r}{\cos^4\varphi} \ge 0 \; \Rightarrow \; \varphi \in \left[ 0; \pi \right] \\
        & \text{Площадь фигуры: } \\
        & S = \dfrac12 \int\limits_{0}^{\pi} r^2(\varphi)\, d\varphi = 
        \dfrac12 \int\limits_{0}^{\pi} \sin^2\varphi \cos^8\varphi\, d\varphi = 
        \dfrac12 \int\limits_{0}^{\pi} (1 - \cos^2\varphi) \cos^8\varphi\, d\varphi = \\
        & \dfrac12 \left( \int\limits_{0}^{\pi} \cos^8\varphi\, d\varphi - \int\limits_{0}^{\pi} \cos^{10}\varphi\, d\varphi \right) =
        \dfrac12 \left( \int\limits_{0}^{\pi} \cos^8\varphi\, d\varphi - \dfrac{\cos^9\varphi \sin\varphi}{10} \Bigm|_{0}^{\pi} - \dfrac9{10}\int\limits_{0}^{\pi} \cos^8\varphi\, d\varphi \right) = \\
        & \dfrac1{20} \int\limits_{0}^{\pi} \cos^8\varphi\, d\varphi = 
        \dfrac1{20} \left( \dfrac{\cos^7\varphi \sin\varphi}8 \Bigm|_{0}^{\pi} + \dfrac78 \int\limits_{0}^{\pi} \cos^6\varphi\, d\varphi \right) = 
        \dfrac7{160} \left( \dfrac{\cos^5\varphi \sin\varphi}6 \Bigm|_{0}^{\pi} + \dfrac56 \int\limits_{0}^{\pi} \cos^4\varphi\, d\varphi \right) = \\
        & \dfrac7{192} \left( \dfrac{\cos^3\varphi \sin\varphi}4 \Bigm|_{0}^{\pi} + \dfrac34 \int\limits_{0}^{\pi} \cos^2\varphi\, d\varphi \right) = 
        \dfrac7{256} \left( \dfrac{\cos\varphi \sin\varphi}2 \Bigm|_{0}^{\pi} + \dfrac12 \int\limits_{0}^{\pi} \varphi\, d\varphi \right) = \dfrac{7\pi}{512} \\
    \end{align*}
    
    % \subsection*{Задача 6}
    
    Найти объем тела, заданного параметрически:
    % \subsection*{Задача 7}
    
    % \subsection*{Задача 8}
    
    \subsection*{Задача 9}
    \begin{align*}
        & x^2 + y^2 \le 1, \; 0 \le z \le 1 - 2y^2\; \Rightarrow \\
        & \Rightarrow \; x \in [-1; 1], \, y \in [-\sqrt{1-x^2}; \sqrt{1-x^2}], \, z \in [0; 1 - 2y^2] \\
        & z \ge 0 \Rightarrow 1- 2y^2 \ge 0 \Rightarrow y \in \left[ -\frac1{\sqrt2}; \frac1{\sqrt2} \right] \\
        & \sqrt{1-x^2} \ge \dfrac1{\sqrt2} \; \Leftrightarrow \; 1-x^2 \ge \dfrac12 \; \Leftrightarrow \; x^2 \le \dfrac12 \; \Leftrightarrow \; x \in \left[ -\frac1{\sqrt2}; \frac1{\sqrt2} \right] \\
        & \text{Объем фигуры: } \\
        & V = \iiint\limits_{D} dxdydz = \int\limits_{-1}^{-\frac1{\sqrt2}} dx \int\limits_{-\sqrt{1-x^2}}^{\sqrt{1-x^2}} dy \int\limits_{0}^{1 - 2y^2} dz + \int\limits_{\frac1{\sqrt2}}^{1} dx \int\limits_{-\sqrt{1-x^2}}^{\sqrt{1-x^2}} dy \int\limits_{0}^{1 - 2y^2} dz + \int\limits_{-\frac1{\sqrt2}}^{\frac1{\sqrt2}} dx \int\limits_{-\frac1{\sqrt2}}^{\frac1{\sqrt2}} dy \int\limits_{0}^{1 - 2y^2} dz = \\
        & = \int\limits_{-1}^{-\frac1{\sqrt2}} dx \int\limits_{-\sqrt{1-x^2}}^{\sqrt{1-x^2}} (1 - 2y^2)\, dy + \int\limits_{\frac1{\sqrt2}}^{1} dx \int\limits_{-\sqrt{1-x^2}}^{\sqrt{1-x^2}} (1 - 2y^2)\, dy  + \int\limits_{-\frac1{\sqrt2}}^{\frac1{\sqrt2}} dx \int\limits_{-\frac1{\sqrt2}}^{\frac1{\sqrt2}} (1 - 2y^2)\, dy = \\
        & = \int\limits_{-1}^{-\frac1{\sqrt2}} dx \cdot \left( y - \dfrac23 y^3 \right) \Bigm|_{-\sqrt{1-x^2}}^{\sqrt{1-x^2}} + \int\limits_{\frac1{\sqrt2}}^{1} dx \cdot \left( y - \dfrac23 y^3 \right) \Bigm|_{-\sqrt{1-x^2}}^{\sqrt{1-x^2}} + \int\limits_{-\frac1{\sqrt2}}^{\frac1{\sqrt2}} dx \cdot \left( y - \dfrac23 y^3 \right) \Bigm|_{-\frac1{\sqrt2}}^{\frac1{\sqrt2}} = \\
        & = \dfrac43 \int\limits_{\frac1{\sqrt2}}^{1} \sqrt{1-x^2}(1+2x^2)\, dx - \dfrac83 \int\limits_{\frac1{\sqrt2}}^{1} (1-x^2)^{3/2}\, dx + \left( \sqrt2 - \dfrac{\sqrt2}3 \right)x \Bigm|_{-\frac1{\sqrt2}}^{\frac1{\sqrt2}} \\
        & \text{Замена: } t = \arcsin x, \, t \in \left[-\frac{\pi}2; \frac{\pi}2\right], \, \frac{dt}{dx} = \dfrac1{\sqrt{1-x^2}} \\
        & \dfrac43 \int\limits_{\frac{\pi}4}^{\frac{\pi}2} (1+2\sin^2t)\, dt - \dfrac83 \int\limits_{\frac{\pi}4}^{\frac{\pi}2} (1-\sin^2t)\, dt + \left( 2 - \dfrac{2}3 \right) = 
        \dfrac43 \int\limits_{\frac{\pi}4}^{\frac{\pi}2} (4\sin^2t - 1)\, dt + \dfrac43 = \\
        & = \dfrac43 (-2\cos t \sin t + t) \Bigm|_{\frac{\pi}4}^{\frac{\pi}2} + \dfrac43 = 
        \dfrac43 (-\sin 2t + t) \Bigm|_{\frac{\pi}4}^{\frac{\pi}2} + \dfrac43 = \dfrac43 (1 + \frac{\pi}4) + \dfrac43 = \frac83 + \frac{\pi}3
    \end{align*}
    % \subsection*{Задача 10}
    
    % \subsection*{Задача 11}
    
    % \subsection*{Задача 12}
    
    % \subsection*{Задача 13}
    
    % \subsection*{Задача 14}
    
\end{document}
