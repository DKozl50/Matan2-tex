\documentclass[a4paper, fleqn]{article}
\usepackage{header}

\title{Семинарский лист 2.6}
\author{
    % Александр Богданов   \\ \href{https://t.me/SphericalPotatoInVacuum}{Telegram} \and
    % Алиса Вернигор       \\ \href{https://t.me/allisyonok}{Telegram} \and
    % Анастасия Григорьева \\ \href{https://t.me/weifoll}{Telegram} \and
    % Василий Шныпко       \\ \href{https://t.me/yourvash}{Telegram} \and
    % Данил Казанцев       \\ \href{https://t.me/vserosbuybuy}{Telegram} \and
    % Денис Козлов         \\ \href{https://t.me/DKozl50}{Telegram} \and
    Елизавета Орешонок   \\ \href{https://t.me/eaoresh}{Telegram} \and
    % Ира Голобородько     \\ \href{https://t.me/Ira4kgl}{Telegram} \and
    % Иван Пешехонов       \\ \href{https://t.me/JohanDDC}{Telegram} \and
    % Иван Добросовестнов  \\ \href{https://t.me/ivankot13}{Telegram} \and
    % Настя Городилова     \\ \href{https://t.me/nastygorodi}{Telegram} \and
    % Никита Насонков      \\ \href{https://t.me/nnv_nick}{Telegram} \and
    % Сергей Лоптев        \\ \href{https://t.me/beast_sl}{Telegram}
}

\date{Версия от {\ddmmyyyydate\today} \currenttime}

\begin{document}
    \maketitle
    

    Перейдя к полярным или обобщенным полярным координатам, вычислить площадь фигуры, ограниченной кривой.
    \subsection*{Задача 1}
    \begin{align*}
        & (x^2 + y^2)^2 = 2x^3 \\
        & \text{В полярных координатах: } x = r \cos \varphi, \, y = r \sin \varphi, |J| = r \\
        & (x^2 + y^2)^2 = 2x^3 \; \Leftrightarrow \; r^4 = 2r^3 \cos^3 \varphi \; 
        \Leftrightarrow \; r = 2\cos^3\varphi \; \Leftrightarrow \; \cos \varphi = \sqrt[3]{\dfrac{r}2} \\
        & r \ge 0 \; \Rightarrow \; \cos \varphi \ge 0 \; \Rightarrow \; \varphi \in \left[ -\frac{\pi}2; \frac{\pi}2 \right] \\
        & \text{Площадь фигуры: } \\
        & S = \dfrac12 \int\limits_{-\frac{\pi}2}^{\frac{\pi}2} r^2(\varphi)\, d\varphi = 
        \dfrac12 \int\limits_{-\frac{\pi}2}^{\frac{\pi}2} 4\cos^6\varphi\, d\varphi = 
        2 \int\limits_{-\frac{\pi}2}^{\frac{\pi}2} \cos^6\varphi\, d\varphi =
        \dfrac{\cos^5\varphi \sin\varphi}3 \Bigm|_{-\frac{\pi}2}^{\frac{\pi}2} + \dfrac53 \int\limits_{-\frac{\pi}2}^{\frac{\pi}2} \cos^4\varphi\, d\varphi = \\
        & = \dfrac{5\cos^3\varphi \sin\varphi}{12} \Bigm|_{-\frac{\pi}2}^{\frac{\pi}2} + \dfrac54 \int\limits_{-\frac{\pi}2}^{\frac{\pi}2} \cos^2\varphi\, d\varphi =
        \dfrac{5\cos\varphi \sin\varphi}{8} \Bigm|_{-\frac{\pi}2}^{\frac{\pi}2} + \dfrac58 \int\limits_{-\frac{\pi}2}^{\frac{\pi}2} d\varphi = \dfrac{5\pi}8
    \end{align*}
    
    \subsection*{Задача 2}

    $\underline{(x^2 + y^2)^3 = x^4 + y^4}$

    $\bullet \; $ Используем пол. координаты: $\begin{cases}
    x = r \cos \varphi;\\
    y = r \sin \varphi \\
    \end{cases}$


    $\bullet \; $ Кривая становится $r^6 = r^4 \cos ^4 \varphi  + r^4 \sin ^4 \varphi \iff r^2 = \cos^4 \varphi + \sin^4 \varphi = (\cos^2 \varphi + \sin^2 \varphi)^2 - 2 \cos^2 \varphi \, \sin^2 \varphi = \\ 1 - \frac{\sin^2 2 \varphi}{2} \implies r = \sqrt{1 - \frac{\sin^2 2 \varphi}{2}}.$

    $\bullet \; $ $\displaystyle S = \int\limits_{0}^{2 \pi} d \varphi \int\limits_{0}^{\sqrt{1 - \sin^2 2 \varphi/2}} \underbrace{r}_{\text{Якобиан}} \; dr = \int\limits_{0}^{2 \pi} \frac{1}{2} \cdot \left( 1 - \frac{\sin^2 2 \varphi}{2} \right) \; d \varphi = \frac{1}{2} \int \limits_{0}^{2 \pi} 1 -  \frac{(1/2 - 1/2 \cos 4 \varphi)}{2} = \frac{1}{4} \int \limits_{0}^{2 \pi} \frac{3}{2} + \frac{\cos 4 \varphi}{2} \; d \varphi  = \frac{3 \pi}{4} + \frac{1}{16} \int\limits_{0}^{8 \pi} \cos \Theta \; d \Theta =\boxed{ \frac{3 \pi}{4}} \; .$

    
    \subsection*{Задача 3}
    \begin{align*}
        & (x^2 + y^2)^2 = xy \\
        & \text{В полярных координатах: } x = r \cos \varphi, \, y = r \sin \varphi \\
        & (x^2 + y^2)^2 = xy \; \Leftrightarrow \; r^4 = r^2 \sin\varphi \cos\varphi \; 
        \Leftrightarrow \; r^2 = \dfrac12 \sin 2\varphi \; \Leftrightarrow \; \sin 2\varphi = 2r^2 \\
        & r^2 \ge 0 \; \Rightarrow \; \sin 2\varphi \ge 0 \; \Rightarrow \; \varphi \in \left[ 0; \frac{\pi}2 \right] \cup \left[ \pi; \frac{3\pi}2 \right]\\
        & \text{Площадь фигуры: } \\
        & S = \dfrac12 \left( \int\limits_{0}^{\frac{\pi}2} r^2(\varphi)\, d\varphi + \int\limits_{\pi}^{\frac{3\pi}2} r^2(\varphi)\, d\varphi \right) = 
         \dfrac12 \left( \int\limits_{0}^{\frac{\pi}2} \sin 2\varphi\, d\varphi + \int\limits_{\pi}^{\frac{3\pi}2} \sin 2\varphi\, d\varphi \right) = \\
        & = -\dfrac14 \left( \cos 2\varphi \Bigm|_{0}^{\frac{\pi}2} + \cos 2\varphi \Bigm|_{\pi}^{\frac{3\pi}2} \right) = 
        -\dfrac14 \left( -1 - 1 - 1 - 1 \right) = 1
    \end{align*}
    
    % \subsection*{Задача 4}
    
    \subsection*{Задача 5}
    \begin{align*}
        & (x^2 + y^2)^3 = x^4y \\
        & \text{В полярных координатах: } x = r \cos \varphi, \, y = r \sin \varphi \\
        & (x^2 + y^2)^3 = x^4y \; \Leftrightarrow \; r^6 = r^5 \sin\varphi \cos^4\varphi \; 
        \Leftrightarrow \; r = \sin\varphi \cos^4\varphi \\
        & r \ge 0, \cos^4\varphi \ge 0 \; \Rightarrow \; \sin\varphi = \dfrac{r}{\cos^4\varphi} \ge 0 \; \Rightarrow \; \varphi \in \left[ 0; \pi \right] \\
        & \text{Площадь фигуры: } \\
        & S = \dfrac12 \int\limits_{0}^{\pi} r^2(\varphi)\, d\varphi = 
        \dfrac12 \int\limits_{0}^{\pi} \sin^2\varphi \cos^8\varphi\, d\varphi = 
        \dfrac12 \int\limits_{0}^{\pi} (1 - \cos^2\varphi) \cos^8\varphi\, d\varphi = \\
        & \dfrac12 \left( \int\limits_{0}^{\pi} \cos^8\varphi\, d\varphi - \int\limits_{0}^{\pi} \cos^{10}\varphi\, d\varphi \right) =
        \dfrac12 \left( \int\limits_{0}^{\pi} \cos^8\varphi\, d\varphi - \dfrac{\cos^9\varphi \sin\varphi}{10} \Bigm|_{0}^{\pi} - \dfrac9{10}\int\limits_{0}^{\pi} \cos^8\varphi\, d\varphi \right) = \\
        & \dfrac1{20} \int\limits_{0}^{\pi} \cos^8\varphi\, d\varphi = 
        \dfrac1{20} \left( \dfrac{\cos^7\varphi \sin\varphi}8 \Bigm|_{0}^{\pi} + \dfrac78 \int\limits_{0}^{\pi} \cos^6\varphi\, d\varphi \right) = 
        \dfrac7{160} \left( \dfrac{\cos^5\varphi \sin\varphi}6 \Bigm|_{0}^{\pi} + \dfrac56 \int\limits_{0}^{\pi} \cos^4\varphi\, d\varphi \right) = \\
        & \dfrac7{192} \left( \dfrac{\cos^3\varphi \sin\varphi}4 \Bigm|_{0}^{\pi} + \dfrac34 \int\limits_{0}^{\pi} \cos^2\varphi\, d\varphi \right) = 
        \dfrac7{256} \left( \dfrac{\cos\varphi \sin\varphi}2 \Bigm|_{0}^{\pi} + \dfrac12 \int\limits_{0}^{\pi} \varphi\, d\varphi \right) = \dfrac{7\pi}{512} \\
    \end{align*}
    
    % \subsection*{Задача 6}
    
      \subsection*{Задача 7}
    
    $\underline{x^2 + y^2 \leq 1, \; z \geq 0, \; x + y + z \leq 4}$
    
    $\bullet \; \; x^2 + y^2 \leq 1 \implies x \in \left[-\sqrt{1 - y^2}, \; \sqrt{1 - y^2}\right]; \; \; y \in [-1, \; 1].$\\
    
    $\bullet \; \; x + y + z \leq 4 \implies z \leq 4 - x - y. \; $ Условие $x + y \leq 4$ при данных ограничениях выполнено всегда.
    
    $S = \int\limits_{-1}^{1}  dy \int\limits_{-\sqrt{1 - y^2}}^{\sqrt{1 - y^2}} dx \int\limits_{0}^{4 -x - y} dz =  \int\limits_{-1}^{1}  dy \int\limits_{-\sqrt{1 - y^2}}^{\sqrt{1 - y^2}} 4 -x - y \; \;  dx =  \int\limits_{-1}^{1}  dy \int\limits_{-\sqrt{1 - y^2}}^{\sqrt{1 - y^2}} 4 - y \; \; dx -  \underbrace{\int\limits_{-1}^{1}  dy \int\limits_{-\sqrt{1 - y^2}}^{\sqrt{1 - y^2}} x \, \; dx}_{0} = \\
    2 \int\limits_{-1}^{1}  \sqrt{1 - y^2} \cdot (4 - y) \;  dy = 
    8 \int\limits_{-1}^{1}  \sqrt{1 - y^2}  \;  dy - 2 \int\limits_{-1}^{1}  \sqrt{1 - y^2} \cdot  y \;  dy .$
    
    Замена $\begin{cases}
    t = \arcsin y;\\
    t \in \left[-\pi/2 , \; \pi/2\right];\\
    dt = \frac{dy}{\sqrt{1 - t^2}}
    \end{cases}$
    
    $\dots \displaystyle 8 \int\limits_{-\pi /2}^{\pi/2} dt - \underbrace{2 \int\limits_{-\pi /2}^{\pi/2} \sin t \; dt}_{0} = \boxed{8 \pi} \; . $
    
    
    
    \subsection*{Задача 8}
    
    $\underline{x^2 + y^2 \leq 1, \; z \geq 0, \; x + y + z \leq 1}$
    
    $\bullet \; \; x^2 + y^2 \leq 1 \implies x \in \left[-\sqrt{1 - y^2}, \; \sqrt{1 - y^2}\right]; \; \; y \in [-1, \; 1].$\\
    
    $\bullet \; \; x + y + z \leq 1 \implies z \leq 1 - x - y. \; $ 
    
    $\bullet \; $ Также необходимо условие $x + y \leq 1$.
    
    При $y \in [-1, \, 0]$ оно соблюдается, в ином случае $x$ сверху придётся ограничить значением $1 - y$.
    
    \textit{Советую сделать сначала предыдущий номер, там подробно описана замена, которую я использую.}
    
    $\bullet \; \, S = \displaystyle \int\limits_{-1}^{0} dy \int\limits_{-\sqrt{1- y^2}}^{\sqrt{1- y^2}} dx \int\limits_{0}^{1 - x - y} dz + \displaystyle \int\limits_{0}^{1} dy \int\limits_{-\sqrt{1- y^2}}^{1 - y} dx \int\limits_{0}^{1 - x - y} dz =
    2 \int\limits_{-1}^{0} (1- y) \sqrt{1- y^2} \; dy -
    \underbrace{\int\limits_{-1}^{0} dy \int\limits_{-\sqrt{1- y^2}}^{\sqrt{1- y^2}} x\;  dx}_{0} + \\ 
    \displaystyle \int\limits_{0}^{1} (1 - y) \left((1 - y) + \sqrt{1 - y^2} \right) \; dy  - \int\limits_{0}^{1} dy \int\limits_{-\sqrt{1- y^2}}^{1 - y} x\;  dx =
    2 \int\limits_{-1}^{0}  \sqrt{1- y^2} \; dy  - 2 \int\limits_{-1}^{0} y \sqrt{1- y^2} \; dy + \\
    +
    \displaystyle \int\limits_{0}^{1}  \left(1 - 2y + y^2 \right) \; dy  + \int\limits_{0}^{1} (1 - y) \sqrt{1 - y^2}  \; dy  - \frac{1}{2} \int\limits_{0}^{1} 1 -2y + y^2 - 1 + y^2 \; \, dy  =
    2 \int\limits_{-\pi/2}^{0}  dt  - 2 \int\limits_{-\pi/2}^{0} \sin t  \; dt + \\
    +
    \displaystyle \left(t - t^2 + \frac{t^3}{3} \; \; \Bigg|_{0}^{1}   \right) \;  + \int\limits_{0}^{1} \sqrt{1 - y^2}  \; dy   - \int\limits_{0}^{1} y \, \sqrt{1 - y^2}  \; dy - \frac{1}{2} \int\limits_{0}^{1}  -2y + 2y^2 \; \, dy = \pi + -2\left(  - \cos t \;\; \Big|_{-\pi/2}^{0 } \right) + \frac{1}{3} + \\ \int\limits_{0}^{\pi/2} dt - \int\limits_{0}^{\pi/2} \sin t \; dt + \int\limits_{0}^{1} y - y^2 \; \, dy = \pi + 2 + \frac{1}{3} + \frac{\pi}{2} + \left(\cos t \; \; \Bigg|_{0}^{\pi/2} \right) + \left(\frac{y^2}{2} - \frac{y^3}{3} \; \; \Bigg|_{0}^{1} \right) = \frac{3 \pi}{2} + 2 + \frac{1}{3} - 1 + \frac{1}{2} - \frac{1}{3} = \boxed{\frac{3 \pi}{2} +\frac{3}{2}} \, .$
    
    % \subsection*{Задача 9}
    
    \subsection*{Задача 10}
    
    $\underline{x \leq x^2 + y^2 \leq 2x, \; 0 \leq z \leq x^2 y^2}$
    
    $\bullet \; $ Циллиндрические координаты: $\begin{cases}
    x = r \cos \varphi;\\
    y = r \sin \varphi; \\
    z = z
    \end{cases}$
    
    $\bullet \; \, x \leq x^2 + y^2 \leq 2x \implies r \cos \varphi \leq r^2 \leq 2 r \cos \varphi \implies \cos \varphi \leq r \leq 2 \cos \varphi \implies \cos \varphi \geq 0 \implies \varphi \in \left[-\pi/2 ,\, \pi/2\right].$
    
    $\bullet \; \,$  $0 \leq z \leq x^2 y^2 = r^4 \cos^2 \varphi \, \sin^2 \varphi = \frac{r^4}{4} \sin^2 2 \varphi.$
    
    $\bullet \; \int\limits_{-\pi/2}^{\pi/2} d \varphi \int\limits_{\cos \varphi}^{2 \cos \varphi} dr \int\limits_{0}^{\frac{r^4}{4} \sin^2 2 \varphi} r \; dz = 
    \int\limits_{-\pi/2}^{\pi/2} d \varphi \int\limits_{\cos \varphi}^{2 \cos \varphi} \frac{r^5}{4} \sin^2 2 \varphi \; dr  =
    \int\limits_{-\pi/2}^{\pi/2}  \sin^2 2 \varphi \; d \varphi \int\limits_{\cos \varphi}^{2 \cos \varphi} \frac{r^5}{4} \; dr  = \frac{1}{4} \int\limits_{-\pi/2}^{\pi/2}  \sin^2 2 \varphi \; d \varphi \left(\frac{r^6}{6} \; \; \; \Bigg|_{\cos \varphi}^{2 \cos \varphi} \right) = \\
    =  \frac{63}{24} \int\limits_{-\pi/2}^{\pi/2}  \sin^2 2 \varphi \, \cos^6 \varphi \; \ \; d \varphi = 
      \frac{63}{24} \int\limits_{-\pi/2}^{\pi/2}  \left(\frac{1}{2} - \frac{1}{2} \cos 4 \varphi \right) \cdot \left(\frac{1}{2} + \frac{1}{2} \cos 2 \varphi \right)^3 \; \ \; d \varphi =
      \frac{63}{24} \cdot \frac{1}{16} \int\limits_{-\pi/2}^{\pi/2} (1 - \cos 4 \varphi) \cdot (1 + \cos 2 \varphi)^3 \; d \varphi = \\
      \frac{63}{24} \cdot \frac{1}{16} \int\limits_{-\pi/2}^{\pi/2} (1 - \cos 4 \varphi) \left(1 + 3 \cos 2 \varphi + 3 \left( \frac{1}{2} + \frac{1}{2} \cos 4 \varphi
      \right) + \cos 2 \varphi \cdot \left( \frac{1}{2} + \frac{1}{2} \cos 4 \varphi
      \right)  \right) \; d\varphi = \\
      \frac{63}{24 \cdot 16} \left( \frac{5}{2} \cdot  \pi  -\frac{3}{2} \cdot  \frac{1}{2} \cdot \pi \right) =
      \boxed{\frac{63}{24 \cdot 16} \cdot \frac{7}{4} \,  \pi} \; .$
      
      
      \textbf{Как мы это получили:} воспользуемся периодичностью и поймём, что после подстановки все слагаемые обратятся в 0, за исключением $1 + \frac{3}{2}$,  что даёт вклад $\frac{5 }{2}; \; $ и $ \frac{-3}{2} \cdot \cos^2 4 \varphi = \frac{-3}{2} \left( \frac{1}{2} + \frac{1}{2} \cos 8 \varphi \right),$ что даёт вклад $\frac{-3}{2} \cdot \frac{1}{2}. $ \\ 
      ($\cos 8 \varphi$ аналогично уходит по периодичности).
      
    
    % \subsection*{Задача 11}
    
    % \subsection*{Задача 12}
    
    % \subsection*{Задача 13}
    
    % \subsection*{Задача 14}
    
\end{document}
