\documentclass[a4paper, fleqn]{article}
\usepackage{header}

\title{Семинарский лист 2.4}
\author{
    % Александр Богданов   \\ \href{https://t.me/SphericalPotatoInVacuum}{Telegram} \and
    % Алиса Вернигор       \\ \href{https://t.me/allisyonok}{Telegram} \and
    % Анастасия Григорьева \\ \href{https://t.me/weifoll}{Telegram} \and
    % Василий Шныпко       \\ \href{https://t.me/yourvash}{Telegram} \and
    % Данил Казанцев       \\ \href{https://t.me/vserosbuybuy}{Telegram} \and
    % Денис Козлов         \\ \href{https://t.me/DKozl50}{Telegram} \and
    % Елизавета Орешонок   \\ \href{https://t.me/eaoresh}{Telegram} \and
    % Иван Пешехонов       \\ \href{https://t.me/JohanDDC}{Telegram} \and
    % Иван Добросовестнов  \\ \href{https://t.me/ivankot13}{Telegram} \and
    % Настя Городилова     \\ \href{https://t.me/nastygorodi}{Telegram} \and
    % Никита Насонков      \\ \href{https://t.me/nnv_nick}{Telegram} \and
    % Сергей Лоптев        \\ \href{https://t.me/beast_sl}{Telegram}
}

\date{Версия от {\ddmmyyyydate\today} \currenttime}

\begin{document}
    \maketitle
   
    \section*{Предполагая функцию $f$ непрерывной на $D$, запишите тройной интеграл от $f$ по $D$ в виде
    одного из повторных, если $D$ задано неравенствами.}
    
    % \subsection*{Задача 1}
    
    \subsection*{Задача 2}
    
    % \subsection*{Задача 3}
    
    % \subsection*{Задача 4}
    
    % \subsection*{Задача 5}
    
    % \subsection*{Задача 6}
    
    \section*{Вычислите интеграл.}
    % \subsection*{Задача 7}
    
    % \subsection*{Задача 8}
    
    % \subsection*{Задача 9}
    
    % \subsection*{Задача 10}
    
    \section*{Измените порядок интегрирования и вычислите интеграл.}
    % \subsection*{Задача 11}
    
    % \subsection*{Задача 12}
    
    % \subsection*{Задача 13}
    
    % \subsection*{Задача 14}
    
    % \subsection*{Задача 15}
    
    % \subsection*{Задача 16}
    
    % \subsection*{Задача 17}
    
    \section*{Вычислите многократный интеграл.}
    % \subsection*{Задача 18}
    
    % \subsection*{Задача 19}
    
    % \subsection*{Задача 20}
    
    % \subsection*{Задача 21}
    
    % \subsection*{Задача 22}
    
    % \subsection*{Задача 23}
    
\end{document}
