\documentclass[a4paper, fleqn]{article}
\usepackage{header}

\title{Семинарский лист 3.4}
\author{
    % Александр Богданов   \\ \href{https://t.me/SphericalPotatoInVacuum}{Telegram} \and
    % Алиса Вернигор       \\ \href{https://t.me/allisyonok}{Telegram} \and
    Денис Козлов         \\ \href{https://t.me/DKozl50}{Telegram} \and
    % Елизавета Орешонок   \\ \href{https://t.me/eaoresh}{Telegram} \and
    % Ира Голобородько     \\ \href{https://t.me/Ira4kgl}{Telegram} \and
}

\date{Версия от {\ddmmyyyydate\today} \currenttime}

\begin{document}
\maketitle

% начинаем с 7-12

\section*{Обоснуйте возможность занесения предела под знак интеграла и вычислите предел.}
% \subsection*{Задача 1}

% \subsection*{Задача 2}

% \subsection*{Задача 3}

% \subsection*{Задача 4}

% \subsection*{Задача 5}

% \subsection*{Задача 6}

\section*{Найдите область определения функции, заданной интегралом, 
    и исследуйте эту функцию на непрерывность}
\subsection*{Задача 7}
\begin{flalign*}
    & \int_0^{\infty} \frac{\cos px}{\sqrt{1 + x^3}} dx = I(p) \quad 
    \text{в нуле особенности нет, а на бесконечности есть по определению} \\
    & \left| \frac{\cos px}{\sqrt{1 + x^3}} \right| \leq \left| \frac{1}{\sqrt{x^3}} \right| \quad
    \Rightarrow \quad \int_1^{\infty} \frac{dx}{\sqrt{x^3}} 
    \text{ сходится (от 1 потому что интересно только на беск)} \\
    & \text{По признаку сравнения сходится } \int_0^{\infty} \frac{\cos px}{\sqrt{1 + x^3}} dx 
    \forall p \\
    & \begin{array} {l}
        \text{Признак сравнения порождает Вейрштрасса, поэтому } I(p) \text{ сх. равн} \\
        f(x, p) = \frac{\cos px}{\sqrt{1 + x^3}}, x > 0 \text{ непр} 
    \end{array} 
    \Bigg\} I(p) \text{ непр на } \RR
\end{flalign*}

% \subsection*{Задача 8}

% \subsection*{Задача 9}

% \subsection*{Задача 10}

% \subsection*{Задача 11}

% \subsection*{Задача 12}


\end{document}
