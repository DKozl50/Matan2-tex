\documentclass[a4paper, fleqn]{article}
\usepackage{header}

\title{Семинарский лист 3.4}
\author{
    % Александр Богданов   \\ \href{https://t.me/SphericalPotatoInVacuum}{Telegram} \and
    % Алиса Вернигор       \\ \href{https://t.me/allisyonok}{Telegram} \and
    Денис Козлов         \\ \href{https://t.me/DKozl50}{Telegram} \and
    Елизавета Орешонок   \\ \href{https://t.me/eaoresh}{Telegram} \and
    % Ира Голобородько     \\ \href{https://t.me/Ira4kgl}{Telegram} \and
}

\date{Версия от {\ddmmyyyydate\today} \currenttime}

\begin{document}
\maketitle

% начинаем с 7-12

\section*{Обоснуйте возможность занесения предела под знак интеграла и вычислите предел.}
% \subsection*{Задача 1}

% \subsection*{Задача 2}

% \subsection*{Задача 3}

% \subsection*{Задача 4}

% \subsection*{Задача 5}

% \subsection*{Задача 6}

\section*{Найдите область определения функции, заданной интегралом, 
    и исследуйте эту функцию на непрерывность}
\subsection*{Задача 7}
\begin{flalign*}
    & \int_0^{\infty} \frac{\cos px}{\sqrt{1 + x^3}} dx = I(p) \quad 
    \text{в нуле особенности нет, а на бесконечности есть по определению} \\
    & \left| \frac{\cos px}{\sqrt{1 + x^3}} \right| \leq \left| \frac{1}{\sqrt{x^3}} \right| \quad
    \Rightarrow \quad \int_1^{\infty} \frac{dx}{\sqrt{x^3}} 
    \text{ сходится (от 1 потому что интересно только на беск)} \\
    & \text{По признаку сравнения сходится } \int_0^{\infty} \frac{\cos px}{\sqrt{1 + x^3}} dx 
    \forall p \\
    & \begin{array} {l}
        \text{Признак сравнения порождает Вейрштрасса, поэтому } I(p) \text{ сх. равн} \\
        f(x, p) = \frac{\cos px}{\sqrt{1 + x^3}}, x > 0 \text{ непр} 
    \end{array} 
    \Bigg\} I(p) \text{ непр на } \RR
\end{flalign*}

\subsection*{Задача 8}
    \begin{flalign*}
        & \int\limits_0^{+\infty} e^{-(x + p)^2}\,dx = \int\limits_0^{+\infty} \frac1{e^{(x + p)^2}}\,dx \\
        & (x + p)^2 \text{ --- парабола с ветвями вверх, наим. значение $0$ достигается при } x = -p \\
        & \text{При } p \ge 0 \; \left| e^{-(x + p)^2} \right| \le \frac1{e^{x^2}} < \frac1{x^2},
        \int\limits_0^{+\infty} \frac1{x^2}\,dx \text{ сходится } \Rightarrow \\
        & \Rightarrow\; \int\limits_0^{+\infty} e^{-(x + p)^2}\,dx \text{ сходится равномерно по признаку Вейерштрасса} \\
        & \frac1{\sqrt{1 + x^3}} \le \frac1{\sqrt{x^3}} = \frac1{x^{3/2}}, \; 3/2 > 1
        \Rightarrow \frac1{x^{3/2}} \text{ сходится} \Rightarrow 
        \frac{\cos\,px}{\sqrt{1 + x^3}} \text{ при $p \ge 0$ сходится равномерно по пр-ку Вейерштрасса} \\
        & \text{При $p < 0$ сделаем замену $u = x - p$ и получим интеграл} \\
        & \int\limits_{-p}^{+\infty} e^{-u^2}\,du = 
        \int\limits_{-p}^{0} e^{-u^2}\,du + \int\limits_{0}^{+\infty} e^{-u^2}\,du = 
        \int\limits_{0}^{p} e^{-u^2}\,du + \int\limits_{0}^{+\infty} e^{-u^2}\,du, \\
        & \text{также сходящийся по признаку Вейерштрасса (док-во аналогично случаю выше)} \\
        & \text{При этом } e^{-(x + p)^2} 
        \text{ непрерывна на } [0; +\infty) \times (-\infty; +\infty) \;\Rightarrow \\
        & \Rightarrow \int\limits_0^{+\infty} e^{-(x + p)^2}\,dx \text{ непрерывна при } p \in \RR
    \end{flalign*}

\subsection*{Задача 9}
\begin{flalign*}
    & I(p) = \int_0^{\pi} \frac{dx}{\sin^p x} = 
    2 \int_0^{\pi/2} \frac{dx}{\sin^p x} \;
    \left\{ \! \text{ так только одна особенность в } 0 \right\} \simeq
    2 \int_0^{\pi/2} \frac{dx}{x^p} \text{ сх. при } p < 1 \\
    & \text{Крайняя точка } p = 1 \quad I(1) = \int_0^{\pi} \frac{dx}{\sin x} \simeq 
    2 \int_0^{\pi/2} \frac{dx}{x} \text{ расх } \Rightarrow 
    \text{ по  методу крайней точки } I(p < 1) \text{ может не сходиться} \\
    & \text{Придется подтвердить более потным способом: } \\
    & \left| \frac{1}{\sin^p x} \right| = 
    \left| \frac{1}{\sin x} \right|^p \leq 
    \left| \frac{1}{\sin x} \right|^{\beta} \leq 
    \left| \frac{1}{\sin^{\beta} x} \right| \quad 
    \beta < 1 \Rightarrow \int_0^{\pi} \frac{dx}{\sin^{\beta} x} \text{ сх } \Rightarrow 
    \int_0^{\pi} \frac{dx}{\sin^p x} \text{ сх равн по Вейрштрассу на } [\alpha, \beta] \\
    & \forall [\alpha, \beta] \in (-\infty, 1): I(p) \text{ сх равн. на } [\alpha, \beta] \\
    & \forall p \in (-\infty, 1) \exists [\alpha, \beta] \in (-\infty, 1), p \in [\alpha, \beta] 
    \Rightarrow I(p) \text{ непр в } p \Rightarrow I(p) \text{ непр на } (-\infty, 1) \\
\end{flalign*}

% \subsection*{Задача 10}

% \subsection*{Задача 11}

% \subsection*{Задача 12}


\end{document}
