\documentclass[a4paper, fleqn]{article}
\usepackage{header}

\newcommand{\ds}{\displaystyle}

\title{Семинарский лист 3.3}
\author{
    % Александр Богданов   \\ \href{https://t.me/SphericalPotatoInVacuum}{Telegram} \and
    % Алиса Вернигор       \\ \href{https://t.me/allisyonok}{Telegram} \and
    % Денис Козлов         \\ \href{https://t.me/DKozl50}{Telegram} \and
    % Елизавета Орешонок   \\ \href{https://t.me/eaoresh}{Telegram} \and
    % Ира Голобородько     \\ \href{https://t.me/Ira4kgl}{Telegram} \and
}

\date{Версия от {\ddmmyyyydate\today} \currenttime}

\begin{document}
\maketitle

% Оптимизации ради предлагаю сначала сделать 11-18
% А потом уже заняться остальными

% \subsection*{Задача 1}

% \subsection*{Задача 2}

% \subsection*{Задача 3}

% \subsection*{Задача 4}

% \subsection*{Задача 5}

% \subsection*{Задача 6}

% \subsection*{Задача 7}

% \subsection*{Задача 8}

% \subsection*{Задача 9}

% \subsection*{Задача 10}

\section*{Покажите, используя признак Вейрштрасса, что интеграл сходится равномерно.}
% \subsection*{Задача 11}

% \subsection*{Задача 12}

\subsection*{Задача 13}
$\displaystyle\int\limits_1^{\infty} \frac{\ln^p x}{x^3+1} dx, \quad p \in (0; p_0)$

Заметим что $\displaystyle\ln^px \leq \ln^{p_0}x$.

$\displaystyle\exists C: \frac{\ln^{p_0} x}{x^3+1} \leq  \frac{Cx}{x^3+1} <  \frac{C}{x^2}, \quad C$ - не зависит от p 

$\displaystyle\int\limits_1^\infty \frac{C}{x^2} dx$ -интегрируема и равномерно сходится по p(она вообще от него не зависит) $\implies$ наш интеграл сходится равн. (по пр. Вейерштрасса)
% \subsection*{Задача 14}

\section*{Покажите, используя признак Дирихле или Абеля, что интеграл сходится равномерно.}
% \subsection*{Задача 15}

% \subsection*{Задача 16}

% \subsection*{Задача 17}

% \subsection*{Задача 18}

% \subsection*{Задача 19}

% \subsection*{Задача 20}

% \subsection*{Задача 21}

% \subsection*{Задача 22}

% \subsection*{Задача 23}

% \subsection*{Задача 24}

% \subsection*{Задача 25}

% \subsection*{Задача 26}

% \subsection*{Задача 27}

% \subsection*{Задача 28}

% \subsection*{Задача 29}


\end{document}
