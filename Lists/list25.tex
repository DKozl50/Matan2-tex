\documentclass[a4paper, fleqn]{article}
\usepackage{header}

\title{Семинарский лист 2.5}
\author{
    % Александр Богданов   \\ \href{https://t.me/SphericalPotatoInVacuum}{Telegram} \and
    % Алиса Вернигор       \\ \href{https://t.me/allisyonok}{Telegram} \and
    % Анастасия Григорьева \\ \href{https://t.me/weifoll}{Telegram} \and
    % Василий Шныпко       \\ \href{https://t.me/yourvash}{Telegram} \and
    % Данил Казанцев       \\ \href{https://t.me/vserosbuybuy}{Telegram} \and
    % Денис Козлов         \\ \href{https://t.me/DKozl50}{Telegram} \and
    % Елизавета Орешонок   \\ \href{https://t.me/eaoresh}{Telegram} \and
    % Ира Голобородько     \\ \href{https://t.me/Ira4kgl}{Telegram} \and
    % Иван Пешехонов       \\ \href{https://t.me/JohanDDC}{Telegram} \and
    % Иван Добросовестнов  \\ \href{https://t.me/ivankot13}{Telegram} \and
    % Настя Городилова     \\ \href{https://t.me/nastygorodi}{Telegram} \and
    % Никита Насонков      \\ \href{https://t.me/nnv_nick}{Telegram} \and
    % Сергей Лоптев        \\ \href{https://t.me/beast_sl}{Telegram}
}

\date{Версия от {\ddmmyyyydate\today} \currenttime}

\begin{document}
    \maketitle
    
    \section*{Задайте в полярных координатах множество $D$, заданное неравенствами в декартовых координатах.
    Предполагая функцию $f$ непрерывной на $D$, преобразуйте интеграл в полярных координатах к повторному.}
    % \subsection*{Задача 1}
    
    % \subsection*{Задача 2}
    
    \subsection*{Задача 3}
    
    $(x^2 + y^2)^2 \leq 4xy.$ Перепишем неравенство в полярных координатах:
    
    $(r^2\cos^2\varphi + r^2\sin^2\varphi)^2 \leq 4 r^2 \cos \varphi \cdot \sin \varphi \iff r^2 \leq 2\sin 2\varphi. \; \; \circledast$
    
    Делаем вывод, что $2 \sin 2 \varphi \geq 0 \iff 2 \varphi \in [2n \pi, (2n + 1)\pi] \implies \varphi \in \left[0, \frac{\pi}{2}\right] \cup \left[\pi, \frac{3\pi}{2} \right].$
    
    $\circledast \to \; r \leq \sqrt{2 \sin 2 \varphi}.$ 
    
    Рисунок тут даже не требуется, можно сразу записать интеграл.
    
    $\displaystyle\int\limits_{0}^{\pi/2} d \varphi \int\limits_{0}^{\sqrt{2 \sin 2 \varphi}} f \cdot r \; d r + \int\limits_{\pi}^{3\pi/2} d \varphi \int\limits_{0}^{\sqrt{2 \sin 2 \varphi}} f \cdot r \; d r. $
    
    % \subsection*{Задача 4}
    
    \subsection*{Задача 5}
    
    $D$ -- множество, лежащее в круге $x^2 + y^2 \leq 1$ вне кривой $r = \cos 3 \varphi.$
    
    
    Для начала простое. $x^2 + y^2 \leq \iff r \leq 1 \implies r \in [0,1].$
    
    Теперь разберёмся с кривой $r  = \cos 3 \varphi$.
    
    Если $r \in [0,1], \; $ то $\cos 3 \varphi \in [0,1] \implies 3 \varphi \in \left[ -\frac{\pi}{2} + 2\pi n, \frac{\pi}{2} + 2\pi n \right].$
    
    \doublespacing $\begin{cases}
    n = 0 \; \to \; \varphi \in \left[ -\frac{\pi}{6},  \frac{\pi}{6} \right];\\
    n = 1\; \to \; \left[\frac{\pi}{2},  \frac{5\pi}{6} \right]; \\
    n = 2\; \to \;  \left[ \frac{7\pi}{6},  \frac{3\pi}{2} \right]. \\
    \end{cases}$
    
    Т.е. $\varphi \in  \left[ -\frac{\pi}{6},  \frac{\pi}{6} \right] \cup \left[\frac{\pi}{2},  \frac{5\pi}{6} \right]\cup \left[ \frac{7\pi}{6},  \frac{3\pi}{2} \right].$
    
    [picture is coming soon]
    
    \singlespacing Мы готовы записать интеграл, не забываем указывать якобиан!
    
    $\boxed{ \displaystyle \int\limits_{-\pi/6}^{\pi/6} d \varphi \int\limits_{\cos 3 \varphi}^{1} r \cdot f \; dr + \int\limits_{\pi/2}^{5\pi/6} d \varphi \int\limits_{\cos 3 \varphi}^{1} r \cdot f \; dr + 
    \int\limits_{7\pi/6}^{3\pi/2} d \varphi \int\limits_{\cos 3 \varphi}^{1} r \cdot f \; dr} \; $
    
    
    % \subsection*{Задача 6}
    
    \section*{Перейдя к полярным координатам, вычислите интеграл.}
    \subsection*{Задача 7}
    
    $\displaystyle \iint\limits_{D} \ln (1 + x^2 + y^2) dx dy, \; \, D = \{(x,y) \mid x^2 + y^2 \leq 4 \} .$
    
    $D$ -- круг с радиусом $2 \implies r \in [0,2]$.
    
    $\varphi \in [0, 2 \pi),$ как и всегда.
    
    $\displaystyle \int\limits_{0}^{2} dr \int\limits_{0}^{2 \pi} r \cdot \ln(1 + r^2) d \varphi = \int\limits_{0}^{2} 2 \pi \cdot r \cdot \ln(1 + r^2) dr =  \pi \int\limits_{0}^{2} 2  \cdot r \cdot \ln(1 + r^2) dr.$
    
    Замена $\begin{cases} t = 1 + r^2;\\
    dt = 2r \, dr;\\
    r \in [0, 2] \implies t \in [1,5] \end{cases}$
    
    $\dots = \pi \int\limits_{1}^{5} \ln (t) \, dt = \pi \cdot \left(  t \ln t - t \, \Bigg|_{1}^{5} \right) = \pi \cdot \left( 5 \ln 5 - 5 + 1\right) = \boxed{ \pi (5 \ln 5 - 4) } 
    \; .$
    
    % \subsection*{Задача 8}
    
    \section*{Перейдите к переменным, в которых область интегрирования имеет вид прямоугольника, 
    и вычислите интеграл.}
    
    \subsection*{Задача 9}
    
    $\displaystyle\iint\limits_{D} \frac{(x + y)^2}{x} dx dy; \; \, D = \{(x,y) \mid 1 - x \leq y \leq 3 - x, \; \; \frac{x}{2} \leq y \leq 2x \}.$
    
    Область интегрирования имеет вид прямоугольника -- значит, нам нужно, чтобы граница каждого интеграла изменялась от одной константы до другой константы.
    
\doublespacing    $\begin{cases} u := x + y \implies 1 \leq u \leq 3 \text{ из 1-го нер-ва.} \\
    v := \frac{y}{x} \implies \frac{1}{2} \leq v \leq 2 \text{ из 2-го нер-ва. }
    \end{cases}$
    
    \singlespacing Т.е. с помощью подобной замены мы смогли выполнить поставленную задачу (обе координаты ограничены только константами и не зависят друг от друга).  
    
    Выражаем теперь, обратно, $x$ и $y$ через $u$ и $v$.
    
    $x = u - y.$
    
    $\begin{cases}
    y = v \cdot x = v \cdot (u - y)  = v \cdot u - v \cdot y \implies y = \frac{v \cdot u}{1 + v}; \\
    x = u - \frac{v \cdot u}{1 + v} = \frac{u + uv - uv}{1 + v} = \frac{u}{1 + v} . \end{cases}$
    
    Считаем якобиан. Для удобства взятия производной, лучше игрек перепишем как $y = \frac{v \cdot u}{1 + v} = u - \frac{u}{1 + v}.$
    
     
    $J = \left| \begin{matrix} x_u & y_u \\ x_v & y_v \end{matrix} \right| = $ {\setstretch{2.4} $
     \left| \begin{matrix} \left(\frac{u}{1 + v}\right)_u & \left(u - \frac{u}{1 + v}\right)_u \\ \left(\frac{u}{1 + v}\right)_v & \left(u - \frac{u}{1 + v} \right)_v \end{matrix} \right| =  
     \left| \begin{matrix} \frac{1}{1 + v} & \frac{v}{1 + v} \\ \frac{-u}{(1 + v)^2} & \frac{u}{(1 + v)^2} \end{matrix} \right| $} = $\frac{u}{(1 + v)^3} + \frac{uv}{(1 + v)^3} = \frac{u(1 + v)}{(1 + v)^3} = \frac{u}{(1 + v)^2} \; .$
     
     Теперь перепишем функцию в новых координатах.
     
     $f = \frac{(x + y)^2}{x} = \frac{u^2}{u/(1 + v)} = u(1 + v).$
    
     Всё, теперь мы готовы брать интеграл.
     
     $\displaystyle \int\limits_{1/2}^{2} dv \int\limits_{1}^{3} (1 + v) \cdot u \cdot \left| \frac{u}{(1 + v)^2} \right| \, du = [\text{помним, что } u \text{ -- положительный}] = \int\limits_{1/2}^{2} \frac{1}{1 + v} \int\limits_{1}^{3} u^2 \, du = 
     \int\limits_{1/2}^{2} \frac{1}{1 + v} \cdot  \left(\frac{u^3}{3} \; \;  \Bigg|_{1}^{3} \right) = \\
     = \int\limits_{1/2}^{2} \frac{1}{1 + v} \cdot \frac{26}{3} dv = \frac{26}{3} \; \int\limits_{1/2}^{2} \frac{1}{1 + v}dv. $
     
     Замена $\begin{cases} t = 1 + v;\\
     dt  = dv;\\
     v \in [1/2, \; 2] \implies t \in [3/2, \; 3]
     \end{cases}$
     
     $\dots = \frac{26}{3} \int \limits_{3/2}^{3} \frac{1}{t} = \frac{26}{3} \left( \ln t \;  \Bigg|_{3/2}^{3} \right) = \frac{26}{3} \cdot (\ln 3 - \ln (3/2)) = \boxed{\frac{26}{3} \ln 2 } \; .$
     
    
    % \subsection*{Задача 10}
    
    \subsection*{Задача 11}
    
    $\displaystyle \iint\limits_{D} xy \;  dxdy; \; \; D = \{(x,y) \mid x^3 \leq y \leq 2 x^3, \; 2x \leq y^2 \leq 3x\}.$
    
    { \setstretch{2} $\begin{cases}u = \frac{y}{x^3} \implies 1 \leq u \leq 2; \\
    v = \frac{y^2}{x} \implies 2 \leq v \leq 3
    \end{cases}$}
    
    Выражаем $x$ и $y$:
    
    $x = \frac{y^2}{v}.$\\
    
    $\begin{cases}
    y = x^3 \cdot u = \frac{y^6}{v^3} \cdot u \implies \frac{1}{y^5} = \frac{u}{v^3} \implies y = \sqrt[5]{ \frac{v^3}{u}} = v^{3/5} \cdot u^{-1/5};\\
    x = \frac{y^2}{v} =  v^{6/5} \cdot u^{-2/5} \cdot v^{-1} = v^{1/5} \cdot u^{-2/5}
    \end{cases}$
    
    
    Перепишем функцию:
    
    $f = xy = v^{3/5} \cdot v^{1/5} \cdot u^{-1/5} \cdot u^{-2/5}  = v^{4/5} \cdot u^{-3/5}.$
    
    
    Ищем якобиан.
    
     $J = \left| \begin{matrix} x_u & y_u \\ x_v & y_v \end{matrix} \right| = $ {\setstretch{2} $
     \left| \begin{matrix} \left(v^{1/5} \cdot u^{-2/5}\right)_u & \left(v^{3/5} \cdot u^{-1/5} \right)_u \\ \left(v^{1/5} \cdot u^{-2/5} \right)_v & \left(v^{3/5} \cdot u^{-1/5}  \right)_v \end{matrix} \right| =  
     \left| \begin{matrix}   -2/5 \cdot v^{1/5} \cdot u^{-7/5} & -1/5 \cdot v^{3/5} \cdot u^{-6/5}\\ 
     1/5 \cdot v^{-4/5} \cdot u^{-2/5} & 3/5 \cdot v^{-2/5} \cdot u^{-1/5}\\   \end{matrix} \right| $} = \\ $-\frac{6}{25} v^{-1/5} \cdot u^{-8/5} + 
     \frac{1}{25} v^{-1/5} \cdot u^{-8/5} = - \frac{v^{-1/5} \cdot u^{-8/5}}{5}.$
     
     Всё готово для подсчёта интеграла. Заметим, что $J < 0 \implies |J| = -J.$
     
     $I = \displaystyle \int \limits_{1}^{2} du \int \limits_{2}^{3} v^{4/5} \cdot u^{-3/5} \cdot |J| \;  dv = \frac{1}{5} \int\limits_{1}^{2} du \int \limits_{2}^{3} v^{4/5} \cdot u^{-3/5} \cdot v^{-1/5} \cdot u^{-8/5} \; dv= 
     \frac{1}{5} \int\limits_{1}^{2} du \int \limits_{2}^{3} v^{3/5} \cdot u^{-11/5} \; dv = \\ 
     = \frac{1}{5} \int\limits_{1}^{2} u^{-11/5} \; du \left(\frac{8}{5} \cdot v^{5/8} \;  \Bigg|_{2}^{3} \right) = \frac{1}{8} \cdot  (3^{8/5} - 2^{8/5}) \cdot \int\limits_{1}^{2} u^{-11/5} \; du =  
     \frac{1}{8} \cdot  (3^{8/5} - 2^{8/5}) \cdot \left( -\frac{5}{6} u^{-6/5}
     \; \, \Bigg|_{1}^{2} \right)\; du = \\ \boxed{\frac{5}{46} \cdot (3^{8/5} - 2^{8/5}) \cdot (1 - 2^{-6/5})} \; .$
    
    
    
    % \subsection*{Задача 12}
    
    \section*{Задайте в цилиндрических координатах множество $D$, заданное неравенствами в декартовых
    координатах. Пердполагаю функцию $f$ непрерывной на $D$, преобразуйте интеграл в цилиндрических
    координатах к повторному.}
    
    \subsection*{Задача 13}
    
    $x^2 + y^2 \leq 1, \; x^2 + y^2 + z^2 \leq 4.$
    
    $\bullet \; $ Первое неравенство:
    
    $r^2 \leq 1 \implies r \in [0,1].$
    
    $\bullet \; $ Второе неравенство:
    
    $r^2 + z^2 \leq 4 \implies |z| \leq \sqrt{4 - r^2} \implies z \in [-\sqrt{4 - r^2}, \; \sqrt{4 - r^2}].$
    
    Ограничений на  $\varphi$ никаких нет. Можем приступать к записи интеграла. Помним, что в циллиндрических координатах $|J| = r.$
    
    $I = \boxed{\displaystyle \int\limits_{0}^{2 \pi} d \varphi \int\limits_{0}^{1} dr \int\limits_{-\sqrt{4 - r^2}}^{\sqrt{4 - r^2}} f(r \cos \varphi, \; r \sin \varphi, z) \; dz} \; .$
    
    
    
    
    
    % \subsection*{Задача 14}
    
    \section*{Перейдя к цилиндрическим координатам, вычислите интеграл.}
    % \subsection*{Задача 15}
    
    % \subsection*{Задача 16}
    
    \section*{Задайте в сферических координатах множество $D$, заданное неравенствами в декартовых координатах.
    Предполагая функцию $f$ непрерывной на $D$, преобразуйте интеграл в сферических координатах к повторному.}
    
    
    \subsection*{Задача 17}
    
    $x^2 + y^2 + z^2 \leq 1; \; x^2 + y^2 + z^2 \leq 2z.$
    
    \underline{Используем следующую замену:}
    
    $\begin{cases}
    x = r \cos \varphi \sin \Theta;\\
    y = r \sin \varphi \sin \Theta;\\
    z = r  \cos \Theta;\\
    \end{cases}$
    
    $\bullet \; $ Первое неравенство:
    
    $x^2 + y^2 + z^2 \leq 1 \implies r^2 \leq 1 \implies r \in [0,1].$
    
    $\bullet \; $ Второе неравенство:
    
    $x^2 + y^2 + z^2 \leq 2z \iff r^2 \leq 2 r \cos \Theta \iff r \leq 2 \cos \Theta.$
    
    Заметим, что при этих ограничениях: $\; \; \begin{cases} 
    2 \cos \Theta > 1 \implies r \in [0,1];\\
    2 \cos \Theta \leq 1 \implies r \in [0, 2 \cos \Theta]
    \end{cases} \iff 
    \begin{cases} 
    r \in [0,1], \; \, \Theta \in \left[\frac{\pi}{3}, \; \pi \right];\\
    r \in \left[0, 2 \cos \Theta \right], \; \, \Theta \in \left[0, \; \frac{\pi}{3} \right)
    \end{cases}$
    
    $\varphi$ остаётся без ограничений.
    
    Якобиан в сферических координатах (при данной! замене) равен $r^2 \sin \Theta.$
    
    Итог:
    
    $\boxed{\displaystyle \int\limits_{0 }^{2 \pi} d \varphi\left( 
    \int\limits_{0 }^{\pi/3} d \Theta \int\limits_{0 }^{2 \cos \Theta} f \cdot r^2 \sin \Theta \; dr + 
    \int\limits_{\pi/3 }^{\pi} d \Theta \int\limits_{0 }^{1} f \cdot r^2 \sin \Theta \; dr
    \right) } \; .$
    
    
    
    
    % \subsection*{Задача 18}
    
    \section*{Перейдя к сферическим координатам, вычислите интеграл.}
    
    
    \subsection*{Задача 19}
    
    $\displaystyle \iiint\limits_{D} (x^2 + y^2 + z^2) \; dx \, dy\, dz; \; D = \{(x,y,z) \mid x^2 + y^2 + z^2 \leq 1, \; y^2 + z^2 \leq x^2, \; x \geq 0 \}. $
    
    [picture is coming soon]
    
    \underline{Лайфхак}: если мы видим, что с какой-то координатой возникают потенциальные трудности, делаем её в качестве оси, отвечающей за "широту". В данной ситуации нам подойдёт следующая замена:
    
    $\begin{cases}
    x = r \cos \Theta;\\
    y = r \sin  \varphi \sin  \Theta;\\
    z = r \cos  \varphi \sin  \Theta \\
    \end{cases} \; \; \; \; 
    \begin{cases}
    \Theta \in [0,\pi];\\
    \varphi \in [0, 2 \pi)
    \end{cases}$
    
    $\bullet$ Первое неравенство.
    
    $x^2 + y^2 + z^2 \leq 1 \implies r^2 \leq 1 \implies r \in [0,1].$
    
    $\bullet$ Третье неравенство.
    
    $x \geq 0 \implies r \cos \Theta \geq 0 \implies \cos \Theta \geq 0 \implies \Theta \in \left[0, \frac{\pi}{2}\right]. $
    
    (Мы использовали тот факт, что $r \geq 0$).
    
    $\bullet$ Второе неравенство.
    
    $y^2 + z^2 \leq x^2 \iff
    r^2 \sin^2  \varphi \cdot \sin^2  \Theta +
    r^2 \cos^2  \varphi \cdot  \sin^2  \Theta \leq 
    r^2 \cos^2 \Theta \implies r^2 \sin^2 \Theta \leq r^2 \cos^2 \Theta \iff \tg^2 \Theta \leq 1 \implies \Theta \in \left[ 0, \frac{\pi}{4} \right].$
    
    На $\varphi$ ограничений нет. $|J| = r^2 \sin \Theta,$ как и при обычной сферической замене.
    
    Интегрируем:
    
    $\displaystyle \int \limits_{0}^{\pi /4} d\Theta  \int \limits_{0}^{1} dr  \int \limits_{0}^{2 \pi}   r^2 \cdot r^2 \sin \Theta \; d \varphi  = 2 \pi \int \limits_{0}^{\pi /4} \sin \Theta \;  d\Theta  \int \limits_{0}^{1}  r^4   \; dr = 2 \pi \cdot  \left( -\cos \Theta \;  \Bigg|_{0}^{\pi /4} \right) \cdot \left( \frac{r^5}{5} \; \Bigg|_0^1 \right) = \boxed{\frac{2\pi}{5} \cdot \left(1 - \frac{1}{\sqrt{2}} \right)} \; .$
    
    
    
    
    % \subsection*{Задача 20}
    
\end{document}
