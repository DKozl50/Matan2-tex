\documentclass[a4paper, fleqn]{article}
\usepackage{header}

\title{Семинарский лист 2.5}
\author{
    % Александр Богданов   \\ \href{https://t.me/SphericalPotatoInVacuum}{Telegram} \and
    % Алиса Вернигор       \\ \href{https://t.me/allisyonok}{Telegram} \and
    % Анастасия Григорьева \\ \href{https://t.me/weifoll}{Telegram} \and
    % Василий Шныпко       \\ \href{https://t.me/yourvash}{Telegram} \and
    % Данил Казанцев       \\ \href{https://t.me/vserosbuybuy}{Telegram} \and
    % Денис Козлов         \\ \href{https://t.me/DKozl50}{Telegram} \and
    % Елизавета Орешонок   \\ \href{https://t.me/eaoresh}{Telegram} \and
    % Ира Голобородько     \\ \href{https://t.me/Ira4kgl}{Telegram} \and
    % Иван Пешехонов       \\ \href{https://t.me/JohanDDC}{Telegram} \and
    % Иван Добросовестнов  \\ \href{https://t.me/ivankot13}{Telegram} \and
    % Настя Городилова     \\ \href{https://t.me/nastygorodi}{Telegram} \and
    % Никита Насонков      \\ \href{https://t.me/nnv_nick}{Telegram} \and
    % Сергей Лоптев        \\ \href{https://t.me/beast_sl}{Telegram}
}

\date{Версия от {\ddmmyyyydate\today} \currenttime}

\begin{document}
    \maketitle
    
    \section*{Задайте в полярных координатах множество $D$, заданное неравенствами в декартовых координатах.
    Предполагая функцию $f$ непрерывной на $D$, преобразуйте интеграл в полярных координатах к повторному.}
    \subsection*{Задача 1}
    \begin{flalign*}
        & x^2+y^2 \le 2x \\[5 pt]
        & x = r \cos \varphi, \; y = r \sin \varphi \Rightarrow \\
        & \Rightarrow  D := r^2 \le 2\, r \cos \varphi \Leftrightarrow r (r - 2 \cos \varphi) \le 0 
        \Leftrightarrow r \in \left[0; 2 \cos \varphi\right], \cos \varphi \ge 0 \\
        & \iint\limits_D f(r \cos \varphi, r \sin \varphi)\, d\varphi\,dr
        = \int\limits_{-\pi/2}^{\pi/2} d\varphi \int\limits_{0}^{2 \cos \varphi} f(r \cos \varphi, r \sin \varphi)\, r \,dr = \\
        & \left\{\, r^2 \le 2\, r \cos \varphi \Leftrightarrow \cos \varphi \ge \frac{r}2 
        \Leftrightarrow \varphi \in \left[ -\arccos \frac{r}2; \arccos \frac{r}2 \right] \,\right\} \\
        & = \int\limits_{0}^{2} r\, dr \int\limits_{-\arccos \frac{r}2}^{\arccos \frac{r}2} f(r \cos \varphi, r \sin \varphi)\, d\varphi 
    \end{flalign*}
    
    \subsection*{Задача 2}
    \begin{flalign*}
        & (x-1)^2+y^2 \le 1, \; x^2 + y^2 \ge 1 \\[5 pt]
        & x = r \cos \varphi, \; y = r \sin \varphi \Rightarrow \\
        & \Rightarrow  D := \left\{\begin{array}{lll} r^2 &\le& 2\,r \cos \varphi, \\ r^2 &\ge& 1 \end{array}\right. 
        \Leftrightarrow \left\{\begin{array}{rcl} 
            r &\in& \left[ 0; 2 \cos \varphi \right], \\  
            \cos \varphi &\ge& 0, \\ 
            r &\ge& 1 
        \end{array}\right. \Leftrightarrow \left\{\begin{array}{rcl} 
            r &\in& \left[ 1; 2 \cos \varphi \right], \\  
            \cos \varphi &\ge& 1/2, \\ 
        \end{array}\right. \\
        & \cos \varphi \ge 1/2 \Leftrightarrow \varphi \in \left[ -\dfrac{\pi}3; \dfrac{\pi}3 \right] \\
        & \iint\limits_D f(r \cos \varphi, r \sin \varphi)\, d\varphi\,dr
        = \int\limits_{-\pi/3}^{\pi/3} d\varphi \int\limits_{1}^{2 \cos \varphi} f(r \cos \varphi, r \sin \varphi)\, r \,dr = \\
        & \left\{\, r^2 \le 2\,r \cos \varphi \Leftrightarrow \cos \varphi \ge \frac{r}2 
        \Leftrightarrow \varphi \in \left[ -\arccos \frac{r}2; \arccos \frac{r}2 \right] \,\right\} \\
        & = \int\limits_{1}^{2} r\, dr \int\limits_{-\arccos \frac{r}2}^{\arccos \frac{r}2} f(r \cos \varphi, r \sin \varphi)\, d\varphi 
    \end{flalign*}
    
    % \subsection*{Задача 3}
    
    % \subsection*{Задача 4}
    
    % \subsection*{Задача 5}
    
    % \subsection*{Задача 6}
    
    \section*{Перейдя к полярным координатам, вычислите интеграл.}
    % \subsection*{Задача 7}
    
    % \subsection*{Задача 8}
    
    \section*{Перейдите к переменным, в которых область интегрирования имеет вид прямоугольника, 
    и вычислите интеграл.}
    % \subsection*{Задача 9}
    
    % \subsection*{Задача 10}
    
    % \subsection*{Задача 11}
    
    % \subsection*{Задача 12}
    
    \section*{Задайте в цилиндрических координатах множество $D$, заданное неравенствами в декартовых
    координатах. Пердполагаю функцию $f$ непрерывной на $D$, преобразуйте интеграл в цилиндрических
    координатах к повторному.}
    % \subsection*{Задача 13}
    
    % \subsection*{Задача 14}
    
    \section*{Перейдя к цилиндрическим координатам, вычислите интеграл.}
    % \subsection*{Задача 15}
    
    % \subsection*{Задача 16}
    
    \section*{Задайте в сферических координатах множество $D$, заданное неравенствами в декартовых координатах.
    Предполагая функцию $f$ непрерывной на $D$, преобразуйте интеграл в сферических координатах к повторному.}
    % \subsection*{Задача 17}
    
    % \subsection*{Задача 18}
    
    \section*{Перейдя к сферическим координатам, вычислите интеграл.}
    % \subsection*{Задача 19}
    
    % \subsection*{Задача 20}
    
\end{document}
