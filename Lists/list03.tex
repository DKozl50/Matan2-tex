\documentclass[a4paper,fleqn]{article}
\usepackage{header}

\title{Семинарский лист 1337}
\author{
    Александр Богданов \\ \href{https://t.me/SphericalPotatoInVacuum}{Telegram} \and
    Алиса Вернигор     \\ \href{https://t.me/allisyonok}{Telegram} \and
    Василий Шныпко     \\ \href{https://t.me/yourvash}{Telegram} \and
    Денис Козлов       \\ \href{https://t.me/DKozl50}{Telegram} \and
    Иван Пешехонов     \\ \href{https://t.me/JohanDDC}{Telegram}
}

\date{Версия от {\ddmmyyyydate\today} \currenttime}

\begin{document}
    \maketitle
    \section*{Применяя признак Вейрштрасса, покажите, что ряд сходится абсолютно.}
    %Task 1 

    %Task 2

    %Task 3
    
    \section*{Применяя признак Лейбница, покажите, что ряд сходится.}
    %Task 4

    %Task 5

    %Task 6
    \subsection*{6}
    \begin{flalign*}
        & \sum_{n=1}^{\infty} \frac{{(-1)}^n \sqrt{n}}{3n - 2} \;\;\;\;\;\;
        a_n = \frac{{(-1)}^n \sqrt{n}}{3n - 2} \text{ очевидно знакочередующийся} \;\;\;\;\;\; 
        |a_n| = \frac{\sqrt{n}}{3n - 2} \\
        & \lim_{n \to \infty} \left| a_n \right| = 0 \;\;\;\;\;\; 
        |a_n|' = \frac{\frac{3n-2}{2\sqrt{n}} - 3\sqrt{n}}{{(3n-2)}^2} = 
        \frac{3n - 2 - 6n}{2\sqrt{n}{(3n-2)}^2} = - \frac{3n + 2}{2 \sqrt{n} {(3n - 2)}^2} < 0 \; \forall n > 0 
        \implies \text{ монотонно убывает} \\
        & \text{По признаку Лейбница ряд сходится} 
    \end{flalign*}

    \section*{Применяя группировку членов постоянного знака, покажите, что ряд расходится.}
    %Task 7

    %Task 8
    \subsection*{8}
    \begin{flalign*}
        & \sum_{n=1}^{\infty} \frac{{(-1)}^{\left[ \sqrt{n} \right]}}{\sqrt{n} + 2} = \sum_{k=1}^{\infty} A_k \\
        & \left[ \sqrt{n} \right] = k \implies k \leq \sqrt{n} < k + 1 \implies
        k^2 \leq n < {(k+1)}^2 \;\;\;\;\;\; 
        A_k = {(-1)}^k \sum_{n = k^2}^{{(k+1)}^2 - 1} \frac{1}{\sqrt{n} + 2} \\
        & |A_k| = \sum_{n = k^2}^{{(k+1)}^2 - 1} \frac{1}{\sqrt{n} + 2} \geq 
        \frac{{(k+1)}^2 - 1 - k^2 + 1 }{\sqrt{{(k + 1)}^2 - 1} + 2} \geq 
        \frac{2k + 1}{\sqrt{{(k+1)}^2} + 2} = \frac{2k + 1}{k + 3} \to 2 \neq 0 \implies \\
        & \implies \text{ не выполняется необходимое условие сходимости.}
    \end{flalign*}

    %Task 9

    %Task 10

    %Task 11
    
    \section*{Применяя признак Дирихле или Абеля, покажите, что ряд сходится.}
    %Task 12

    %Task 13

    %Task 14

    %Task 15

    %Task 16

    %Task 17
    
    \section*{Исследуйте ряд на сходимость и абсолютную сходимость, используя асимптотику общего члена.}
    %Task 18
    \subsection*{18}
    \begin{flalign*}
        & \sum_{n=1}^{\infty} \sin \left( \pi \sqrt{n^2 + 2} \right) \\
        & \sin \left( \pi \sqrt{n^2 + 2} \right) = \sin \left( \pi n \sqrt{1 + \frac{2}{n^2} } \right) =
        \sin \left( \pi n \left( 1 + \frac{2}{2n^2} + \mathcal{O} \left( \frac{1}{n^4} \right) \right)  \right) = 
        \sin \left( \pi n + \frac{1}{n} + \mathcal{O} \left( \frac{1}{n^3} \right) \right) = \\
        & = {(-1)}^{n} \sin \left( \frac{\pi}{n} + \mathcal{O} \left( \frac{1}{n^3} \right) \right) =
        {(-1)}^{n} \frac{\pi}{n} + \mathcal{O} \left( \frac{1}{n^3} \right) \\
        & \sum_{n=1}^{\infty} \Bigg( {(-1)}^{n} \frac{\pi}{n} + \mathcal{O} \left( \frac{1}{n^3} \right) \Bigg) 
        \text{ сходится по Лейбницу}
    \end{flalign*}

    %Task 19

    %Task 20

    %Task 21

    %Task 22

    %Task 23
    
    \section*{Вычислите произведение рядов.}
    %Task 24

    %Task 25

\end{document}
