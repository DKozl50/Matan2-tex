\documentclass[a4paper,fleqn]{article}
\usepackage{header}

\title{Семинарский лист 1337}
\author{
	Александр Богданов \\ \href{https://t.me/SphericalPotatoInVacuum}{Telegram} \and
	Алиса Вернигор     \\ \href{https://t.me/allisyonok}{Telegram} \and
	Василий Шныпко     \\ \href{https://t.me/yourvash}{Telegram} \and
	Денис Козлов       \\ \href{https://t.me/DKozl50}{Telegram} \and
	Иван Пешехонов     \\ \href{https://t.me/JohanDDC}{Telegram}\and
	Иван Добросовестнов \\ \href{https://t.me/ivankot13}{Telegram}
}

\date{Версия от {\ddmmyyyydate\today} \currenttime}

\begin{document}
	\maketitle
	\section*{Применяя признак Вейрштрасса, покажите, что ряд сходится абсолютно.}
	%Task 1 
	\subsection*{Задача 1}
		 $ \sum_{n=1}^{\infty} \frac{(-1)^{n} \cos n^{2}}{\sqrt{n^{3}+3}} $ --- сходится по признаку сравнения
		
		$ |a_n| = \frac{|\cos n^{2}|}{\sqrt{n^{3}+3}} \leq \dfrac{1}{\sqrt{n^{3}+3}} \sim \dfrac{1}{\sqrt{n^3}} = \dfrac{1}{n^{3/2}} $ --- сходящийся ряд
	
	%Task 2
	\subsection*{Задача 2}
		$ \sum_{n=1}^{\infty} \frac{(-1)^{n} n^{\ln n}}{2^{\sqrt{n}}} $
		$\; \; \; \; \;  |a_n| = \dfrac{n^{\ln n}}{2^{\sqrt{n}}} = \exp \underbrace{\left[\ln^2 n - \sqrt{n}\ln 2\right]}_{b_n}  = e^{b_n}$
		
		Рассмотрим $ b_n $.
		Заметим, что $ \ln n  = o(n^p), \; n \to \inf, \; p>0$. В нашем случае $ p_n = \dfrac{1}{2}, \; \; \; \ln^2n = o(\sqrt{n}) \Rightarrow b_n \sim -\sqrt{n}\ln 2$ 
		
		Оценим $ b_n $   --- $ \; \sqrt{n}\ln2 \leq 2 \ln n \Rightarrow -\sqrt{n}\ln 2 \leq 2\ln n$
		
		Подставим эту оценку для $ b_n $ --- $\; e^{b_n} \leq e^{-2\ln n}  = \dfrac{1}{n^2}$ 
		
		$ |a_n| $ мажорируется $ \dfrac{1}{n^2} \Rightarrow \;$ ряд $ \sum_{i = 1}^{\infty} |a_n|$ сходится по признаку сравнения 
	%Task 3
	\subsection*{Задача 3}
		$ \sum_{n=1}^{\infty} \frac{\left(2^{n}+3^{n}\right) \sin n}{2^{n}+n^{2} \cdot 3^{n}} $
	\section*{Применяя признак Лейбница, покажите, что ряд сходится.}
	%Task 4
	
	%Task 5
	
	%Task 6
	
	\section*{Применяя группировку членов постоянного знака, покажите, что ряд расходится.}
	%Task 7
	
	%Task 8
	
	%Task 9
	
	%Task 10
	
	%Task 11
	
	\section*{Применяя признак Дирихле или Абеля, покажите, что ряд сходится.}
	%Task 12
	
	%Task 13
	
	%Task 14
	
	%Task 15
	
	%Task 16
	
	%Task 17
	
	\section*{Исследуйте ряд на сходимость и абсолютную сходимость, используя асимптотику общего члена.}
	%Task 18
	
	%Task 19
	
	%Task 20
	
	%Task 21
	
	%Task 22
	
	%Task 23
	
	\section*{Вычислите произведение рядов.}
	%Task 24
	
	%Task 25
	
\end{document}

